\documentclass{fiwthesis}

% ========
%  Pakete
% ========

\usepackage{textgreek}           % griechische Buchstaben außerhalb des Math-Mode
\usepackage{amsmath}             % zentrierte Formeln
\usepackage{amssymb}             % erweiterter Formelsatz mathem. Symbole

\usepackage{boldline}            % breitere Linien in Tabellen
\usepackage{booktabs}            % typographisch richtige Tabellen setzen
\usepackage{tabularx}            % Erweiterte Tabellendarstellung
\usepackage{multirow}            % Spalte über mehrere Zeilen oder Spalten ausdehnen
\usepackage{xltabular}           % Zeilenumbrüche in tabularx erlauben

\usepackage{graphicx}            % ermöglicht das Einbinden von Grafiken
\usepackage{subcaption}          % mehrere Bilder in einem Bild
\usepackage{pgfplots}            % Grafiken erzeugen
\usepackage{smartdiagram}        % schnelle und einfache Grafiken

% ===========
%  Metadaten
% ===========

\thesis{Expose zur Master-Thesis}
\title{Homomorphe Post-Quanten Kryptographie - Bewertung von Homomorphen Verschlüsselungsalgorithmen auf Basis von CRYSTALS - Kyber}
\author{Pascal Stehling}
\matrnr{455051}
\bdate{18.12.1997}
\bcity{Wismar}
\supervisor{Prof.~Dr.-Ing.~habil.~Andreas Ahrens}
% \secsupervisor{ZWEITBETREUER}
\keywords{Logik, Mathematik}

% Metadaten in die PDF-Datei schreiben
\makepdfmetadata

% ===============
%  Präambel
% ===============

% PGF Kompatibilitätseinstellung
\pgfplotsset{width=0.95\textwidth,compat=newest}

% % Bibliographie einbinden
\bibliography{quellen}

% % Glossar einbinden
% % this cant be removed, for some reason the build breaks
\newdualentry{dos}% label
{DoS}% short form
{Denial of Service}% long form
{Ein Denial of Service (im Deutschen: Dienstverweigerung) ist ein Angriffe auf Computer- oder Netzwerksysteme, wobei das Zielsystem durch Überlastung oder durch andere Mittel außer Betrieb gesetzt wird}% description


% % Abkürzungen einbinden
% \input{verzeichnisse/abkuerzungen}

% % Symbole einbinden
% \input{verzeichnisse/symbole}

% % Glossar- und Abkürzungsverzeichniserstellung
\makeglossaries{}

% Index erzeugen
\makeindex[
  intoc=true,
  title=Index,
  columns=2]{}
\indexsetup{headers={\indexname}{\indexname}}

% ===============
%  Eigene Makros
% ===============

\newcommand*{\code}[1]{\texttt{#1}}

\begin{document}

% Titelseite
\maketitle

% ==========
%  Textteil
% ==========

\chapter{Problemstellung}
\label{Problemstellung}

Im Jahr 2009 veröffentlichte Craig Gentry in seiner Doktorarbeit \cite{Gentry2009AFH} den ersten voll Homomorphen Algorithmus. Dieser erlaubte es erstmals beliebige viele Berechnungen auf verschlüsselten Nachrichten durchzuführen, ohne sie dafür entschlüsseln zu müssen. Als Mathematische Grundlage wurde ein Ansatz basierend auf \glqq Ideal Lattice\grqq{} gewählt. Seit der Entdeckung des ersten voll Homomorphen Systems von Craig Gentry, wurden weitere solcher Algorithmen entwickelt (\cite{BGV}, \cite{FV}, \cite{GSW}), welche sich jedoch das Learning with Errors (LWE) und Ring-LWE (RLWE) Problem als Mathematische Grundlage nutzen.

Im Juli 2022 Veröffentlichte das Amerikanische National Institute of Standards and Technology (kurz NIST) die erste Gruppe an Gewinnern ihres Wettbewerbs für quantensichere Algorithmen \cite{nistAnouncement}. Der Gewinner für die generelle asymmetrische Verschlüsselung war dabei der CRYSTALS-Kyber Algorithmus \cite{crystalsKyberWeb}, welcher auf Module-LWE basiert. Dies ist eine Erweiterung des Ring-LWE Methode.

Dadurch stellt sich die Frage, ob es möglich ist, ein Homomorphes Kryptosystem um des Module-LWE verfahren zu erweitern und wie er sich mit den bereits etablierten Algorithmen vergleicht.
\chapter{Zielsetzung}
\label{Zielsetzung}

Das Ziel dieser Master Thesis ist der Versuch ein Homomorphes Kryptosystem auf Basis von Module-LWE zu konstruieren. Dabei kann versucht werden verschiedene Schemas wie \cite{BGV} so zu erweitern, das sie dies ermöglichen. Anschließend soll eine praktischer Vergleich zwischen dem neuen, sowie bereits existierenden Verfahren durchgeführt werden.

Um dieses Ziel zu erreichen wird es zuerst eine Einführung in die Verwendete Mathematik, sowie einen Einstieg in das Thema der Gitterbasierten Verschlüsselung vorgenommen. Damit werden die 3 Ansätze der Gitterbasierten Kryptographie (LWE, RLWE, MLWE) genauer Beschrieben und voneinander abgegrenzt. Anschließend werden die Eigenschaften Homomorpher Systeme analysiert und auf die Verschiedenen Ansätze angewandt, sodass mehrere Voll homomorphe Kryptosysteme entstehen. Um diese Vergleichen zu könne, werden diverse Vergleichskriterien definiert und ein Testaufbau erstellt. Nach den Durchführungen der Tests, werden die Algorithmen mithilfe der vorher definierten Vergleichskriterien miteinander vergleichen.

\chapter{Abgrenzung}
\label{Abgrenzung}

Ziel dieser Arbeit soll es nicht sein ein komplett neues Kryptosystem zu entwickeln, sondern es soll der Versuch unternommen werden, bereits vorhandene Systeme auf Basis des (Ring-)LWE Verfahrens um das Module-LWE zu erweitern. Anschließend sollen die Vorhandenen und die neuen Systeme bei einem Praktischen Test miteinander vergleichen werden.
\chapter{Stand der Forschung}
\label{StandDerForschung}
Stand der Forschung
% \chapter{Vorgehensweisen und Methoden}
\label{VorgehensweisenUndMethoden}
Vorgehensweisen und Methoden
\chapter{Gliederung}
\label{Gliederung}
Gliederung
\chapter{Zeitplan}
\label{Zeitplan}
Zeitplan


% ===============
%  Verzeichnisse
% ===============

% Verzeichnisse mit einzeiligem Zeilenabstand
\singlespacing

% Literaturverzeichnis
\listofreferences

% falls ein anderer Glossar-Stil genutzt wird und die zweite Spalte zu schmal ist:
% \setlength{\glsdescwidth}{0.8\linewidth}

% Glossar einfügen
\printglossary

% Index einfügen
\printindex

% wieder auf 1½-fachen Zeilenabstand umschalten
\normalspacing

\end{document}
