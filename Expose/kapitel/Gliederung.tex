\chapter{Gliederung}
\label{Grob-Gliederung}

\begin{enumerate}
  \item Einleitung, Motivation und Zielstellung
  \item Mathematische Grundlagen
        \begin{enumerate}
          \item Gitterbasierte Kryptographie/Gitter Problem
          \item Polynomring
          \item Beschreibung von LWE, RLWE und MLWE
        \end{enumerate}
  \item Beschreibung von voll homomorphen Kryptosystemen
        \begin{enumerate}
          \item Eigenschaften homomorpher Kryptosysteme
          \item Beschreibung voll homomorpher Kryptosysteme auf Basis von LWE und RLWE
          \item Erweiterung eines FHE Kryprosystems um MLWE
          \item Definieren von von Kriterien zum vergleichen von FHE-Kryptosystemen
        \end{enumerate}
  \item Praktischer Vergleich unterschiedlicher homomorpher Kryptosysteme
        \begin{enumerate}
          \item Erstellen eines Testaufbau auf basis der definierten Vergleichskriterien
          \item Durchführung der Tests
          \item Ergebnisse und Auswertung
        \end{enumerate}
  \item Fazit und Ausblick
\end{enumerate}

% 1. Einleitung, Motivation und Zielstellung
% 2. Mathematische Grundlagen
%     1. Gitterbasierte Kryptographie/Gitter Problem
%     2. Polynomring
% 3. Verknüpfung von quanten resistenten ansätzen mit dem homomorphismus
%     1. Einführung
%     2. Quantenresisitente Ansätze:der Kryptpgraphie
%         - LWE
%         - RLWE
%         - MLWE
%     3. Eigenschaften Homomorpher Kryptosysteme
%         - Teilweise und Voll Homomorphe Kryptosysteme
%         - Bootstrapping
%     4. Anwendung der Eigenschaften auf LWE, RLWE, MLWE
% 4. Proof of Concept
%     1. Vergleichskriterien:
%         - Vergleich Schlüsselgröße bei gleicher Sicherheit
%         - Vergleich der Laufzeit für: Schlüsselgenerierung, Verschlüsselung, Entschlüsselung, Addition, Multiplikation
%         - Vergleich der Fehlerate bei: Addition, Multiplikation
%     2. Aufbau des Tests
%         - einheitliche implementierung der Algorithmen (LWE, RLWE, MLWE), um gliches/ähnliches Laufzeitverhaltenzu zu erhalten
%             - Der Geschwindigkeitsunterschied durch verschiedenen optimierte Implementierungen muss minimiert werden, ansonsten schelchte vergleichbarkeit
%         - einheitlicher Testrahmen für die verschiedenen Algorithmen
%             - WICHTIG: gleichen Zahlen in der selben Reihenfolge für die verschiedenen operationen, um vergleichbarkeit zu verbessern
%         - Testen der oben gennaten Funktionen nach ihren Eigenschaften
%     3. Ergebnisse und Auswertung
%         - Vergleich der Ergebnnisse untereinander
%         - Vergleich der Schlüsselgröße mit bereits existierenden Algorithmen
%         - Vergleich der Laufzeit von Schlüsselgenerierung, Verschlüsselung, Entschlüsselung mit existierenden Algorithmen
%         - Überprüfung der Laufzeit mit Blick auf Fehlerrate von Addition, Multiplikation auf praktische Anwendbarkeit
% 5. Diskussion Praktischer Probleme