\chapter{Zielsetzung}
\label{Zielsetzung}

Das Ziel dieser Master Thesis ist der Versuch ein Homomorphes Kryptosystem auf Basis von Module-LWE zu konstruieren. Dabei kann versucht werden verschiedene Schemas wie beispielsweise BGV \cite{BGV} so zu erweitern oder abzuändern, dass dies möglichen ist. Anschließend soll eine praktischer Vergleich zwischen dem neuen, sowie bereits existierenden Verfahren durchgeführt werden.

Um dieses Ziel zu erreichen wird es zuerst eine Einführung in die Verwendete Mathematik, sowie einen Einstieg in das Thema der Gitterbasierten Verschlüsselung vorgenommen. Dabei soll das mathematische Probleme des \glqq Lernen mit Fehlern (LWE)\grqq{} genauer Beschrieben und von den Erweiterungen Ring-LWE (RLWE) und Module-LWE (MLWE) abgegrenzt werden. Anschließend werden die Eigenschaften Homomorpher Systeme analysiert und auf die Verschiedenen Probleme angewandt, sodass mehrere Voll homomorphe Kryptosysteme entstehen. Um diese Vergleichen zu könne, werden diverse Vergleichskriterien definiert und ein Testaufbau erstellt. Nach den Durchführungen der Tests, werden die Algorithmen mithilfe der vorher definierten Vergleichskriterien miteinander vergleichen.
