\chapter{Problemstellung}
\label{Problemstellung}

Im Jahr 2009 veröffentlichte Craig Gentry in seiner Doktorarbeit \cite{Gentry2009AFH} den ersten voll Homomorphen Algorithmus. Dieser erlaubte erst erstmals beliebige viele Berechnungen auf verschlüsselten Nachrichten durchzuführen, ohne sie dafür entschlüsseln zu müssen. Als Mathematische Grundlage wurde ein Ansatz basierend auf \glqq Ideal Lattice\grqq{} gewählt. Seit der Entdeckung des ersten voll Homomorphen Systems von Craig Gentry, wurden weitere solcher Algorithmen entwickelt (\cite{BGV}, \cite{FV}, \cite{GSW}), welche sich jedoch das Learning with Errors (LWE) und Ring-LWE (RLWE) Problem als Mathematische Grundlage nutzen.

Im Juli 2022 Veröffentlichte das Amerikanische National Institute of Standards and Technology (kurz NIST) die erste Gruppe an Gewinnern ihres Wettbewerbs für quantensichere Algorithmen \cite{nistAnouncement}. Der Gewinner für die generelle asymmetrische Verschlüsselung war dabei der CRYSTALS-Kyber Algorithmus \cite{crystalsKyberWeb}, welcher auf Module-LWE basiert. Dies ist eine Erweiterung des Ring-LWE Methode.

Dadurch stellt sich die Frage, ob es möglich ist, ein Homomorphes Kryptosystem um des Module-LWE verfahren zu erweitern und wie er sich mit den bereits etablierten Algorithmen vergleicht.