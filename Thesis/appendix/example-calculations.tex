\chapter{Example Calculations}

\section{Example Multidimensional Ring Calculation}
\label{app:ExampleMultiRingCalc}

Consider the ring $R = \mathbb{Z}_5[x]/(x^3+1)$ and 
$$
  f =\begin{bmatrix}
    1+2x+3x^2 & 2+3x+4x^2 \\
    3+4x+x^2  & 1+3x+4x^2 \\
  \end{bmatrix} \in R^{2\times 2}
$$

$$
  g = \begin{bmatrix}
    1+x+x^2   \\
    2+2x+2x^2 \\
  \end{bmatrix} \in R^2
$$

\begin{align*}
  f \cdot g & = {
  \begin{bmatrix}
    1+2x+3x^2 & 2+3x+4x^2 \\
    3+4x+x^2  & 1+3x+4x^2 \\
  \end{bmatrix}
  \cdot
  \begin{bmatrix}
    1+x+x^2   \\
    2+2x+2x^2 \\
  \end{bmatrix}
  }               \\
            & = {
  \begin{bmatrix}
    \begin{bmatrix}
      1 & -3 & -2 \\
      2 & 1  & -3 \\
      3 & 2  & 1  \\
    \end{bmatrix} & 
    \begin{bmatrix}
      2 & -4 & -3 \\
      3 & 2  & -4 \\
      4 & 3  & 2  \\
    \end{bmatrix}   \\
    \begin{bmatrix}
      3 & -1 & -4 \\
      4 & 3  & -1 \\
      1 & 4  & 3  \\
    \end{bmatrix} & 
    \begin{bmatrix}
      1 & -4 & -3 \\
      3 & 1  & -4 \\
      4 & 3  & 1  \\
    \end{bmatrix}   \\
  \end{bmatrix}
  \cdot
  \begin{bmatrix}
    \begin{bmatrix}
      1 \\
      1 \\
      1 \\
    \end{bmatrix} \\
    \begin{bmatrix}
      2 \\
      2 \\
      2 \\
    \end{bmatrix} \\
  \end{bmatrix}
  }               \\
            & = {
  \begin{bmatrix}
    1 & -3 & -2 & 2 & -4 & -3 \\
    2 & 1  & -3 & 3 & 2  & -4 \\
    3 & 2  & 1  & 4 & 3  & 2  \\
    3 & -1 & -4 & 1 & -4 & -3 \\
    4 & 3  & -1 & 3 & 1  & -4 \\
    1 & 4  & 3  & 4 & 3  & 1 
  \end{bmatrix}
  \cdot
  \begin{bmatrix}
    1 \\
    1 \\
    1 \\
    2 \\
    2 \\
    2 
  \end{bmatrix}
  }               \\
            & = {
  \begin{bmatrix}
    -14 \\
    2   \\
    24  \\
    -14 \\
    6   \\
    24
  \end{bmatrix}
  \mod 5
  }
  = {
  \begin{bmatrix}
    1 \\
    2 \\
    4 \\
    1 \\
    1 \\
    4
  \end{bmatrix}
  }
  = {
  \begin{bmatrix}
    \begin{bmatrix}
      1 \\
      2 \\
      4
    \end{bmatrix} \\
    \begin{bmatrix}
      1 \\
      1 \\
      4
    \end{bmatrix} \\
  \end{bmatrix}
  }               \\
            & = {
  \begin{bmatrix}
    1+2x+4x^2 \\
    1+1x+4x^2
  \end{bmatrix}
  }
\end{align*}


\section{Plain LWE}
\label{app:PlainLweCalc}
The following calculations should show the working of the Plain LWE encryption for the algorithms \ref{alg: SampleLweKeyGen} to \ref{alg: SampleLweDecryption}. The ring used for this calculations is defined as $R=\mathbb{Z}_{100}$ and $n=2$ for the dimensions. Starting first with the key generation:


\begin{align*}
  s  & = \begin{bmatrix}1 \\ 2 \end{bmatrix}
  A  = \begin{bmatrix}56 & 77 \\ 29 & 59 \end{bmatrix}
  e  = \begin{bmatrix}99 \\ 1 \end{bmatrix}       \\
  b  & = As+e                                     \\
     & = \begin{bmatrix}
           56 & 77 \\
           29 & 59
         \end{bmatrix}
  \cdot
  \begin{bmatrix}
    1 \\
    2
  \end{bmatrix}
  +
  \begin{bmatrix}
    99 \\ 
    1 
  \end{bmatrix}
  \\
     & = 1
  \cdot
  \begin{bmatrix}
    56 \\
    29
  \end{bmatrix}
  + 2 
  \cdot
  \begin{bmatrix}
    77 \\ 
    59 
  \end{bmatrix}
  + 
  \begin{bmatrix}
    99 \\ 
    1 
  \end{bmatrix}                                  \\
     & = \begin{bmatrix}
           309 \\
           148\end{bmatrix}_{100}                   \\
     & = \begin{bmatrix}
           9 \\ 
           48 
         \end{bmatrix}                           \\
  sk & = s =  \begin{bmatrix}1 \\ 2 \end{bmatrix} \\
  pk & = (A, b) = \left (
  \begin{bmatrix}
      56 & 77  \\
      29 & 59 
    \end{bmatrix},
  \begin{bmatrix}
      9 \\
      48 
    \end{bmatrix} \right )                          \\
\end{align*}

With the secret and public key generated, the next step is to encrypt the message $m=1$ with the public key $pk$

\begin{align*}
  r       & = \begin{bmatrix}0 \\ 1 \end{bmatrix}
  e_1 = \begin{bmatrix}2 \\ 0 \end{bmatrix}
  e_2 = 99                                                          \\
  \\
  u       & = A^T \cdot r + e_1                                     \\
          & = \begin{bmatrix}
                56 & 77 \\
                29 & 59
              \end{bmatrix}^T
  \cdot
  \begin{bmatrix}
    0 \\
    1
  \end{bmatrix}
  +
  \begin{bmatrix}
    2 \\
    0
  \end{bmatrix}                                                    \\
          & = \begin{bmatrix}
                56 & 29 
                \\ 77 & 59 
              \end{bmatrix}
  \cdot 
  \begin{bmatrix}
    0 \\
    1 
  \end{bmatrix}
  +
  \begin{bmatrix}
    2 \\
    0
  \end{bmatrix}                                                    \\
          & = 0\cdot
  \begin{bmatrix}
    56 \\
    77
  \end{bmatrix}
  + 1 \cdot 
  \begin{bmatrix}
    29 \\
    59
  \end{bmatrix}
  +
  \begin{bmatrix}
    2 \\
    0
  \end{bmatrix}                                                    \\
          & = \begin{bmatrix}
                31 \\ 
                61 
              \end{bmatrix}_{100}
  = 
  \begin{bmatrix}
    31 \\
    61
  \end{bmatrix}                                                    \\
  \\
  v       & = b^T \cdot r + e_2 + (m*\left\lfloor q/2\right\rfloor) \\
          & =\begin{bmatrix}
               9 \\
               48
             \end{bmatrix}^T
  \cdot
  \begin{bmatrix}
    0 \\
    1
  \end{bmatrix}
  + 99 + 1 \cdot \left\lfloor 100/2\right\rfloor                    \\
          & =\begin{bmatrix}
               9 & 48
             \end{bmatrix}
  \cdot
  \begin{bmatrix}
    0 \\ 
    1 
  \end{bmatrix}
  + 99 + 50                                                         \\
          & = 9 \cdot 0 +48 \cdot 1 + 99 + 50                       \\
          & = 197_{100}                                             \\
          & = 97                                                    \\
  m_{enc} & = (u, v) = \left (
  \begin{bmatrix}
      31 \\
      61
    \end{bmatrix}, 97   \right )                                      \\
\end{align*}

Now the encrypted message $M_{enc}$ can be decrypted again, using the secret key $sk$:
\begin{align*}
  m & = \left\lfloor \frac{1}{\left\lfloor q/2\right\rfloor} *(v-s^T \cdot u)\right\rceil _2 \\
    & = \left\lfloor \frac{1}{\left\lfloor 100/2\right\rfloor} * \left (97-
  \begin{bmatrix}
      1 \\
      2
    \end{bmatrix}^T
  \cdot
  \begin{bmatrix}
      31 \\
      61
    \end{bmatrix} \right )\right\rceil _2                                                      \\
    & = \left\lfloor \frac{1}{50} * \left (97-
  \begin{bmatrix}
      1 & 2 
    \end{bmatrix}
  \cdot 
  \begin{bmatrix}
      31 \\ 
      61
    \end{bmatrix}\right )\right\rceil _2                                                       \\
    & = \left\lfloor \frac{1}{50} * (97-(31 \cdot 1 + 61 \cdot 2))\right\rceil _2            \\
    & = \left\lfloor \frac{1}{50} * (-56)_{100}\right\rceil _2                               \\
    & = \left\lfloor \frac{1}{50} * 44\right\rceil _2                                        \\
    & = \left\lfloor \frac{44}{50}\right\rceil _2  = \left\lfloor 0.88\right\rceil _2        \\
    & = 1                                                                                    \\
\end{align*}

\section{R-LWE}
\label{app:RlweExampleCalc}
The following calculations show the working of the Ring LWE encryption for the algorithms \ref{alg: SampleLweKeyGen} to \ref{alg: SampleLweDecryption}. The ring used for this calculations is defined as $R=\mathbb{Z}[x]_{100}/(x^3+1)$. Starting first with the key generation:

\begin{align*}
  s  & = 1 + 0x + 1x^2                                                   \\
  A  & = 28 + 56x + 1x^2                                                 \\
  e  & = 1 + -1x + 2x^2= 1 + 99x + 2x^2                                  \\
  b  & = As + e                                                          \\
     & = (28 + 56x + 1x^2)*(1 + 0x + 1x^2) + (1 + 99x + 2x^2)            \\
     & = (28 + 28x^2) + (56x + 56x^3) + (1x^2 + 1x^4) + (1 + 99x + 2x^2) \\
     & = 29 + 155x + 31x^2 + 56x^3 + 1x^4 \mod x^3+1                     \\
     & = 29 + 155x + 31x^2 - 56 - 1x                                     \\
     & = -27 + 154x + 31x^2  \mod 100                                    \\
     & = 73 + 54x + 31x^2                                                \\
  sk & = s =1 + 0x + 1x^2                                                \\
  pk & = (A, b) = (28 + 56x + 1x^2, 73 + 54x + 31x^2)                    \\
\end{align*}

Encryption of message $m=(1,1,0)$:
\begin{align*}
  r       & =0 + 1 + 1x^2                                                                    \\
  e_1     & =98 + 0x + 98x^2                                                                 \\
  e_2     & =1 + 0x + 0x^2                                                                   \\
  m       & = (1, 1, 0) = 1+ 1x+ 0x^2                                                        \\
  u       & = A \cdot r + e_1                                                                \\
          & = (28 + 56x + 1x^2)*(0 + 1 + 1x^2) + (98 + 0x + 98x^2)                           \\
          & = (28x+84x^2+57x^3+x^4) + (98 + 0x + 98x^2)                                      \\
          & = 98+28x+182x^2+57x^3+x^4 \mod (x^3+1)                                           \\
          & = 98+28x+182x^2-57-x                                                             \\
          & = 41+27x+182x^2  \mod 100                                                        \\
          & = 41+27x+82x^2                                                                   \\
  v       & = b \cdot r + e_2 + (m*\left\lfloor q/2\right\rfloor)                            \\
          & = (73 + 54x + 31x^2) * (0 + 1 + 1x^2) + (1 + 0x + 0x^2) +  (1+ 1x+ 0x^2)*(100/2) \\
          & = (73x + 127x^2 + 85x^3 + 31x^4) + (1 + 0x + 0x^2) +  (50+ 50x+ 0x^2)            \\
          & = 51 + 123x + 127x^2 + 85x^3 + 31x^4 \mod (x^3+1)                                \\
          & = 51 + 123x + 127^2 - 85 - 31x                                                   \\
          & = -34 + 92x + 127x^2 \mod 100                                                    \\
          & = 66 + 92x + 27x^2                                                               \\
  m_{enc} & = (u,v) = (41+27x+82x^2, 66 + 92x + 27x^2)                                       \\
\end{align*}

Decryption:
\begin{align*}
  m & = \left\lfloor \frac{1}{\left\lfloor q/2\right\rfloor}*(v-s^T \cdot u)\right\rceil _2                                               \\
    & = \left\lfloor \frac{1}{\left\lfloor 100/2\right\rfloor}*((66 + 92x + 27x^2 )- (1 + 0x + 1x^2) \cdot (41+27x+82x^2))\right\rceil _2 \\
    & = \left\lfloor \frac{1}{50}*((66 + 92x + 27x^2 )- (41 + 27x + 123x^2 + 27x^3 + 82x^4))\right\rceil _2                               \\
    & = \left\lfloor \frac{1}{50}*(25 + 65x - 96x^2 - 27x^3 - 82x^4)\right\rceil _2 \mod (x^3+1)                                          \\
    & = \left\lfloor \frac{1}{50}*(25 + 65x - 96x^2 + 27 + 82x)\right\rceil _2                                                            \\
    & = \left\lfloor \frac{1}{50}*(52+147x-96x^2)\right\rceil _2 \mod 100                                                                 \\
    & = \left\lfloor \frac{1}{50}*(52+47x+4x^2)\right\rceil _2                                                                            \\
    & = \left\lfloor 1.04 + 0.64x + 0.08x^2\right\rceil _2                                                                                \\
    & = 1 + 1x + 0x^2                                                                                                                     \\
    & = (1, 1, 0)
\end{align*}

\section{M-LWE}
\label{app:MlweExampleCalc}
The following calculations show the working of the Module LWE encryption for the algorithms \ref{alg: SampleLweKeyGen} to \ref{alg: SampleLweDecryption}. The ring used for this calculations is defined as $R=\mathbb{Z}[x]_{100}/(x^3+1)$ and the dimension $n=2$.

Starting first with the key generation:
\begin{align*}
  s  & = \begin{bmatrix}2+ 1x + 0x^2 \\ 3+1x+1x^2 \end{bmatrix}                                 \\
  A  & = \begin{bmatrix}27+2x+43x^2 & 30+10x+35x^2 \\ 91+34x+50x^2 & 82+21x+94x^2 \end{bmatrix} \\
  e  & = \begin{bmatrix}1+1x+2x^2 \\ -3+3x+3x^2=97+3x+3x^2 \end{bmatrix}                        \\
  b  & = As+e                                                                                   \\
     & = \begin{bmatrix}27+2x+43x^2 & 30+10x+35x^2 \\ 91+34x+50x^2 & 82+21x+94x^2 \end{bmatrix}
  \cdot
  \begin{bmatrix}2+ 1x + 0x^2 \\ 3+1x+1x^2 \end{bmatrix}
  +
  \begin{bmatrix}1+1x+2x^2 \\ 97+3x+3x^2 \end{bmatrix}
  \\
     & = 
  \begin{bmatrix}
    56+56x+233x^2 \\
    263+210x+519x^2
  \end{bmatrix}
  + 
  \begin{bmatrix}1+1x+2x^2 \\ 97+3x+3x^2 \end{bmatrix}                                          \\
     & =   \begin{bmatrix}
             57+57x+235x^2 \\
             360+213x+522x^2
           \end{bmatrix}_{100}                                                                  \\
     & = \begin{bmatrix}
           57+57x+35x^2 \\
           60+13x+22x^2
         \end{bmatrix}                                                                         \\
  sk & = s =  \begin{bmatrix}2+ 1x + 0x^2 \\ 3+1x+1x^2 \end{bmatrix}                            \\
  pk & = (A, b) = \left (
  \begin{bmatrix}27+2x+43x^2 & 30+10x+35x^2 \\ 91+34x+50x^2 & 82+21x+94x^2 \end{bmatrix},
  \begin{bmatrix}
      57+57x+35x^2 \\
      60+13x+22x^2
    \end{bmatrix} \right )                                                                        \\
\end{align*}

Encryption of message $m=(1,1,0)$:
\begin{align*}
  r       & = \begin{bmatrix}1+1x+1x^2 \\ 1+0x+0x^2 \end{bmatrix}                                       \\
  e_1     & = \begin{bmatrix}2+98x+3x^2 \\ 97+3x+3x^2 \end{bmatrix}                                     \\
  e_2     & = 2+97x+97x^2                                                                               \\
  m       & =(1,1,0) = 1x+1x+0x^2                                                                       \\
  \\
  % Calculating u
  u       & = A^T \cdot r + e_1                                                                         \\
  % Step 2
          & = \begin{bmatrix}27+2x+43x^2 & 30+10x+35x^2 \\ 91+34x+50x^2 & 82+21x+94x^2 \end{bmatrix}^T
  \cdot
  \begin{bmatrix}1+1x+1x^2 \\ 1+0x+0x^2 \end{bmatrix}
  +
  \begin{bmatrix}2+98x+3x^2 \\ 97+3x+3x^2 \end{bmatrix}                                                 \\
  % Step 3
          & = \begin{bmatrix}27+2x+43x^2 & 91+34x+50x^2 \\ 30+10x+35x^2  & 82+21x+94x^2 \end{bmatrix}
  \cdot
  \begin{bmatrix}1+1x+1x^2 \\ 1+0x+0x^2 \end{bmatrix}
  +
  \begin{bmatrix}2+98x+3x^2 \\ 97+3x+3x^2 \end{bmatrix}                                                 \\
  %  Step 4
          & = \begin{bmatrix}73+20x+122x^2 \\ 67+26x+169x^2 \end{bmatrix}
  +
  \begin{bmatrix}2+98x+3x^2 \\ 97+3x+3x^2 \end{bmatrix}                                                 \\
  % Step 5
          & = \begin{bmatrix}75+118x+125x^2 \\ 164+23x+172x^2 \end{bmatrix}_{100}
  = 
  \begin{bmatrix}75+18x+25x^2 \\ 64+23x+72x^2 \end{bmatrix}                                             \\
  \\
  % Calculating V
  v       & = b^T \cdot r + e_2 + (m*\left\lfloor q/2\right\rfloor)                                     \\
  % Step 2
          & =\begin{bmatrix}
               57+57x+35x^2 \\
               60+13x+22x^2
             \end{bmatrix} ^T
  \cdot
  \begin{bmatrix}1+1x+1x^2 \\ 1+0x+0x^2 \end{bmatrix}
  + (2+97x+97x^2) + (1x+1x+0x^2) \cdot \left\lfloor 100/2\right\rfloor                                  \\
  % Step 4
          & = (25+92x+171x^2) + (2+97x+97x^2) + (50x+50x+0x^2)                                          \\
  % Step 5
          & = (77+239x+268x^2)_{100}                                                                    \\
  % Step 6
          & = 77+39x+68x^2                                                                              \\
  m_{enc} & = (u, v) = \left (
  \begin{bmatrix}75+18x+25x^2 \\ 64+23x+72x^2 \end{bmatrix}, 77+39x+68x^2   \right )                    \\
\end{align*}

Decryption:
\begin{align*}
  m & = \left\lfloor \frac{1}{\left\lfloor q/2\right\rfloor} *(v-s^T \cdot u)\right\rceil _2                                         \\
    & = \left\lfloor \frac{1}{\left\lfloor 100/2\right\rfloor} * \left (77+39x+68x^2-
  \begin{bmatrix}2+ 1x + 0x^2 \\ 3+1x+1x^2 \end{bmatrix}^T
  \cdot
  \begin{bmatrix}75+18x+25x^2 \\ 64+23x+72x^2 \end{bmatrix} \right )\right\rceil _2                                                  \\
    & = \left\lfloor \frac{1}{50} * (77+39x+68x^2-(216+190x+377x^2))\right\rceil _2                                                  \\
    & = \left\lfloor \frac{1}{50} * (-139-151x-309x^2)_{100}\right\rceil _2                                                          \\
    & = \left\lfloor \frac{1}{50} * 61+49x+91x^2\right\rceil _2                                                                      \\
    & = \left\lfloor \frac{61}{50} +\frac{49}{50}x+\frac{91}{50}x^2\right\rceil _2  = \left\lfloor 1.22+0.98x+1.82x^2\right\rceil _2 \\
    & = \left\lfloor1+1x+2x^2\right\rceil _2 =           1+1x+0x^2                                                                   \\
    &= (1,1,0)
\end{align*}