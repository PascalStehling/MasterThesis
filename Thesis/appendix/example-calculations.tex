\chapter{Example Calculations}

\section{Example Multidimensional Ring Calculation}
\label{app:ExampleMultiRingCalc}

Consider the ring $R = \mathbb{Z}_5[x]/(x^3+1)$ and 
$$
f =\begin{bmatrix}
    1+2x+3x^2 & 2+3x+4x^2 \\
    3+4x+x^2  & 1+3x+4x^2 \\
  \end{bmatrix} \in R^{2\times 2}
$$

$$
g = \begin{bmatrix}
    1+x+x^2   \\
    2+2x+2x^2 \\
  \end{bmatrix} \in R^2
$$

\begin{align*}
  f \cdot g & = {
  \begin{bmatrix}
    1+2x+3x^2 & 2+3x+4x^2 \\
    3+4x+x^2  & 1+3x+4x^2 \\
  \end{bmatrix}
  \cdot
  \begin{bmatrix}
    1+x+x^2   \\
    2+2x+2x^2 \\
  \end{bmatrix}
  }               \\
            & = {
  \begin{bmatrix}
    \begin{bmatrix}
      1 & -3 & -2 \\
      2 & 1  & -3 \\
      3 & 2  & 1  \\
    \end{bmatrix} & 
    \begin{bmatrix}
      2 & -4 & -3 \\
      3 & 2  & -4 \\
      4 & 3  & 2  \\
    \end{bmatrix}   \\
    \begin{bmatrix}
      3 & -1 & -4 \\
      4 & 3  & -1 \\
      1 & 4  & 3  \\
    \end{bmatrix} & 
    \begin{bmatrix}
      1 & -4 & -3 \\
      3 & 1  & -4 \\
      4 & 3  & 1  \\
    \end{bmatrix}   \\
  \end{bmatrix}
  \cdot
  \begin{bmatrix}
    \begin{bmatrix}
      1 \\
      1 \\
      1 \\
    \end{bmatrix} \\
    \begin{bmatrix}
      2 \\
      2 \\
      2 \\
    \end{bmatrix} \\
  \end{bmatrix}
  }               \\
            & = {
  \begin{bmatrix}
    1 & -3 & -2 & 2 & -4 & -3 \\
    2 & 1  & -3 & 3 & 2  & -4 \\
    3 & 2  & 1  & 4 & 3  & 2  \\
    3 & -1 & -4 & 1 & -4 & -3 \\
    4 & 3  & -1 & 3 & 1  & -4 \\
    1 & 4  & 3  & 4 & 3  & 1 
  \end{bmatrix}
  \cdot
  \begin{bmatrix}
    1 \\
    1 \\
    1 \\
    2 \\
    2 \\
    2 
  \end{bmatrix}
  }               \\
            & = {
  \begin{bmatrix}
    -14 \\
    2   \\
    24  \\
    -14 \\
    6   \\
    24
  \end{bmatrix}
  \mod 5
  }
  = {
  \begin{bmatrix}
    1 \\
    2 \\
    4 \\
    1 \\
    1 \\
    4
  \end{bmatrix}
  }
  = {
  \begin{bmatrix}
    \begin{bmatrix}
      1 \\
      2 \\
      4
    \end{bmatrix} \\
    \begin{bmatrix}
      1 \\
      1 \\
      4
    \end{bmatrix} \\
  \end{bmatrix}
  }               \\
            & = {
  \begin{bmatrix}
    1+2x+4x^2 \\
    1+1x+4x^2
  \end{bmatrix}
  }
\end{align*}


\section{Plain LWE}
\label{app:PlainLweCalc}
The following calculations should show the working of the Plain LWE encryption for the algorithms \ref{alg: SampleLweKeyGen} to \ref{alg: SampleLweDecryption}. The ring used for this calculations is defined as $R=\mathbb{Z}_{100}$ and $n=2$ for the dimensions. Starting first with the key generation:


\begin{align*}
  s  & = \begin{bmatrix}1 \\ 2 \end{bmatrix}
  A  = \begin{bmatrix}56 & 77 \\ 29 & 59 \end{bmatrix}
  e  = \begin{bmatrix}99 \\ 1 \end{bmatrix}  \\
  b  & = As+e                                \\
     & = \begin{bmatrix}
           56 & 77 \\
           29 & 59
         \end{bmatrix}
  \cdot
  \begin{bmatrix}
    1 \\
    2
  \end{bmatrix}
  +
  \begin{bmatrix}
    99 \\ 
    1 
  \end{bmatrix}
  \\
     & = 1
  \cdot
  \begin{bmatrix}
    56 \\
    29
  \end{bmatrix}
  + 2 
  \cdot
  \begin{bmatrix}
    77 \\ 
    59 
  \end{bmatrix}
  + 
  \begin{bmatrix}
    99 \\ 
    1 
  \end{bmatrix}                             \\
     & = \begin{bmatrix}
           309 \\
           148\end{bmatrix}_{100}              \\
     & = \begin{bmatrix}
           9 \\ 
           48 
         \end{bmatrix}                      \\
  sk & = s                                   \\
  pk & = (A, b) = \left (
  \begin{bmatrix}
      56 & 77  \\
      29 & 59 
    \end{bmatrix},
  \begin{bmatrix}
      9 \\
      48 
    \end{bmatrix} \right )                     \\
\end{align*}

With the secret and public key generated, the next step is to encrypt the message $m=1$ with the public key $pk$

\begin{align*}
  r       & = \begin{bmatrix}0 \\ 1 \end{bmatrix}
  e_1 = \begin{bmatrix}2 \\ 0 \end{bmatrix}
  e_2 = 99                                                          \\
  \\
  u       & = A^T \cdot r + e_1                                     \\
          & = \begin{bmatrix}
                56 & 77 \\
                29 & 59
              \end{bmatrix}^T
  \cdot
  \begin{bmatrix}
    0 \\
    1
  \end{bmatrix}
  +
  \begin{bmatrix}
    2 \\
    0
  \end{bmatrix}                                                    \\
          & = \begin{bmatrix}
                56 & 29 
                \\ 77 & 59 
              \end{bmatrix}
  \cdot 
  \begin{bmatrix}
    0 \\
    1 
  \end{bmatrix}
  +
  \begin{bmatrix}
    2 \\
    0
  \end{bmatrix}                                                    \\
          & = 0\cdot
  \begin{bmatrix}
    56 \\
    77
  \end{bmatrix}
  + 1 \cdot 
  \begin{bmatrix}
    29 \\
    59
  \end{bmatrix}
  +
  \begin{bmatrix}
    2 \\
    0
  \end{bmatrix}                                                    \\
          & = \begin{bmatrix}
                31 \\ 
                61 
              \end{bmatrix}_{100}
  = 
  \begin{bmatrix}
    31 \\
    61
  \end{bmatrix}                                                    \\
  \\
  v       & = b^T \cdot r + e_2 + (m*\left\lfloor q/2\right\rfloor) \\
          & =\begin{bmatrix}
               9 \\
               48
             \end{bmatrix}^T
  \cdot
  \begin{bmatrix}
    0 \\
    1
  \end{bmatrix}
  + 99 + 1 \cdot \left\lfloor 100/2\right\rfloor                    \\
          & =\begin{bmatrix}
               9 & 48
             \end{bmatrix}
  \cdot
  \begin{bmatrix}
    0 \\ 
    1 
  \end{bmatrix}
  + 99 + 50                                                         \\
          & = 9 \cdot 0 +48 \cdot 1 + 99 + 50                       \\
          & = 197_{100}                                             \\
          & = 97                                                    \\
  m_{enc} & = (u, v) = \left (
  \begin{bmatrix}
      31 \\
      61
    \end{bmatrix}, 97   \right )                                      \\
\end{align*}

Now the encrypted message $M_{enc}$ can be decrypted again, using the secret key $sk$:
\begin{align*}
  m & = \left\lfloor \frac{1}{\left\lfloor q/2\right\rfloor} *(v-s^T \cdot u)\right\rceil _2 \\
    & = \left\lfloor \frac{1}{\left\lfloor 100/2\right\rfloor} * \left (97-
  \begin{bmatrix}
      1 \\
      2
    \end{bmatrix}^T
  \cdot
  \begin{bmatrix}
      31 \\
      61
    \end{bmatrix} \right )\right\rceil _2                                                      \\
    & = \left\lfloor \frac{1}{50} * \left (97-
  \begin{bmatrix}
      1 & 2 
    \end{bmatrix}
  \cdot 
  \begin{bmatrix}
      31 \\ 
      61
    \end{bmatrix}\right )\right\rceil _2                                                       \\
    & = \left\lfloor \frac{1}{50} * (97-(31 \cdot 1 + 61 \cdot 2))\right\rceil _2            \\
    & = \left\lfloor \frac{1}{50} * (-56)_{100}\right\rceil _2                               \\
    & = \left\lfloor \frac{1}{50} * 44\right\rceil _2                                        \\
    & = \left\lfloor \frac{44}{50}\right\rceil _2  = \left\lfloor 0.88\right\rceil _2        \\
    & = 1                                                                                    \\
\end{align*}