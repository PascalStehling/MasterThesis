
In 2009, Craig Gentry published the first fully homomorphic encryption (FHE) algorithm in his PhD dissertation. With that it was possible for the first time to do any number of calculations on encrypted messages without having to decrypt them. Since the discovery of this first FHE algorithm, other such algorithms have been developed, but instead of using Ideal Lattices as the mathematical foundation, newer ones tend to use the Learning with Errors (LWE) and Ring-LWE (RLWE) problems.

In July 2022, the American National Institute of Standards and Technology (NIST) published the first group of winners of its competition for quantum-safe algorithms. The winner for general asymmetric encryption was the CRYSTALS-Kyber algorithm, which is based on Module-LWE (MLWE). This is an extension of the RLWE method in which polynomials in higher dimensions (vectors and matrices) are used. Even though it needs more computational power, in contrast to RLWE, it can be offset through parallelization of the calculations and higher security.

With these two distinct developments, the questions can be asked, if it possible to combine them and see if it’s possible to transfer existing FHE cryptosystems, which are based on RLWE to the MLWE method. This will first be researched theoretically and then verified with practical tests. 
These tests will be used to examine the advantages and disadvantages of the different Methods in terms of various properties, such as computation speed, error rate during decryption, number of possible calculations without errors and others. In order to structure these tests and thus establish good comparability, a test concept will be created in this thesis. 
At the end, the following question should be answered: Are their practical advantages to transferring existing FHE systems from RLWE to MLWE or whether the associated increased computing effort nullifies the advantages again?