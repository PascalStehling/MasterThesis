\chapter{Mathematical Background}
\label{MathBack}

In order to understand the mathematical concepts behind the encryption algorithms described here, some basic concepts are explained here. However, you should have a basic knowledge of linear algebra and polynomial calculus.

\section{Lattice}

% Based on \cite{LatticeTutorial}.

All of the algorithms discussed in this thesis are based on lattices, which is why we will briefly focus on them in more detail. In general, lattices behave like any other vector space, but they only consist of discrete vectors. This means that the vectors only contain integers and not real numbers as in a vector space.

Let $\textbf{B} = \{\textbf{b}_1, \textbf{b}_2, \ldots, \textbf{b}_m\}$ be a set of linearly independent vectors of $\mathbb{R}^n$. The lattice $\textit{L}$ generated by $\textbf{B}$ is the set of integer linear combinations of $\textbf{B}$. $\textbf{B}$ is called the basis of the lattice $\textit{L}$. That is,
$$L(\textbf{B}) = \{a_1\textbf{b}_1 + \ldots + a_m\textbf{b}_m | a_1, \ldots, a_m \in \mathbb{Z}  \} \subset \mathbb{R}^n$$


Using a matrix $\textbf{B}$, which contains the basis vectors as column vectors, we can generate $\textit{L}$ equivalently.

$$L(\textbf{B}) = \{\textbf{B}\cdot\textbf{x} | \textbf{x} \in \mathbb{Z}^m  \} \subset \mathbb{R}^n$$

As in this definition, the integer $n$ is the \textbf{dimension} of the lattice and $m$ is its \textbf{rank}. If $m = n$, then $\textit{L}$ is a \textbf{full-rank} lattice, which is the usual case in this thesis. 

An example of a lattice based on a basis $\textbf{B}$ and all the points that can be created with it, also called the \textbf{span}, can be seen in the figure \ref{fig:latticeGrid}.

\begin{figure}[ht]
  \centering
  \includegraphics[scale=0.2]{images/LatticeGrid.png}
  \caption[Span of an Lattice]{The span of an two-dimensional lattice with basis  $B = \{b_1, b_2\}$.}
  \label{fig:latticeGrid}
\end{figure}

% Nur dann benötigt wenn ich genauer auf die Funktionsweise von LWE eingehen will
% \section{Shortest Vector \& Closest Vector Problem}


\section{Integer \& Polynomial Rings with modulus}
This section is based on \cite{Algebra}.

\subsection*{Rings}
A ring is a set $R$ on which addition ($+$) and multiplication ($\cdot$) can be performed and results in a new Element, which is also part of the set R.
\begin{center}
  $ +: R+R\rightarrow R$ (Addition) and $\cdot: R \cdot R \rightarrow R$ (Multiplication)
\end{center}

These calculations need to fulfill the following conditions:
\begin{description}
  \item for addition: $R$ is an abelian group
        \begin{itemize}
          \item Associative property: $(a+b)+b = a+(b+c) | a,b,c \in R$
          \item Commutative property: $a+b = b+a | a,b \in R$
          \item Additive identity: There exists and element $0 \in R$ so that $a+0 = a | a \in R$
          \item Additive inverse: For each $a \in R$ there is an $-a \in R$ so that $a+(-a)=0$
        \end{itemize}
  \item for multiplication: R is an monoid
        \begin{itemize}
          \item Associative property: $(a\cdot b) \cdot b = a \cdot(b\cdot c) | a,b,c \in R$
          \item Multiplicative identity: There exists and element $1 \in R$ so that $a \cdot 1 = 1 \cdot a = a | a \in R$
        \end{itemize}
  \item Addition and Multiplication are distributive
        \begin{itemize}
          \item  $a\cdot (b + c) = a\cdot b + a\cdot c | a,b,c \in R$
          \item  $(a + b) \cdot c= a\cdot c + b\cdot c | a,b,c \in R$
        \end{itemize}
\end{description}

A ring is also called commutative if the multiplication is also commutative. For example, the ring over all integers $\mathbb{Z}$ is a commutative ring.

\subsection*{Modular arithmetic on Rings}

Congruence arithmetic, or modular arithmetic, is the term used to describe arithmetic with remainders when dividing integers. In everyday life, this is mainly encountered in connection with clocks. After 60 minutes, the minute hand returns to the same position as before. 

More generally, this can be described as $a \equiv b \mod n | a,b \in \mathbb{Z}, n \in \mathbb{N}$, where $n$ is the module by which $a$ and $\textbf{B}$ are divided until the remainder of both is less than $n$. If $a$ and $\textbf{B}$ are then equal, they are congruent. Or in a more mathematical expression: If there is a $k$, such as $a-b = k\cdot m$, then $a$ and $\textbf{B}$ are congruent.

An congruence relation with module $n$ on the set $\mathbb{Z}$, has the following properties:
% Based on \cite{Algebra} Seite 13.
$k,n \in \mathbb{N}$ and $a, a', b, b', c \in \mathbb{Z}$
\begin{enumerate}
  \item $a \equiv a \mod n$ (Reflexivity)
  \item $a \equiv b \mod n$ if $b \equiv a \mod n$ (Symmetry)
  \item If $a \equiv b \mod n$ and $b \equiv c \mod n$ then $a \equiv c \mod n$ (Transitivity)
  \item If $a \equiv a' \mod n$ and $b \equiv b' \mod n$ then $a+b \equiv a'+b' \mod n$
  \item If $a \equiv a' \mod n$ and $b \equiv b' \mod n$ then $a\cdot b \equiv a'\cdot b' \mod n$
  \item If $c$ and $n$ are coprime and $c \cdot a \equiv c \cdot b \mod n$ then $a \equiv b \mod n$
  \item If $a \equiv b \mod k\cdot n$ then $a \equiv b \mod n$
\end{enumerate}

The congruence class is the set of all numbers for an integer $a \in \mathbb{Z}$ modulus $n$ that produce the same remainder. It is defined as
$$[a]_n = \{b \in \mathbb{Z} | a \equiv b \mod n\}$$.

It follows that two numbers are congruent if both congruence classes are equal:

$$a \equiv b \mod n \Leftrightarrow [a]_n = [b]_n$$

With this we can create a set of all congruence classes modulo $n$:
$$\mathbb{Z}_n = \mathbb{Z}/n = \mathbb{Z} \mod n = \{[a]_n | a = 0, 1, \dot, n-1 \}$$.

For example, $Z_3 = \{[0]_3, [1]_3, [2]_3\}$. With addition and multiplication it is possible to create a commutative ring from $\mathbb{Z}_n$.
\begin{center}
  $[a]_n + [b]_n = [a+b]_n$ (addition) and $[a]_n \cdot [b]_n = [a\cdot b]_n$ (multiplication)
\end{center}

This allows to create finite rings $\mathbb{Z}_n$ for every natural number $n$ with $n$ elements in each ring and to perform calculations inside these ring. For example, an ring with $n=60$ can be created, which represents the minutes in every hour. If the minute hand shows now $48$ and we want to know where it is after $3$ times $13$ minutes, we can calculate it like:
$$[48]_{60} + [3]_{60}\cdot [13]_{60} = [48+3\cdot 13]_{60} = [87]_{60} = [27]_{60}$$

\subsection*{Polynomial Rings}

A polynomial with coefficients in a ring $R$ is expressed as 
$$f = a_0+ a_1x+\cdots+ a_{n-1}x^{n-1}+a_nx^n | a_0, \cdots, a_n \in R$$

The variable $n$ defines the degree $\deg(f)$ of the polynomial, which is the largest exponent in a polynomial.

They can be added and multiplied like any other polynomial. Such a polynomial ring with one variable $x$ and its coefficients in $R$ is written as $R[x]$. This is a generalization of the rings we had before, because $R$ is a subset of $R[x]$ ($R \subset R[x]$), since $R$ is a polynomial with $\deg(0)$: $R = R[x] := a\cdot x^0 = a$.

Such a polynomial ring can also be defined over a finite ring, so that each coefficient is part of that finite ring. This is written as $R_n = \mathbb{Z}_n[x]$. The coefficients follow the same rules for addition and multiplication as described above. The following example takes place in the ring $R_5 = \mathbb{Z}_5[x]$ and $f, g \in R_5$ with $f=1+2x+3x^2$ and $g=4+2x$:

\begin{align*}
  f\cdot 4 & = (1+2x+3x^2) * 4                                 \\
           & = [4]_5+[8]_5x+[12]_5x^2                          \\
           & = 4+3x+2x^2                                       \\
  f+g      & = (1+2x+3x^2)+(4+2x)                              \\
           & = [5]_5+[4]_5x+[3]_5x^2                           \\
           & = 4x+3x^2                                         \\
  f\cdot g & = (1+2x+3x^2)\cdot(4+2x)                          \\ 
           & = [4]_5+[2]_5x+[8]_5x+[4]_5x^2+[12]_5x^2+[6]_5x^3 \\
           & = [4]_5+[10]_5x+[16]_5x^2+[6]_5x^3                \\
           & = 4+0x+1x^2+1x^3                                  \\
\end{align*}

As with any polynomial multiplication, the degree can increase as you multiply two polynomials, leading to increasingly larger polynomials with each multiplication. Since the modulo operation creates a finite ring, we can also create a modulo operation that creates a finite ring over a polynomial where the degree stays the same or is less than some upper bound. For this we have a ring $R$ and $f, g, q, r \in R[x], g\neq 0$, where $f$ is a polynomial, $g$ is the modulus, and $r$ is the remainder:
\begin{center}
  $f = g\cdot q + r $ and $\deg(r)<\deg(g)$.
\end{center}

After this calculation, $r$ will be the remainder of $f$ with a degree smaller than that of $g$, which will be used for further calculations. With this, we can now define polynomial rings that have a module to generate finite coefficients and a polynomial function to generate finite degree. This is written as 
$$R_n = \mathbb{Z}_n[x]/f(x)$$

For most cases in cryptography, $f(x)$ will be the function $x^d+1$\improvement{These are cyclotomic polynomial, maybe explain}, as this simplifies the calculation of the remainder. Instead of doing lengthy polynomial divisions, one can simply subtract $d$ from the exponent and invert the coefficient if the exponent is greater than or equal to $d$. This must be repeated until the largest exponent is less than $d$. For example, $f \in \mathbb{Z}_5[x]/(x^3+1)$

\begin{align*}
  f & = 3+4x^2+2x^3+x^5+3x^6 & \mod (x^3+1) \\
    & = 3+4x^2-2-x^2-3x^3    & \mod (x^3+1) \\
    & = 3+4x^2-2-x^2+3                      \\
    & = 4+3x^2
\end{align*}

\subsection*{Polynomial Ring arithmetic using Vectors \& Matrices}
\label{sec:PolyMulMath}

When calculating with polynomials, this can also be broken down into vector and matrix calculations. This is done by separating a polynomial into two vectors, the coefficient vector and the variable vector.
$$
  f(x) = a_0+ a_1x+\cdots+ a_{n-1}x^{n-1}+a_nx^n =
  \begin{bmatrix}a_0\\a_1\\ \vdots \\a_{n-1}\\a_n \end{bmatrix}
  \cdot
  \begin{bmatrix}1 & x & \cdots & x^{n-1} &  x^n \end{bmatrix}
$$

When doing addition with such vectors, we just need to make sure that the vectors have the same length, which means that the polynomials must have the same degree. If this is not the case, we can fill the shorter vector with 0 so that they have the same degree. As our polynomials are always defined in a commutative  ring, the associative, commutative and distributive properties apply. So addition would look like this:

\begin{align*}
  f(x) + g(x) & = {
  (a_0+ a_1x+\cdots+ a_{n-1}x^{n-1}+a_nx^n)+
  (b_0+ b_1x+\cdots+ b_{n-1}x^{n-1}+b_nx^n)
  }                 \\
              & = {
  \begin{bmatrix}
    a_0     \\
    a_1     \\
    \vdots  \\
    a_{n-1} \\
    a_n
  \end{bmatrix} \cdot
  \begin{bmatrix}
    1       & 
    x       & 
    \cdots  & 
    x^{n-1} & 
    x^n
  \end{bmatrix} + 
  \begin{bmatrix}
    b_0     \\
    b_1     \\
    \vdots  \\
    b_{n-1} \\
    b_n     \\
  \end{bmatrix} \cdot 
  \begin{bmatrix}
    1       & 
    x       & 
    \cdots  & 
    x^{n-1} & 
    x^n     & 
  \end{bmatrix}
  }                 \\
              & = {
  \left (
  \begin{bmatrix}
      a_0     \\
      a_1     \\
      \vdots  \\
      a_{n-1} \\
      a_n
    \end{bmatrix} + 
  \begin{bmatrix}
      b_0     \\
      b_1     \\
      \vdots  \\
      b_{n-1} \\
      b_n     \\
    \end{bmatrix}
  \right ) \cdot 
  \begin{bmatrix}
    1       & 
    x       & 
    \cdots  & 
    x^{n-1} & 
    x^n 
  \end{bmatrix}
  }                 \\
\end{align*}

The variable vector will always have the same shape as the coefficient vector. Because of the ring properties, the variable vector can always be placed outside the parentheses, since all polynomials have the same one in common. For this reason, only the coefficient vector will be written out in this paper.

Doing the multiplication is a bit trickier, as normally the degree increases when multiplying two polynomials. But as seen previously, this is not the case if we are doing our calculations inside an polynomial ring. So by using the polynomial ring $R_n = \mathbb{Z}_n/(x^d+1)$ the degree of the polynomial will never be greater then $d$. So even when doing multiplications, the degree will always be smaller then $d$.

\unsure[inline]{Cannot find any proof that verefies the following, or a name for it, but its working. Its based on circulant matrix, which have special properties. Maybe I need to dig deeper there}
This makes it possible to turn the polynomial multiplication and subsequent modulo operation into a single matrix-vector multiplication, where the resulting vector is the wanted coefficient vector. This is only possible if we use the modulo polynomial $x^d+1$. For this we need the coefficient vectors of both polynomials of the same degree, both from the same ring $f, g \in \mathbb{Z}_n/(x^d+1)$. One of the vectors has to be transformed into a circulant matrix, where the diagonal and the lower triangle are positive and the upper triangle (without diagonal) is negative. This matrix is then multiplied by the other coefficient vector, resulting in the new coefficient vector.
$$
  \begin{bmatrix}
    a_0     & -a_{n}  & -a_{n-1} & \cdots & -a_2   & -a_1   \\
    a_1     & a_0     & -a_{n}   & \cdots & -a_3   & -a_2   \\
    a_2     & a_1     & a_0      & \cdots & -a_4   & -a_3   \\
    \vdots  & \vdots  & \vdots   & \ddots & \vdots & \vdots \\
    a_{n-1} & a_{n-2} & a_{n-3}  & \cdots & a_0    & -a_{n} \\
    a_{n}   & a_{n-1} & a_{n-2}  & \cdots & a_1    & a_0    \\
  \end{bmatrix}
  \cdot
  \begin{bmatrix}
    b_0     \\
    b_1     \\
    b_2     \\
    \vdots  \\
    b_{n-2} \\
    b_{n-1}
  \end{bmatrix}
$$

The following examples will show this for two polynomials $f, g \in \mathbb{Z}_n/(x^3+1)$ with $f(x) = 3x^2+4x+1$ and $g(x) = x^2+6x+3$, using a normal polynomial multiplication and subsequent modulo, and the same with an single matrix-vector multiplication.

\begin{align*}
  f(x)\cdot g(x) & = (1+4x+3x^2) \cdot (3+6x+x^2)                & \mod x^3+1 \\
                 & = 3+6x+x^2 + 12x+24x^2+4x^3 + 9x^2+18x^3+3x^4 & \mod x^3+1 \\
                 & = 3+18x+34x^2+22x^3+3x^4                      & \mod x^3+1 \\
                 & = 3+18x+34x^2-22-3x                           &            \\
                 & = -19+15x+34x^2                               &            \\
  f(x)\cdot g(x) & = {
  \begin{bmatrix}
    1 & -3 & -4 \\
    4 & 1  & -3 \\
    3 & 4  & 1  \\
  \end{bmatrix}
  \cdot 
  \begin{bmatrix}
    3 \\
    6 \\
    1 \\
  \end{bmatrix} }                                                             \\
                 & = {
  3 \cdot \begin{bmatrix}
            1 \\
            4 \\
            3 \\
          \end{bmatrix}
  + 6 \cdot   \begin{bmatrix}
                -3 \\
                1  \\
                4  \\
              \end{bmatrix}
  + 1 \cdot   \begin{bmatrix}
                -4 \\
                -3 \\
                1  \\
              \end{bmatrix}
  }                                                                           \\
                 & = \begin{bmatrix}
                       -19 \\
                       15  \\
                       34  \\
                     \end{bmatrix}                                           \\
                 & = -19+15x+34x^2                                            \\                     
\end{align*}

After these calculations the modulo operation could be applied on the coefficients either in the polynomial or in the vector, to generate the finite coefficient ring.

\subsection*{Multidimensional Rings}

Rings can be not only in one dimension, but also in higher dimensions. This is written, as usual, with the dimensions as exponents in the ring. So a $m \times n$ ring matrix with modulus $q$ would be written as $\mathbb{Z}^{m\times n}_q$.

The same can be done with finite polynomial rings. To make it easier, we will first define the ring $R = \mathbb{Z}_q[x]/(x^d+1)$ and based on this ring we will define the form of a variable like $\textbf{A} \in R^{m\times n}$, which would result in a $m \times n$ matrix where all values are elements of the finite polynomial ring $R$.

All calculations in higher dimensions follow the usual rules. To multiply higher dimensional finite polynomial rings, the method described above is used. This means that each polynomial is transformed into a vector, or a matrix if it is a multiplication. Therefore, the size of the current dimension is increased by the degree $d$. An Example can be found in Appendix \ref{app:ExampleMultiRingCalc}