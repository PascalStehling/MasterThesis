\chapter{Introduction}
\label{Introduction}

% \section{Background}

In the early months of 1978, one of the most significant cryptographic systems, the RSA system \cite{RSA}, was published. With the advent of the Internet in the 1990s and the subsequent need for secure data transfer, it became one of the most widely used encryption schemes to date. In the subsequent period of slightly more than half a year, two of the authors of the RSA paper published a new concept, based on the RSA concept, which they designated ''privacy homomorphism`` \cite{Rivest1978}. This concept would later be known as homomorphic encryption. This is an encryption system whereby operations can be executed directly on encrypted data, eliminating the necessity of first decrypting it, running the operations, and then encrypting it again. Such a system would not only eliminate the necessity for decryption and encryption at the processing stage, it would also ensure that the plain text is not accessible to the party undertaking this processing. However, at the system's inception, only one operation was feasible: multiplication. To develop a system capable of general computing, the addition operation was necessary as a second operation, as these two operations enable the recreation of all other operations at the bit level. Unfortunately, the creation of a homomorphic encryption scheme with unlimited additions and multiplications, also known as full homomorphic encryption (FHE), proved to be a formidable challenge.\\
In 1994, Peter Shor published his algorithm \cite{Shor}, which describes how a quantum computer could factorize numbers in polynomial time. In contrast to classical computers, for which this problem is categorized as a hard problem. As the RSA cryptosystem is based on this exact issue being hard to solve, it would be possible to find the private key for any public key, thus undermining the cryptosystem's security. Fortunately, no quantum computer capable of such an operation was anywhere near availability at the time, so this problem remained theoretical.\\
Approximately a decade later, in 2005, O. Regev devised a novel mathematical framework, termed Learning with Error (LWE) \cite{Regev2005OnLL}, which enables the construction of new cryptosystems. This framework is based on an error term within a linear system of equations constructed on a lattice. The mathematical problem that he exploits for security is the hardness of the shortest vector problem (SVP). There are variants of this problem, called Ring-LWE, where a polynomial is used instead of a matrix, and M-LWE, which mixes Ring-LWE and the (Plain-)LWE together, resulting in matrices of polynomials. In 2009, Craig Gentry published the first full-homomorphic encryption scheme \cite{Gentry2009AFH}. This development prompted renewed optimism regarding the advancement of FHE schemes, as it became evident that the concept was indeed feasible. However, the primary challenge that remained was the issue of performance. To enhance the efficiency of this scheme, the initial version, which was based on the ideal lattice, was adapted to the R-LWE scheme. Over time, significant advancements have been made in the development of these FHE schemes, which are constructed on basis of R-LWE \cite{FHESurvey}. However, the primary challenge persists, namely the performance, which is frequently 1000s of times slower than operations on the plain text.\\
In recent years, there has been a resurgence of interest in quantum computers as various companies compete to develop the first practical and useful quantum computer \cite{googleQuantumComputing} \cite{ibmQuantumComputing}. Consequently, the performance of these computers has been steadily improving. If the promises made are accurate, it is possible that in 10 years, viable quantum computers will be available on the market. These computers could run Shor's Algorithm and thereby breach the security of RSA (and other) cryptosystems, potentially undermining the security of the internet as it currently stands. To circumvent such potential issues, the US National Institute of Standards and Technology (NIST) initiated an open competition in 2016, wherein individuals could submit novel cryptographic systems for analysis. Research teams from around the globe would then endeavor to identify vulnerabilities in these systems. In 2022, the NIST announced the first four winners \cite{nistAnouncement}, three of which were based on LWE. The two most recommended systems, CRYSTALS-Kyber \cite{CyrstalsKyber} and CRYSTALS-Dilithium \cite{crystalsDilithium}, are both based on M-LWE.


% \section{Goal of this Thesis}

In light of these recent advancements in M-LWE-based encryption schemes and the established R-LWE-based homomorphic encryption schemes, a question arises concerning the potential for integrating these two approaches: Is it possible to port the R-LWE-based homomorphic encryption schemes to M-LWE, and whether this results in an improvement in performance? Should the advantages outweigh the disadvantages, this would facilitate new synergies between the current endeavour to enhance the security of a post-quantum internet and the construction of efficient and dependable homomorphic encryption algorithms. For instance, enhanced and high-performing implementations or even hardware accelerators could be reused, thereby enhancing the efficacy of homomorphic encryption while simultaneously reducing the cost of development.

The thesis is divided into five principal sections. The first section of the thesis provides an introduction to the mathematical background, wherein all necessary mathematical operations will be explained in sufficient detail. Subsequently, the LWE problems will be described in greater detail, and a basic LWE-based encryption scheme will be constructed. The scheme is capable of functioning on Plain-, Ring-, and Module-LWE. Following this, homomorphic encryption will be outlined, and the LWE-based encryption scheme will be expanded to become homomorphic for all three LWE modes. These schemes will then be evaluated based on their memory usage, processing performance, and calculation depth. Ultimately, the aforementioned question will be addressed based on these findings.
