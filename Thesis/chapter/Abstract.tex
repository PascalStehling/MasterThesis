This thesis investigates the conversion of Ring-LWE (R-LWE)-based homomorphic encryption schemes to Module-LWE (M-LWE) and analyses the resulting performance differences. The advantage of M-LWE is that a fixed-sized polynomial degree can be utilized, and the security of the system can be changed by increasing the vector/matrix dimension. This is the same concept that is utilized in the CRYSTALS-Kyber encryption scheme. The feasibility of transferring R-LWE to M-LWE is demonstrated based on the BFV homomorphic encryption scheme, demonstrating that a functioning homomorphic encryption scheme can be maintained. While the addition is straightforward, the multiplication necessitates the generation of multiple relinearization (evaluation) keys. It is demonstrated that the practical performance is only slightly inferior to that of R-LWE, with the advantage of smaller ciphertext sizes. However, there is still considerable scope for improvement in theoretical aspects, such as the study of security benefits and in practice, in enhancing the general performance.