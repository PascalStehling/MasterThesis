This thesis explores the conversion of Ring-LWE-based homomorphic encryption schemes to Module-LWE and the subsequent evaluation of its performance. The benefit of this approach is that fixed-sized polynomial degrees can be employed, and the security of the system can be enhanced by increasing the vector/matrix size. This is the same concept employed in the CRYSTALS-Kyber encryption scheme. The feasibility of transferring R-LWE to M-LWE based on BFV is demonstrated, and it is shown that a functioning homomorphic encryption scheme can be maintained. While the addition was straightforward, the multiplication necessitates the generation of multiple relinearization/evaluation keys, which were used to construct a functional, somewhat homomorphic encryption scheme. It was also demonstrated that the practical performance is only slightly inferior to that of R-LWE, with the advantage of smaller ciphertext sizes. However, there is still considerable scope for improvement in theoretical aspects, such as the study of security benefits and in practice, in enhancing the general performance.