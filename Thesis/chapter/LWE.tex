\chapter{Learning with Errors}
\label{LWE}

In this section, we will take a closer look at the Learning with Errors (LWE) algorithm (also called Plain LWE) and its different versions, namely Ring LWE (R-LWE) and Module LWE (M-LWE).

\section{The Learning with Errors Problem}
\label{sec:LweProblem}

In 2005, Regev first described the LWE problem \cite{Regev2005OnLL}. He also proved its hardness, but we won't go into those details here. The basic idea is to add an error vector to a linear system of equations. This makes the normally trivially solvable system surprisingly hard to solve.

In more mathematical terms, $\mathbb{Z}_q = \mathbb{Z}/q$, $\textbf{A} \in \mathbb{Z}_q^{n \times m}$, $\textbf{s} \in \mathbb{Z}_q^m$, $\textbf{b} \in \mathbb{Z}_q^n$, with which we can form the linear system of equations, where $\textbf{A}$ and $\textbf{b}$ are given and the vector $\textbf{s}$ represents the unknowns we want to retrieve

$$\textbf{A}\cdot \textbf{s} = \textbf{b}$$

Or written as system of equations it would look like:
$$
  \setlength\arraycolsep{0pt}
  \begin{array}{ c  >{{}}c<{{}} c  >{{}}c<{{}}  c >{{}}c<{{}}  c @{{}={}} c }
    \textbf{A}_{11}\textbf{s}_1 & + & \textbf{A}_{12}\textbf{s}_2 & + & \cdots & + & \textbf{A}_{1m}\textbf{s}_m & \textbf{b}_1 \\
    \textbf{A}_{21}\textbf{s}_1 & + & \textbf{A}_{22}\textbf{s}_2 & + & \cdots & + & \textbf{A}_{2m}\textbf{s}_m & \textbf{b}_1 \\
    \vdots                      &   & \vdots                      &   & \vdots &   & \vdots                      & \vdots       \\
    \textbf{A}_{n1}\textbf{s}_1 & + & \textbf{A}_{n2}\textbf{s}_2 & + & \cdots & + & \textbf{A}_{nm}s_m          & \textbf{b}_n \\
  \end{array}
$$

This can easily be solved with the Gaussian algorithm. But if we just add an error vector $\textbf{e} \in \mathbb{Z}_q^n$ with small values, it becomes surprisingly hard. The hardness is based on variants of the Shortest Vector Problem (SVP), which describes the hardness of finding the shortest vector in the lattice. This is easily solvable in smaller dimensions, but gets harder as the dimensions are increased. The equation after adding the small error term is the following:

$$\textbf{A}\cdot \textbf{s} + \textbf{e}= \textbf{b}$$

This is the fundamental equation that underlies all LWE problems. The majority of the differences will be attributable to the ring or dimensions in use. All LWE-based encryption schemes presented in this thesis will be asymmetric encryption schemes (\cite{Eckert2018}). In consequence, each of these schemes comprises a private key, $pk$, and a secret key, $sk$, and is composed of three functions:

\begin{enumerate}
  \item \textbf{KeyGen}: For generating the private and secret key
  \item \textbf{Encryption}: For encrypting some message $m$ with the private key $pk$ creating a ciphertext $ct$
  \item \textbf{Decryption}: For decrypting the ciphertext $ct$ with the secret key $sk$ retrieving the original message $m$
\end{enumerate}

\info[inline]{Maybe going more into detail of the SVP or general the hardness of the LWE Problem?}

\section{LWE based encryption scheme}
\label{sec:Lwe-Encryption}

In this section, we will describe a simple LWE-based encryption scheme and how it can be converted to R-LWE and M-LWE. The following algorithm is loosely based on the Kyber \cite{CyrstalsKyber} scheme, with some simplifications.

All calculations are done in the ring $R = \mathbb{Z}_q$, where $q$ is the modulus. If values from $R$ are chosen uniformly, this is denoted by $x \leftarrow R$. Otherwise, if small values are chosen from $R$, this is written as $x \leftarrow \chi_R$. This can be done by choosing uniformly from a set of small numbers all in $R$ (e.g., ${-4,\ldots, 4}$ if $q$ is big enougth), or by choosing from an error distribution, such as the discrete Gaussian, as described in \cite{Regev2005OnLL}.

The following three algorithms describe the example schema. Algorithm \ref{alg: SampleLweKeyGen}, the key generation, describes how to generate the private key $pk$ and the secret key $sk$. It uses the LWE problem as described above. The secret key, which the owner should never share, is the vector $\textbf{s}$. The public key $pk$, which can be shared, consists of the transformation matrix $\textbf{A}$ and the transformed secret key plus the error $\textbf{b}$. The error $\textbf{e}$ is discarded after the computation of $\textbf{b}$. The values of $\textbf{e}$ and $\textbf{s}$ should be rather small, and $\textbf{A}$ is uniformly sampled from $R$.

\begin{algorithm}[htb]
  \begin{algorithmic}[1]
    \STATE $\textbf{s} \leftarrow \chi_R^n$
    \STATE $\textbf{A} \leftarrow R^{n \times n}$
    \STATE $\textbf{e} \leftarrow \chi_R^n$
    \STATE $\textbf{b} = \textbf{A}\cdot \textbf{s}+\textbf{e}$
    \RETURN $(pk:=(\textbf{A}, \textbf{b}), sk:=\textbf{s} )$
  \end{algorithmic}
  \caption{Sample LWE: KeyGen}
  \label{alg: SampleLweKeyGen}
\end{algorithm}

Algorithm \ref{alg: SampleLweEncryption}, the encryption, describes how to encrypt a message $m$ with the public key $pk$. The errors $\textbf{e}_1$ and $e_2$ are randomly sampled with small values and used to create more uncertainty around the message. The same message can therefore be decrypted with different errors and yield different values. This makes it harder for attackers to find patterns in the decryption. The idea behind $\textbf{r}$ is to select a subset of $\textbf{A}$ and $\textbf{b}$, since $~50\%$ of the values in $\textbf{r}$ will be $0$, meaning that these columns in $\textbf{A}$ and $\textbf{b}$ are irrelevant (multiplied by $0$). This helps to create more entropy between different encryptions, as a different subset of $\textbf{A}$ and $\textbf{b}$ will be used to encrypt each time.

The new values and the public key are used to calculate two values: $\textbf{u}$ and $v$. The first term, $\textbf{u}$, can be considered the cancel term for $\textbf{b}$, where the secret $\textbf{s}$ is missing. $v$ is the actual value term, which is composed of a subset of $\textbf{b}$ with some small error added and the scaled message $m'$. For the scaled message $m' = m\cdot \left\lfloor q/2\right\rfloor$, the message is multiplied with the rounded down version of half the modulus. This operation results in the values of the message $0$ and $1$ in the ring being approximately as distant from each other as possible.

\info[inline]{Als formeln nochmal kurz aufzeigen, dann pseudo code}
\begin{algorithm}[htb]
  \begin{algorithmic}[1]
    \REQUIRE $m \in \mathbb{Z}_2 = \{0, 1\}$, $pk = (\textbf{A}, \textbf{b})$
    \STATE $\textbf{r} \leftarrow \{0, 1\}^n$
    \STATE $\textbf{e}_1 \leftarrow \chi_R^n$
    \STATE $\textbf{u} = \textbf{A}^T \cdot \textbf{r} + \textbf{e}_1$
    \STATE $e_2 \leftarrow \chi_R$
    \STATE $v = \textbf{b}^T \cdot \textbf{r} + e_2 + (m\cdot \left\lfloor q/2\right\rfloor)$
    \RETURN $ct := (\textbf{u}, v)$
  \end{algorithmic}
  \caption{Sample LWE: Encryption}
  \label{alg: SampleLweEncryption}
\end{algorithm}

Algorithm \ref{alg: SampleLweDecryption}, the Decryption, describes how to decrypt an ciphertext $ct$ using the secret key $sk$.



\begin{algorithm}[htb]
  \begin{algorithmic}[1]
    \REQUIRE $ct = (\textbf{u}, v)$, $sk = \textbf{s}$
    \RETURN $\left\lfloor \frac{1}{\left\lfloor q/2\right\rfloor}\cdot \left[v-\textbf{s}^T \cdot \textbf{u}\right]_q\right\rceil _2$
  \end{algorithmic}
  \caption{Sample LWE: Decryption}
  \label{alg: SampleLweDecryption}
\end{algorithm}


To get a better understanding, consider the following simplification of the term in algorithm \ref{alg: SampleLweDecryption}.

\begin{align*}
   & \left\lfloor \frac{1}{\left\lfloor q/2\right\rfloor}\cdot \left[v-\textbf{s}^T \cdot \textbf{u}\right]_q\right\rceil _2                                                                                                                                        \\
   & = \left\lfloor \frac{1}{\left\lfloor q/2\right\rfloor}\cdot \left[\textbf{b}^T \cdot \textbf{r} + e_2 + (m\cdot \left\lfloor q/2\right\rfloor)-\textbf{s}^T \cdot (\textbf{A}^T \cdot \textbf{r} + \textbf{e}_1)\right]_q \right\rceil _2                      \\
   & = \left\lfloor \frac{1}{\left\lfloor q/2\right\rfloor}\cdot \left[(\textbf{As}+\textbf{e})^T \cdot \textbf{r} + e_2 + (m\cdot \left\lfloor q/2\right\rfloor)-\textbf{s}^T \textbf{A}^T \cdot \textbf{r} - \textbf{s}^T \textbf{e}_1\right]_q \right\rceil _2   \\
   & = \left\lfloor \frac{1}{\left\lfloor q/2\right\rfloor}\cdot \left[(\textbf{As})^T \cdot \textbf{r} + \textbf{e}^T\textbf{r}+ e_2 + (m\cdot \left\lfloor q/2\right\rfloor)-(\textbf{As})^T \cdot \textbf{r} - \textbf{s}^T \textbf{e}_1\right]_q\right\rceil _2 \\
   & = \left\lfloor \frac{1}{\left\lfloor q/2\right\rfloor}\cdot \left[\textbf{e}^T\textbf{r}+ e_2 + (m\cdot \left\lfloor q/2\right\rfloor)- \textbf{s}^T \textbf{e}_1\right]_q\right\rceil _2                                                                      \\
   & = \left\lfloor \frac{\textbf{e}^T\textbf{r}}{\left\lfloor q/2\right\rfloor}+ \frac{e_2 }{\left\lfloor q/2\right\rfloor}+ m - \frac{\textbf{s}^T \textbf{e}_1}{\left\lfloor q/2\right\rfloor}\right\rceil _2                                                    \\
   & = \left\lfloor m' \right\rceil _2                                                                                                                                                                                                                              \\
   & = m \in \{0,1\}
\end{align*}
\info[inline]{Die ltzte Zeile weg lassen oder anders schreiben, da das gerunde nur dann gleich m ist wenn der fehler klein genug ist. Das muss entweder in der gleichung oder durch den text genau heraus kommen}

As demonstrated by the calculation, by multiplying the cancellation term $\textbf{u}$ with the secret $\textbf{s}$, the transformation $(\textbf{As})^T \cdot \textbf{r}$ in $v$ can be canceled out. This results in the message with some error values being added to it. The erroneous message will then be rounded, which will result in the original message. This process will only be successful if all error terms together are smaller than $\frac{q}{4}$. This is due to the fact that the possible values in the message are separated by a distance of $\frac{q}{2}$ from each other. Consequently, all values between $-\frac{q}{4}\mod q=\frac{3q}{4}$ and $\frac{q}{4}$ are rounded back to $0$, while all values between $\frac{q}{4}$ and $\frac{3q}{4}$ are rounded to $1$. Consequently, provided that the message (either $0$ or $\frac{q}{2}$) is not shifted by more than $\frac{q}{4}$, it will remain within the rounding area of the original message.

\info[inline]{Maybe add an image which shows the rounding with an clock}

The current definition of this algorithm allows only $1$ bit to be encoded at the time. This could be improved with some tricks, but for simplicity reasons we wont do that here. 

To observe the functioning of this algorithm in practice, please refer to the example calculation in the appendix, which can be found in Appendix \ref{app:PlainLweCalc}.

\section{Transforming LWE to R-LWE and M-LWE}
\label{sec:TransformingLweToRlweAndMlwe}

To transform the algorithms described above into Ring-LWE, only a few changes need to be made. Most importantly, a polynomial ring will be defined as $R = \mathbb{Z}[x]_q/(x^d+1)$, with the dimension $n=1$, which means that there are only polynomials. Instead of having a vector $\textbf{r}$, it will now be a polynomial in the ring $R$, where all coefficients are either $0$ or $1$. The message to be encrypted is also transformed into a polynomial in $R$ with the message bits being the coefficients of the polynomial. Consequently, $d$ bits can now be encoded in one message. As all values are now polynomials, polynomial arithmetic is used in place of matrix arithmetic. However, as previously stated, the polynomial arithmetic in the ring can also be transformed into matrix arithmetic. All equations stay the same and the structure of the Algorithms does not change.

An illustrative example of the three-step process for RLWE can be found in \ref{app:RlweExampleCalc}.

As next step, Ring-LWE can be transformed into Module-LWE. Todo this we only need to increase the dimensions, so that $n>1$. So instead of working with polynomials as in R-LWE, matrices and vectors of these polynomials will be used.

An example can found in Appendix \ref{app:MlweExampleCalc}

So in total, the only real differences between the Plain LWE, R-LWE and M-LWE are the dimensions and the ring. The computation itself stays the same. An summarized overview of the differences can be found in table \ref{table:LweDiffs}

\begin{table}[htbp]
  \caption[LWE variables shape comparison]{Comparison between the shapes of the variables for the different LWE Types}
  \label{table:LweDiffs}
  \centering
  \begin{tabular}{|c|l|l|l|}
    \hline
                                                    & Plain LWE        & R-LWE                     & M-LWE                         \\
    \hline
    Ring $R$                                        & $\mathbb{Z}_q$   & $\mathbb{Z}[x]_q/(x^d+1)$ & $\mathbb{Z}[x]_q/(x^d+1)$     \\
    $\textbf{A}$                                    & $R^{n\times n}$  & $R$                       & $R^{n\times n}$               \\
    $\textbf{s},\textbf{b},\textbf{e},\textbf{e}_1$ & $R^{n}$          & $R$                       & $R^{n}$                       \\
    $e_2$                                           & $R$              & $R$                       & $R$                           \\
    $\textbf{r}$                                    & $\mathbb{Z}_2^n$ & $\mathbb{Z}[x]_2/(x^d+1)$ & $(\mathbb{Z}[x]_2/(x^d+1))^n$ \\
    $m$                                             & $\mathbb{Z}_2$   & $\mathbb{Z}[x]_2/(x^d+1)$ & $\mathbb{Z}[x]_2/(x^d+1)$     \\
    \hline
  \end{tabular}
\end{table}

As the variables of the different LWE types have different dimensions, also the keys that need to be stored and shared and the messages have different dimensions. A comparison can be found in Table \ref{table:LweKeys}. As can be seen there, Plain LWE and R-LWE each depend only on one variable ($n$ or $d$ respectively), while M-LWE depends on both. This results in the secret key and private key for M-LWE being quite large, as they are always matrices or even 3D-tensors. In contrast, the secret key for Plain LWE and R-LWE is the same, but the dimensions are larger for the Plain LWE public key, which consists of a matrix and a vector, in contrast to two vectors in R-LWE.

One significant deficiency of Plain LWE is that only a single bit can be encoded at a time. Consequently, the resulting encrypted messages are of the form $\ell \times (\mathbb{Z}_q^{n}\times\mathbb{Z}_q)$, where $\ell$ is the number of bits that needs to be encoded. In contrast, R-LWE and M-LWE permit the encryption of $\ell$ bits in chunks of size $d$. If the number of bits, denoted by $\ell$, is a multiple of the dimension $d$, then the transformation of each message bit into two ciphertext integers is applicable to R-LWE. In contrast, for Plain LWE and M-LWE, each message bit is transformed into $n+1$ ciphertext integers.  
The security of Plain LWE is entirely reliant on the size of $n$, whereas in M-LWE, it is a combination of $n$ and $d$. In this context, it is possible to conclude that $n$ can be smaller in M-LWE than in Plain LWE. Consequently, it can be stated that R-LWE has the smallest cipher text dimension per bit, after which comes M-LWE, and the largest one has the Plain-LWE algorithm.

\begin{table}[htbp]
  \caption[LWE dimensions]{Comparison between the dimensions for keys and messages for the different LWE Types}
  \label{table:LweKeys}
  \centering
  \begin{tabular}{|c|l|l|l|}
    \hline
         & Plain LWE                                        & R-LWE                                                            & M-LWE                                                                    \\
    \hline
    $sk$ & $\mathbb{Z}_q^{n}$                               & $R_q^{d}$                                                        & $R_q^{n\times d}$                                                        \\
    $pk$ & $\mathbb{Z}_q^{n\times n}\times\mathbb{Z}_q^{n}$ & $R_q^{d}\times R_q^{d}$                                          & $R_q^{n\times n \times d}\times R_q^{n \times d}$                        \\
    $m$  & $\ell \times \mathbb{Z}_2$                       & $\left\lceil \ell / d\right\rceil \times R_2^{d}$                & $\left\lceil \ell / d\right\rceil \times R_2^{d}$                        \\
    $ct$ & $\ell\times(\mathbb{Z}_q^{n}\times\mathbb{Z}_q)$ & $\left\lceil \ell / d\right\rceil \times(R_q^{d}\times R_q^{d})$ & $\left\lceil \ell / d\right\rceil \times(R_q^{n\times d}\times R_q^{d})$ \\    \hline
  \end{tabular}
\end{table}

So in total it can be stated, that R-LWE has the smallest overall dimensions for the keys and for the ciphertext. Plain-LWE in contrast to M-LWE has a smaller key space, but the ciphertext space per bit is bigger.

\section{Criteria for comparing LWE-based encryption schemes}
\label{sec:LweComparisonCriteria}

To evaluate the efficacy of distinct cryptographic protocols, a multitude of quantitative and qualitive criteria can be employed, as outlined in reference \cite{CryptoMetrics}. First, one must consider the theoretical security of the algorithm. In the event that theoretical vulnerabilities are identified, the algorithms can be modified to prohibit certain assumptions that may have led to the vulnerability, or if the problem is fundamental, the algorithm may not be suitable for use. The theoretical security of LWE, R-LWE, and M-LWE has already been demonstrated (see \cite{Regev2005OnLL}, \cite{RLWEproof}, and \cite{MLWEproof}, respectively).

In addition to theoretical considerations, there are numerous practical factors to be taken into account. As is the case with any software, there are a number of security concerns that must be addressed, such as bugs and implementation vulnerabilities. Additionally, practical concerns, including the time and memory costs of the algorithms, must be considered. This thesis will focus on these latter two aspects, as they are relatively straightforward to test and are of significant relevance when considering the potential real-world applications of the algorithms. A comparative analysis of the security features of the various schemes will not be undertaken, as it would require a significant investment of time and resources that would exceed the scope of this thesis.

The \textit{time costs} are derived from the amount of computation required to generate the keys, encrypt, or decrypt the data. The most precise number can be obtained by counting the elementary operations a CPU needs to perform in order to run the algorithms. To simplify this somewhat, the run time for different algorithms will be used to obtain a \textit{time cost}, which can be used for comparisons. In order to obtain a meaningful value, multiple runs will be conducted for the same algorithm, and the median value will be used. This should reduce the noise and return comparable values. This time cost will be called performance of the algorithms.

The \textit{size cost} refers to the runtime and memory requirements for computation and the storage necessary for the storage of obtained results. The storage size in question also holds significance not only for the system in which the algorithm is initially executed, but also for other systems as it represents the data to be transmitted. In this thesis, our focus will be on the storage size, as its easy to compute and compare.

When comparing the three LWE schemes, the primary distinctions are related to two dimension variables, matrix dimension $n$ and polynomial rank $d$. As outlined in previous section, when $n>1$ and $d=1$, an Plain-LWE scheme is obtained. Conversely, when $n=1$ and $d>1$, the resulting scheme is R-LWE, and when $n>1$ and $d>1$, the scheme is M-LWE. The storage size of the output values of the different steps can be compared based on these two dimension variables. The outputs in question are the secret key $sk$, the public key $pk$ and the ciphertext $ct$. To create a comparative analysis of the models performance, an example implementation in Python will be constructed and the resulting data will be evaluated. As the same algorithm, with varying dimensions ($n$ and $d$), can be utilized to describe all three schemes, the relative performance between them can be effectively assessed. However, given that Python is a relatively slow-performing language and the algorithm is not optimized, the absolute values may not be entirely meaningful. Nevertheless, the relative performance between the schemes should remain highly comparable.

The three LWE schemes can be compared to each other based on the aforementioned criteria. The following chapter will present the construction of a homomorphic encryption scheme based on these three schemes. Subsequently, the criteria will be extended to facilitate a comparison between the homomorphic schemes.
