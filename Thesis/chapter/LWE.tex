\chapter{Learning with Errors}
\label{LWE}

This chapter will examine the Learning with Errors (LWE) algorithm (referred to as Plain LWE in this thesis) and its variants, namely Ring LWE (R-LWE) and Module LWE (M-LWE). The objective is to develop a basic LWE-based encryption scheme that will serve as the foundation for a homomorphic encryption system.

\section{The Learning with Errors Problem}
\label{sec:LweProblem}

In 2005, Regev initially described the LWE problem \cite{Regev2005OnLL}. He also demonstrated its hardness, but a detailed examination of this proof is beyond the scope of this discussion. The fundamental concept is based on a discrete system of linear equations, where error values are incorporated. This linear system of equations is composed of four fundamental components: the coefficients $\textbf{A}$, the variables $\textbf{s}$, the additional error $\textbf{e}$, and the results $\textbf{b}$. 

$$
  \setlength\arraycolsep{0pt}
  \begin{array}{ c  >{{}}c<{{}} c  >{{}}c<{{}}  c >{{}}c<{{}} c >{{}}c<{{}}  c @{{}={}} c }
    \textbf{A}_{11}\textbf{s}_1 & + & \textbf{A}_{12}\textbf{s}_2 & + & \cdots & + & \textbf{A}_{1m}\textbf{s}_m & + & \textbf{e}_1 & \textbf{b}_1 \\
    \textbf{A}_{21}\textbf{s}_1 & + & \textbf{A}_{22}\textbf{s}_2 & + & \cdots & + & \textbf{A}_{2m}\textbf{s}_m & + & \textbf{e}_2 & \textbf{b}_1 \\
    \vdots                      &   & \vdots                      &   & \vdots &   & \vdots                      &   & \vdots       & \vdots       \\
    \textbf{A}_{n1}\textbf{s}_1 & + & \textbf{A}_{n2}\textbf{s}_2 & + & \cdots & + & \textbf{A}_{nm}s_m          & + & \textbf{e}_n & \textbf{b}_n \\
  \end{array}
$$

This can be expressed as a matrix and vectors in more mathematical terms, as illustrated in equation \ref{eq:mainLwe}. This is the fundamental equation that underlies all LWE problems. The majority of the observed differences can be attributed to the specific ring or matrix/vector dimensions employed in each case. To construct a plain LWE scheme, it is necessary to set the ring as $R = \mathbb{Z}_q = \mathbb{Z}/q$. In all LWE schemes presented in this thesis, $n = m$.

\begin{equation}
  \begin{split}
     & \textbf{A}\cdot \textbf{s} + \textbf{e}= \textbf{b}                                      \\
     & \textbf{A} \in R_q^{n \times m}, \textbf{s} \in R_q^m, \textbf{e}, \textbf{b} \in R_q^n
  \end{split}
  \label{eq:mainLwe}
\end{equation}

The construction of an asymmetric encryption scheme necessitates the presence of two distinct keys: a secret key, denoted by $sk$, and a public key, represented by $pk$. The public key is used to encrypt a message, generating a ciphertext. However, it is not possible to decrypt the ciphertext with this key in order to recreate the original message. This can only be accomplished with the secret key. In order for this scheme to be operational, it is crucial to ensure that it is not possible to generate the secret key from ciphertexts and/or the private key. For further insights into the general structure of asymmetric encryption, please refer to \cite{Eckert2018}.

In the context of LWE, $\textbf{s}$ represents the secret key, while both $\textbf{A}$ and $\textbf{b}$ collectively serve as the private key. The error term, $\textbf{e}$, will be discarded after the calculation is complete, thus preventing any individual from gaining knowledge of it. If one possesses both the secret and private key, the system of linear equations can be efficiently solved using the Gaussian algorithm. However, for an individual with access only to the private key, the number of unknowns is excessive, rendering the system unsolvable. Despite its seemingly straightforward nature, this equation is, in fact, quite challenging to resolve. The difficulty can be reduced to variants of the Shortest Vector Problem (SVP), which describes the complexity of identifying the shortest vector in a lattice. While this is a relatively straightforward undertaking in smaller dimensions, it becomes increasingly challenging as the dimensions increase.

\info[inline]{Maybe going more into detail of the SVP or general the hardness of the LWE Problem?}

\section{LWE based encryption scheme}
\label{sec:Lwe-Encryption}

In this section, an elementary asymmetric encryption scheme will be constructed based on the principal equation \ref{eq:mainLwe}. This construction is loosely based on the CRYSTALS-Kyber encryption scheme \cite{CyrstalsKyber}, but with some simplifications. The objective is not to develop a practical and secure encryption scheme, but rather one that can serve as a foundation for subsequent investigations into the construction of homomorphic encryption. Additionally, the aim is to transform the scheme from Plain-LWE into R-LWE and M-LWE.

In order to construct a functional encryption scheme, it is necessary to develop three algorithms:

\begin{enumerate}
  \item \textbf{KeyGen}: For generating the private and secret key
  \item \textbf{Encryption}: For encrypting some message $m$ with the private key $pk$ creating a ciphertext $ct$
  \item \textbf{Decryption}: For decrypting the ciphertext $ct$ with the secret key $sk$ retrieving the original message $m$
\end{enumerate}

In order to construct a Plain-LWE encryption scheme, all calculations are done in the ring $R = \mathbb{Z}_q$, where $q$ is the modulus. If values from $R$ are chosen uniformly, this is denoted by $x \leftarrow R$. Otherwise, if small values are chosen from $R$, this is written as $x \leftarrow \chi_R$. This can be done by choosing uniformly from a set of small numbers all in $R$ (e.g., ${-4,\ldots, 4}$ if $q$ is big enougth), or by choosing from an error distribution, such as the discrete Gaussian, as described in \cite{Regev2005OnLL}.

The initial stage of the process is to generate the private key, $pk$, and the secret key, $sk$. This is detailed in Algorithm \ref{alg: SampleLweKeyGen}. It uses the LWE equation from \ref{eq:mainLwe} as described above. The secret key, which the owner should never share, is the vector $\textbf{s}$. The public key $pk$, which can be shared, consists of the transformation matrix $\textbf{A}$ and the transformed secret key plus the error $\textbf{b}$. The error $\textbf{e}$ is discarded after the computation of $\textbf{b}$. The values of $\textbf{e}$ and $\textbf{s}$ should be rather small, and $\textbf{A}$ is uniformly sampled from $R$.

\begin{algorithm}[htb]
  \begin{algorithmic}[1]
    \STATE $\textbf{s} \leftarrow \chi_R^n$
    \STATE $\textbf{A} \leftarrow R^{n \times n}$
    \STATE $\textbf{e} \leftarrow \chi_R^n$
    \STATE $\textbf{b} = \textbf{A}\cdot \textbf{s}+\textbf{e}$
    \RETURN $(pk:=(\textbf{A}, \textbf{b}), sk:=\textbf{s} )$
  \end{algorithmic}
  \caption{Sample LWE: KeyGen}
  \label{alg: SampleLweKeyGen}
\end{algorithm}

The encryption algorithm, which describes how to generate the ciphertext $ct$ from an message $m$ with the public key $pk$ is described in Algorithm \ref{alg: SampleLweEncryption}. The errors $\textbf{e}_1$ and $e_2$ are randomly sampled with small values and used to create more uncertainty around the message. The same message can therefore be encrypted with different errors, which yields different ciphertexts. This makes it harder for attackers to find patterns in the decryption. The value $\textbf{r}$ is sampled randomly form $0$ and $1$. The objective is to select a subset of $\textbf{A}$ and $\textbf{b}$ given that approximately $50\%$ of the values in $\textbf{r}$ will be $0$. Consequently, these columns in $\textbf{A}$ and $\textbf{b}$ are effectively irrelevant (multiplied by $0$). This helps to create more entropy between different encryptions, as a different subset of $\textbf{A}$ and $\textbf{b}$ will be used to encrypt each time.\\
The newly established values and the public key are employed in the calculation of two values: $\textbf{u}$ and $v$. The first term, $\textbf{u}$, is an transposed subset of $\textbf{A}$ with some error: $\textbf{u} = \textbf{A}^T \cdot \textbf{r} + \textbf{e}_1$. It can be regarded as the inverse of $\textbf{b}$, with the secret term, $\textbf{s}$, omitted. The term $v$ represents the actual message value, which is calculated from a subset of the variable $b$ with a small error value added and the scaled message $m'$. The complete calculation is as follows: $v = \textbf{b}^T \cdot \textbf{r} + e_2 + (m\cdot \left\lfloor q/2\right\rfloor)$. 
The scaled message, denoted by $m'$, is obtained by multiplying the original message, $m$, by the rounded-down version of half the modulus. This operation results in the values of the message in the ring, which are either $0$ or $1$, being scaled from the message space into the ciphertext space, ($0$ to $q-1$). The resulting values are approximately as distant from each other as possible, which is $q/2$.

\begin{algorithm}[htb]
  \begin{algorithmic}[1]
    \REQUIRE $m \in \mathbb{Z}_2 = \{0, 1\}$, $pk = (\textbf{A}, \textbf{b})$
    \STATE $\textbf{r} \leftarrow \{0, 1\}^n$
    \STATE $\textbf{e}_1 \leftarrow \chi_R^n$
    \STATE $\textbf{u} = \textbf{A}^T \cdot \textbf{r} + \textbf{e}_1$
    \STATE $e_2 \leftarrow \chi_R$
    \STATE $v = \textbf{b}^T \cdot \textbf{r} + e_2 + (m\cdot \left\lfloor q/2\right\rfloor)$
    \RETURN $ct := (\textbf{u}, v)$
  \end{algorithmic}
  \caption{Sample LWE: Encryption}
  \label{alg: SampleLweEncryption}
\end{algorithm}

The decryption of the ciphertext $ct$ back to the original message $m$ using the secret key $sk$ is described in Algorithm \ref{alg: SampleLweDecryption}.

\begin{algorithm}[htb]
  \begin{algorithmic}[1]
    \REQUIRE $ct = (\textbf{u}, v)$, $sk = \textbf{s}$
    \RETURN $\left\lfloor \frac{1}{\left\lfloor q/2\right\rfloor}\cdot \left[v-\textbf{s}^T \cdot \textbf{u}\right]_q\right\rceil _2$
  \end{algorithmic}
  \caption{Sample LWE: Decryption}
  \label{alg: SampleLweDecryption}
\end{algorithm}


To get a better understanding of the equation, consider the following simplification of the term in algorithm \ref{alg: SampleLweDecryption}.

\begin{align*}
   & \left\lfloor \frac{1}{\left\lfloor q/2\right\rfloor}\cdot \left[v-\textbf{s}^T \cdot \textbf{u}\right]_q\right\rceil _2                                                                                                                                                                                                              \\
   & = \left\lfloor \frac{1}{\left\lfloor q/2\right\rfloor}\cdot \left[\textbf{b}^T \cdot \textbf{r} + e_2 + (m\cdot \left\lfloor q/2\right\rfloor)-\textbf{s}^T \cdot (\textbf{A}^T \cdot \textbf{r} + \textbf{e}_1)\right]_q \right\rceil _2                                                                                            \\
   & = \left\lfloor \frac{1}{\left\lfloor q/2\right\rfloor}\cdot \left[(\textbf{As}+\textbf{e})^T \cdot \textbf{r} + e_2 + (m\cdot \left\lfloor q/2\right\rfloor)-\textbf{s}^T \textbf{A}^T \cdot \textbf{r} - \textbf{s}^T \textbf{e}_1\right]_q \right\rceil _2                                                                         \\
   & = \left\lfloor \frac{1}{\left\lfloor q/2\right\rfloor}\cdot \left[(\textbf{As})^T \cdot \textbf{r} + \textbf{e}^T\textbf{r}+ e_2 + (m\cdot \left\lfloor q/2\right\rfloor)-(\textbf{As})^T \cdot \textbf{r} - \textbf{s}^T \textbf{e}_1\right]_q\right\rceil _2                                                                       \\
   & = \left\lfloor \frac{1}{\left\lfloor q/2\right\rfloor}\cdot \left[\textbf{e}^T\textbf{r}+ e_2 + (m\cdot \left\lfloor q/2\right\rfloor)- \textbf{s}^T \textbf{e}_1\right]_q\right\rceil _2                                                                                                                                            \\
   & = \left\lfloor \frac{\textbf{e}^T\textbf{r}}{\left\lfloor q/2\right\rfloor}+ \frac{e_2 }{\left\lfloor q/2\right\rfloor}+ m - \frac{\textbf{s}^T \textbf{e}_1}{\left\lfloor q/2\right\rfloor}\right\rceil _2                                                                                                                          \\
   & = \left\lfloor m' \right\rceil _2  = \begin{cases}
                                            m \in \{0,1\} & \text{if } \left\lvert \frac{\textbf{e}^T\textbf{r}}{\left\lfloor q/2\right\rfloor}+ \frac{e_2 }{\left\lfloor q/2\right\rfloor} - \frac{\textbf{s}^T \textbf{e}_1}{\left\lfloor q/2\right\rfloor}\right\rvert  < \frac{q}{4} \\
                                            error         & \text{otherwise}
                                          \end{cases}
\end{align*}

As demonstrated by the calculation, by multiplying the cancellation term $\textbf{u}$ with the secret $\textbf{s}$, the transformation $(\textbf{As})^T \cdot \textbf{r}$ in $v$ can be canceled out. This results in the message with some error values being added to it. The erroneous message will then be rounded, which will result in the original message. This process will only be successful if the absolute value of all error terms together is smaller than $\frac{q}{4}$. This is due to the fact that the possible values in the message are separated by a distance of $\frac{q}{2}$ from each other. Consequently, all values between $-\frac{q}{4}\mod q=\frac{3q}{4}$ and $\frac{q}{4}$ are rounded back to $0$, while all values between $\frac{q}{4}$ and $\frac{3q}{4}$ are rounded to $1$. Consequently, provided that the message (either $0$ or $\frac{q}{2}$) is not shifted by more than $\frac{q}{4}$, it will remain within the rounding area of the original message. In the event that the error terms exceed the value of $\frac{q}{4}$, the resulting message will be a erroneous bit, which consequently leads to a corrupted message.

\info[inline]{Maybe add an image which shows the rounding with an clock}

The current definition of this algorithm allows only $1$ bit to be encoded at the time. This could be improved with some tricks, but for simplicity reasons we wont do that here. 

To observe the functioning of this algorithm in practice, please refer to the example calculation in the appendix, which can be found in Appendix \ref{app:PlainLweCalc}.

\section{Transforming LWE to R-LWE and M-LWE}
\label{sec:TransformingLweToRlweAndMlwe}

To transform the Plain-LWE encryption scheme described above into Ring-LWE encryption scheme, only a few changes need to be made. Most importantly, a polynomial ring will be defined as $R = \mathbb{Z}[x]_q/(x^d+1)$, with the matrix dimension $n=1$ and the degree $d>1$. Because $n$ equals $1$ all vectors and matrices only contain a single value, which is a polynomial, as $d$ is now greater than $1$. Consequently, instead of having a vector $\textbf{r}$, it will now be a polynomial in the ring $R$, where all coefficients are either $0$ or $1$. The message to be encrypted is also transformed into a polynomial in $R$ with the message bits being the coefficients of the polynomial. Because of this, $d$ bits can now be encoded in one message. Instead of using matrix arithmetic, polynomial arithmetic is now utilized. However, as stated in section \ref{sec:PolyMulMath}, the polynomial arithmetic in the ring can also be transformed into matrix arithmetic. All equations stay the same and the structure of the Algorithms does not change, only the values which are in- and outputted. An full example of the three-step process for RLWE can be found in \ref{app:RlweExampleCalc}.

In a similar way, Plain-LWE was transformed into R-LWE, R-LWE can be transformed into Module-LWE. Todo this, only the matrix dimension needs to be increase, so that $n>1$. So instead of working with polynomials as in R-LWE, matrices and vectors of polynomials will be used. Thus a combination of matrix and polynomial arithmetic a necessary for computation. An example of such an calculation can found in Appendix \ref{app:MlweExampleCalc}.

Given the differing dimensions of the matrix $n$ and polynomial degree $d$ among the various types of LWE, the variables and keys that must be stored and shared, as well as the messages themselves, exhibit a corresponding variation in dimension. An summarized overview of the differences can be found in table \ref{table:LweDiffs}.


\begin{table}[htbp]
  \caption[LWE variables shape comparison]{Comparison between the shapes of the internal \& external variables for the different LWE Types. The polynomial degree is treated as a vector for better comparison. For the message and the ciphertext $\ell$ refers to the length (number of bits) of the message.}
  \label{table:LweDiffs}
  \centering
  \begin{tabular}{|c|l|l|l|}
    \toprule
                                                    & Plain LWE                                        & R-LWE                                                                              & M-LWE                                                                                      \\
    \midrule
    $\textbf{A}$                                    & $\mathbb{Z}^{n\times n}_q$                       & $\mathbb{Z}^d_q$                                                                   & $\mathbb{Z}^{n\times n \times d}_q$                                                        \\
    $\textbf{s},\textbf{b},\textbf{e},\textbf{e}_1$ & $\mathbb{Z}^{n}_q$                               & $\mathbb{Z}^d_q$                                                                   & $\mathbb{Z}^{n\times d}_q$                                                                 \\
    $e_2$                                           & $\mathbb{Z}_q$                                   & $\mathbb{Z}^d_q$                                                                   & $\mathbb{Z}^d_q$                                                                           \\
    $\textbf{r}$                                    & $\mathbb{Z}_2^n$                                 & $\mathbb{Z}^d_2$                                                                   & $\mathbb{Z}_2^{n \times d}$                                                                \\
    \midrule
    $sk$                                            & $\mathbb{Z}_q^{n}$                               & $\mathbb{Z}_q^{d}$                                                                 & $\mathbb{Z}_q^{n\times d}$                                                                 \\
    $pk$                                            & $\mathbb{Z}_q^{n\times n}\times\mathbb{Z}_q^{n}$ & $\mathbb{Z}_q^{d}\times \mathbb{Z}_q^{d}$                                          & $\mathbb{Z}_q^{n\times n \times d}\times \mathbb{Z}_q^{n \times d}$                        \\
    $m$                                             & $\ell \times \mathbb{Z}_2$                       & $\left\lceil \ell / d\right\rceil \times \mathbb{Z}_2^{d}$                         & $\left\lceil \ell / d\right\rceil \times \mathbb{Z}_2^{d}$                                 \\
    $ct$                                            & $\ell\times(\mathbb{Z}_q^{n}\times\mathbb{Z}_q)$ & $\left\lceil \ell / d\right\rceil \times(\mathbb{Z}_q^{d}\times \mathbb{Z}_q^{d})$ & $\left\lceil \ell / d\right\rceil \times(\mathbb{Z}_q^{n\times d}\times \mathbb{Z}_q^{d})$ \\  
    
    \bottomrule
  \end{tabular}
\end{table}

As illustrated in the aforementioned table, Plain LWE and R-LWE are each dependent on a single variable, either $n$ or $d$, respectively. The internal and external variables for Plain-LWE are frequently distinct in their shape. They include matrices, such as $A$, as well as elements of the private key, in addition to scalar values, such as $e_2$, and elements of the ciphertext. However, the majority of values are vectors. In contrast, R-LWE is characterized by all values being vectors of size $d$. M-LWE, on the other hand, is dependent on both variables, as it is a generalization of the other two. This results in all variables for M-LWE having the combined size of Plain and R-LWE. Consequently, the smallest values are vectors of size $d$, while the largest ones are 3D tensors of size $n \times n \times d$.

One notable shortcoming of Plain LWE is that only a single bit can be encoded at a time. Consequently, the resulting encrypted messages are of the form $\ell \times (\mathbb{Z}_q^{n}\times\mathbb{Z}_q)$, where $\ell$ is the number of bits that needs to be encoded. In contrast, R-LWE and M-LWE permit the encryption of $\ell$ bits in blocks of size $d$. If the number of bits is a multiple of the dimension $d$, it can be split into $\ell / d$ blocks. Otherwise, the last block must be padded with zeros to create a full block. Consequently, in the worst case, an additional encryption with a polynomial of size $d$ is required to encrypt a single bit.

The security of Plain-LWE is dependent on the matrix dimension $n$, whereas R-LWE is reliant on the polynomial degree $d$. Conversely, M-LWE represents a combination of the matrix dimension $n$ and the polynomial degree $d$.  
Furthermore, the security of all three schemes hinges on the modulus $q$. It can be inferred that the value of $n$ can be smaller in M-LWE than in Plain LWE, and that the value of $d$ can be smaller then that in R-LWE. This can be verified by examining three distinct LWE-based encryption schemes and their recommended security parameters, as illustrated in Table \ref{table:LweParameterComparison}. As can be observed in this table, the large value of $n$ for Plain-LWE results in matrices that are quite large. In contrast, the vectors used in R-LWE are significantly smaller, and the matrices in M-LWE are of a size somewhere between those of Plain-LWE and R-LWE. A more detailed analysis of the size cost implications of these schemes will be presented in Chapter \ref{sec:sizeCostComparison}.

\begin{table}[htp]
  \centering
  \caption{Variable size comparison of different LWE based encryption schemes}
  \begin{tabular}{|l|c|c|c|c|}
    \toprule
              & Source                                             & $n$   & $d$   & $q$     \\
    
    \midrule
    Plain-LWE & Frodo \cite{frodo}                                 & $752$ &       & $32767$ \\
    R-LWE     & Practical Key Exchange \cite{PracticalKeyExchange} &       & $512$ & $25601$ \\
    M-LWE     & CRYSTALS-Kyber \cite{CyrstalsKyber}                & $3$   & $256$ & $7681$  \\   
    \bottomrule
  \end{tabular}
  \label{table:LweParameterComparison}
\end{table}

\section{Criteria for comparing LWE-based encryption schemes}
\label{sec:LweComparisonCriteria}

To evaluate the efficacy of distinct cryptographic protocols, a multitude of quantitative and qualitative criteria can be employed, as outlined in reference \cite{CryptoMetrics}. First, one must consider the theoretical security of the algorithm. In the event that theoretical vulnerabilities are identified, the algorithms can be modified to prohibit certain assumptions that may have led to the vulnerability, or if the problem is fundamental, the algorithm may not be suitable for use. The theoretical security of LWE, R-LWE, and M-LWE has already been demonstrated (see \cite{Regev2005OnLL}, \cite{RLWEproof}, and \cite{MLWEproof}, respectively).

In addition to theoretical considerations, there are numerous practical factors to be taken into account. As is the case with any software, there are a number of security concerns that must be addressed, such as bugs and implementation vulnerabilities. Additionally, practical concerns, including the time and memory costs of the algorithms, must be considered. This thesis will focus on these latter two aspects, as they are relatively straightforward to test and are of significant relevance when considering the potential real-world applications of the algorithms. A comparative analysis of the security features of the various schemes will not be undertaken, as it would require a significant investment of time and resources that would exceed the scope of this thesis.

The \textit{time costs} are derived from the amount of computation required to generate the keys, encrypt, or decrypt the data. The most precise number can be obtained by counting the elementary operations a CPU needs to perform in order to run the algorithms. To simplify this somewhat, the run time for different algorithms will be used to obtain a \textit{time cost}, which can be used for comparisons. In order to obtain a meaningful value, multiple runs will be conducted for the same algorithm, and the median value will be used. This should reduce the noise and return comparable values. This time cost will be called performance of the algorithms.

The \textit{size cost} refers to the runtime and memory requirements for computation and the storage necessary for the storage of obtained results. The storage size in question also holds significance not only for the system in which the algorithm is initially executed, but also for other systems as it represents the data to be transmitted. In this thesis, our focus will be on the storage size, as its easy to compute and compare.

The three LWE schemes can be compared to each other based on the aforementioned criteria. The following chapter will present the construction of a homomorphic encryption scheme based on these three schemes. Subsequently, the criteria will be extended to facilitate a comparison between the homomorphic schemes.
