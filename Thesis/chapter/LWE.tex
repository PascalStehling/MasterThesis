\chapter{Learning with Errors}
\label{LWE}

In this section, we will take a closer look at the Learning with Errors (LWE) algorithm (also called Plain LWE) and its different versions, namely Ring LWE (R-LWE) and Module LWE (M-LWE).

\section{The Learning with Errors Problem}

In 2005, Regev first described the LWE problem \cite{Regev2005OnLL}. He also proved its hardness, but we won't go into those details here. The basic idea is to add an error vector to a linear system of equations. This makes the normally trivially solvable system surprisingly hard to solve.

In more mathematical terms $\mathbb{Z}_q = \mathbb{Z}/n$, $A \in \mathbb{Z}_q^{n \times m}$, $s \in \mathbb{Z}_q^m$, $b \in \mathbb{Z}_q^n$, with which we can form the linear equation system, where $A$ and $b$ are given and the vector $s$ represents the unknowns, which we want to retrieve

$$A\cdot s = b$$

This can be easily solved with the Gaussian algorithm. But by adding just a small error vector $e \in \mathbb{Z}_q^n$, this gets surprisingly hard, as not just $s$ is unknown, but also $e$.

$$
  A \cdot s + e = b
$$

This equation is the main equation on which all LWE problems are based on. Most of the differences will come from the Ring or the dimensions that are used.

\section{LWE based encryption scheme}
\label{sec:Lwe-Encryption}

In this section, we will describe a simple LWE-based encryption scheme and how it can be converted to R-LWE and M-LWE. The following algorithm is loosely based on the Kyber \cite{CyrstalsKyber} scheme, with some simplifications.

All calculations are done in the ring $R = \mathbb{Z}_q$, where $q$ is the modulus. If values from $R$ are chosen uniformly, this is denoted by $x \leftarrow R$. Otherwise, if small values are chosen from $R$, this is written as $x \leftarrow \chi_R$. This can be done by choosing uniformly from a set of small numbers all in $R$ (e.g., ${-4,\ldots, 4}$ if $q>50$), or by choosing from an error distribution, such as the discrete Gaussian, as described in \cite{Regev2005OnLL}.

The following three algorithms describe the example schema. Algorithm \ref{alg: SampleLweKeyGen}, the key generation, describes how to generate the private key $pk$ and the secret key $sk$. It uses the LWE problem as described above. The secret key, which the owner should never share, is the vector $s$. The public key $pk$, which can be shared, consists of the transformation matrix $A$ and the transformed secret key plus the error $b$. The error $e$ is discarded after the computation of $b$. The values of $e$ and $s$ should be rather small, and $A$ is uniformly sampled from $R$.

\begin{algorithm}[htb]
  \begin{algorithmic}[1]
    \STATE $s \leftarrow \chi_R^n$
    \STATE $A \leftarrow R^{n \times n}$
    \STATE $e \leftarrow \chi_R^n$
    \STATE $b = A\cdot s+e$
    \RETURN $(pk:=(A, b), sk:=s )$
  \end{algorithmic}
  \caption{Sample LWE: KeyGen}
  \label{alg: SampleLweKeyGen}
\end{algorithm}

Algorithm \ref{alg: SampleLweEncryption}, the encryption, describes how to encrypt a message $m$ with the public key $pk$. The errors $e_1$ and $e_2$ are randomly sampled with small values and used to create more uncertainty around the message. The same message can therefore be decrypted with different errors and yield different values. This makes it harder for attackers to find patterns in the decryption. The idea behind $r$ is to select a subset of $A$ and $b$, since $~50\%$ of the values in $r$ will be $0$, meaning that these columns in $A$ and $b$ are irrelevant (multiplied by $0$). This helps to create more entropy between different encryptions, as a different subset of $A$ and $b$ will be used to encrypt each time.

The new values and the public key are used to calculate two values: $u$ and $v$. The first term, $u$, can be considered the cancel term for $b$, where the secret $s$ is missing. $v$ is the actual value term, which is composed of a subset of $b$ with some small error added and the scaled message $m'$. For the scaled message $m' = m*\left\lfloor q/2\right\rfloor$, the message is multiplied with the rounded down version of half the modulus. This operation results in the values of the message $0$ and $1$ in the ring being approximately as distant from each other as possible.

\begin{algorithm}[htb]
  \begin{algorithmic}[1]
    \REQUIRE $m \in \mathbb{Z}_2 = \{0, 1\}$, $pk = (A, b)$
    \STATE $r \leftarrow \{0, 1\}^n$
    \STATE $e_1 \leftarrow \chi_R^n$
    \STATE $u = A^T \cdot r + e_1$
    \STATE $e_2 \leftarrow \chi_R$
    \STATE $v = b^T \cdot r + e_2 + (m*\left\lfloor q/2\right\rfloor)$
    \RETURN $m_{enc} := (u, v)$
  \end{algorithmic}
  \caption{Sample LWE: Encryption}
  \label{alg: SampleLweEncryption}
\end{algorithm}

Algorithm \ref{alg: SampleLweDecryption}, the Decryption, describes how to decrypt an encrypted message $m_{enc}$ using the secret key $sk$.

As demonstrated by the calculation, by multiplying the cancellation term $u$ with the secret $s$, the transformation $(As)^T \cdot r$ in $v$ can be canceled out. This results in the message and some error values being added to it. The erroneous message will then be rounded, which will result in the original message. This process will only be successful if all error terms together are smaller than $\frac{q}{4}$. This is due to the fact that the possible values in the message are separated by a distance of $\frac{q}{2}$ from each other. Consequently, all values between $-\frac{q}{4}\mod q=\frac{3q}{4}$ and $\frac{q}{4}$ are rounded back to $0$, while all values between $\frac{q}{4}$ and $\frac{3q}{4}$ are rounded to $1$. Consequently, provided that the message (either $0$ or $\frac{q}{2}$) is not shifted by more than $\frac{q}{4}$, it will remain within the rounding area of the original message.


\begin{algorithm}[htb]
  \begin{algorithmic}[1]
    \REQUIRE $m_{enc} = (u, v)$, $sk = s$
    \RETURN \begin{align*}
      & \left\lfloor \frac{1}{\left\lfloor q/2\right\rfloor}*(v-s^T \cdot u)\right\rceil _2                                                                                   \\
        & = \left\lfloor \frac{1}{\left\lfloor q/2\right\rfloor}*(b^T \cdot r + e_2 + (m*\left\lfloor q/2\right\rfloor)-s^T \cdot (A^T \cdot r + e_1))\right\rceil _2             \\
        & = \left\lfloor \frac{1}{\left\lfloor q/2\right\rfloor}*((As+e)^T \cdot r + e_2 + (m*\left\lfloor q/2\right\rfloor)-s^T A^T \cdot r - s^T e_1)\right\rceil _2            \\
        & = \left\lfloor \frac{1}{\left\lfloor q/2\right\rfloor}*((As)^T \cdot r + e^Tr+ e_2 + (m*\left\lfloor q/2\right\rfloor)-(As)^T \cdot r - s^T e_1)\right\rceil _2         \\
        & = \left\lfloor \frac{1}{\left\lfloor q/2\right\rfloor}*(e^Tr+ e_2 + (m*\left\lfloor q/2\right\rfloor)- s^T e_1)\right\rceil _2                                          \\
        & = \left\lfloor \frac{e^Tr}{\left\lfloor q/2\right\rfloor}+ \frac{e_2 }{\left\lfloor q/2\right\rfloor}+ m - \frac{s^T e_1}{\left\lfloor q/2\right\rfloor}\right\rceil _2 \\
        & = \left\lfloor m' \right\rceil _2                                                                                                                                       \\
        & = m \in \{0,1\}
    \end{align*}
  \end{algorithmic}
  \caption{Sample LWE: Decryption}
  \label{alg: SampleLweDecryption}
\end{algorithm}

To observe the functioning of this algorithm in practice, please refer to the example calculation in the appendix, which can be accessed in Appendix \ref{app:PlainLweCalc}.