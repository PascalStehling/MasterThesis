\documentclass{fiwthesis}

% ========
%  Pakete
% ========

\usepackage{textgreek}           % griechische Buchstaben außerhalb des Math-Mode
\usepackage{amsmath}             % zentrierte Formeln
\usepackage{amssymb}             % erweiterter Formelsatz mathem. Symbole

\usepackage{boldline}            % breitere Linien in Tabellen
\usepackage{booktabs}            % typographisch richtige Tabellen setzen
\usepackage{tabularx}            % Erweiterte Tabellendarstellung
\usepackage{multirow}            % Spalte über mehrere Zeilen oder Spalten ausdehnen
\usepackage{xltabular}           % Zeilenumbrüche in tabularx erlauben

\usepackage{graphicx}            % ermöglicht das Einbinden von Grafiken
\usepackage{subcaption}          % mehrere Bilder in einem Bild
\usepackage{pgfplots}            % Grafiken erzeugen
\usepackage{smartdiagram}        % schnelle und einfache Grafiken

\newtheorem{definition}{Definition}

% Commands for TODOs
% \usepackage[pdftex,dvipsnames]{xcolor}  % Coloured text etc.

\usepackage{xargs}
\setlength{\marginparwidth }{2cm}
\usepackage[colorinlistoftodos,prependcaption,textsize=tiny]{todonotes}
\newcommandx{\unsure}[2][1=]{\todo[linecolor=red,backgroundcolor=red!25,bordercolor=red,#1]{#2}}
\newcommandx{\change}[2][1=]{\todo[linecolor=blue,backgroundcolor=blue!25,bordercolor=blue,#1]{#2}}
\newcommandx{\info}[2][1=]{\todo[linecolor=OliveGreen,backgroundcolor=OliveGreen!25,bordercolor=OliveGreen,#1]{#2}}
\newcommandx{\improvement}[2][1=]{\todo[linecolor=Plum,backgroundcolor=Plum!25,bordercolor=Plum,#1]{#2}}
\newcommandx{\thiswillnotshow}[2][1=]{\todo[disable,#1]{#2}}

% ===========
%  Metadaten
% ===========

\thesis{Master-Thesis}
\title{Homomorphic Post-Quantum Cryptography - Evaluation of Module Learning with Error in Homomorphic Cryptography}
\author{Pascal Stehling}
\matrnr{455051}
\bdate{18.12.1997}
\bcity{Wismar}
\supervisor{Prof.~Dr.-Ing.~habil.~Andreas Ahrens}
% \secsupervisor{ZWEITBETREUER}
\keywords{Logik, Mathematik}

% Metadaten in die PDF-Datei schreiben
\makepdfmetadata

% ===============
%  Präambel
% ===============

% PGF Kompatibilitätseinstellung
\pgfplotsset{width=0.95\textwidth,compat=newest}

% % Bibliographie einbinden
\bibliography{quellen}

% Glossar einbinden
% this cant be removed, for some reason the build breaks
\newdualentry{dos}% label
{DoS}% short form
{Denial of Service}% long form
{Ein Denial of Service (im Deutschen: Dienstverweigerung) ist ein Angriffe auf Computer- oder Netzwerksysteme, wobei das Zielsystem durch Überlastung oder durch andere Mittel außer Betrieb gesetzt wird}% description


% Abkürzungen einbinden
\input{verzeichnisse/abkuerzungen}

% Symbole einbinden
\input{verzeichnisse/symbole}

% % Glossar- und Abkürzungsverzeichniserstellung
% \makeglossaries{}

% Index erzeugen
\makeindex[
  intoc=true,
  title=Index,
  columns=2]{}
\indexsetup{headers={\indexname}{\indexname}}

% ===============
%  Eigene Makros
% ===============

\newcommand*{\code}[1]{\texttt{#1}}

\begin{document}

% Titelseite
\maketitle

\maketask{
In 2009, Craig Gentry published the first fully homomorphic encryption (FHE) algorithm in his PhD dissertation. With that it was possible for the first time to do any number of calculations on encrypted messages without having to decrypt them. Since the discovery of this first FHE algorithm, other such algorithms have been developed, but instead of using Ideal Lattices as the mathematical foundation, newer ones tend to use the Learning with Errors (LWE) and Ring-LWE (RLWE) problems.

In July 2022, the American National Institute of Standards and Technology (NIST) published the first group of winners of its competition for quantum-safe algorithms. The winner for general asymmetric encryption was the CRYSTALS-Kyber algorithm, which is based on Module-LWE (MLWE). This is an extension of the RLWE method in which polynomials in higher dimensions (vectors and matrices) are used. Even though it needs more computational power, in contrast to RLWE, it can be offset through parallelization of the calculations and higher security.

With these two distinct developments, the questions can be asked, if it possible to combine them and see if it’s possible to transfer existing FHE cryptosystems, which are based on RLWE to the MLWE method. This will first be researched theoretically and then verified with practical tests. 
These tests will be used to examine the advantages and disadvantages of the different Methods in terms of various properties, such as computation speed, error rate during decryption, number of possible calculations without errors and others. In order to structure these tests and thus establish good comparability, a test concept will be created in this thesis. 
At the end, the following question should be answered: Are their practical advantages to transferring existing FHE systems from RLWE to MLWE or whether the associated increased computing effort nullifies the advantages again?}

\makeabstract{
  Abstract.
}

\maketoc[compact]

% ==========
%  Textteil
% ==========

\chapter{Introduction}
\label{Introduction}

% \section{Background}

In the early months of 1978, one of the most significant cryptographic systems, the RSA system \cite{RSA}, was published. With the advent of the Internet in the 1990s and the subsequent need for secure data transfer, it became one of the most widely used encryption schemes to date. In the subsequent period of slightly more than half a year, two of the authors of the RSA paper published a new concept, based on the RSA concept, which they designated ''privacy homomorphism`` \cite{Rivest1978}. This concept would later be known as homomorphic encryption. This is an encryption system whereby operations can be executed directly on encrypted data, eliminating the necessity of first decrypting it, running the operations, and then encrypting it again. Such a system would not only eliminate the necessity for decryption and encryption at the processing stage, it would also ensure that the plain text is not accessible to the party undertaking this processing. However, at the system's inception, only one operation was feasible: multiplication. To develop a system capable of general computing, the addition operation was necessary as a second operation, as these two operations enable the recreation of all other operations at the bit level. Unfortunately, the creation of a homomorphic encryption scheme with unlimited additions and multiplications, also known as full homomorphic encryption (FHE), proved to be a formidable challenge.\\
In 1994, Peter Shor published his algorithm \cite{Shor}, which describes how a quantum computer could factorize numbers in polynomial time. In contrast to classical computers, for which this problem is categorized as a hard problem. As the RSA cryptosystem is based on this exact issue being hard to solve, it would be possible to find the private key for any public key, thus undermining the cryptosystem's security. Fortunately, no quantum computer capable of such an operation was anywhere near availability at the time, so this problem remained theoretical.\\
Approximately a decade later, in 2005, O. Regev devised a novel mathematical framework, termed Learning with Error (LWE) \cite{Regev2005OnLL}, which enables the construction of new cryptosystems. This framework is based on an error term within a linear system of equations constructed on a lattice. The mathematical problem that he exploits for security is the hardness of the shortest vector problem (SVP). There are variants of this problem, called Ring-LWE, where a polynomial is used instead of a matrix, and M-LWE, which mixes Ring-LWE and the (Plain-)LWE together, resulting in matrices of polynomials. In 2009, Craig Gentry published the first full-homomorphic encryption scheme \cite{Gentry2009AFH}. This development prompted renewed optimism regarding the advancement of FHE schemes, as it became evident that the concept was indeed feasible. However, the primary challenge that remained was the issue of performance. To enhance the efficiency of this scheme, the initial version, which was based on the ideal lattice, was adapted to the R-LWE scheme. Over time, significant advancements have been made in the development of these FHE schemes, which are constructed on basis of R-LWE \cite{FHESurvey}. However, the primary challenge persists, namely the performance, which is frequently 1000s of times slower than operations on the plain text.\\
In recent years, there has been a resurgence of interest in quantum computers as various companies compete to develop the first practical and useful quantum computer \cite{googleQuantumComputing} \cite{ibmQuantumComputing}. Consequently, the performance of these computers has been steadily improving. If the promises made are accurate, it is possible that in 10 years, viable quantum computers will be available on the market. These computers could run Shor's Algorithm and thereby breach the security of RSA (and other) cryptosystems, potentially undermining the security of the internet as it currently stands. To circumvent such potential issues, the US National Institute of Standards and Technology (NIST) initiated an open competition in 2016, wherein individuals could submit novel cryptographic systems for analysis. Research teams from around the globe would then endeavor to identify vulnerabilities in these systems. In 2022, the NIST announced the first four winners \cite{nistAnouncement}, three of which were based on LWE. The two most recommended systems, CRYSTALS-Kyber \cite{CyrstalsKyber} and CRYSTALS-Dilithium \cite{crystalsDilithium}, are both based on M-LWE.


% \section{Goal of this Thesis}

In light of these recent advancements in M-LWE-based encryption schemes and the established R-LWE-based homomorphic encryption schemes, a question arises concerning the potential for integrating these two approaches: Is it possible to port the R-LWE-based homomorphic encryption schemes to M-LWE, and whether this results in an improvement in performance? Should the advantages outweigh the disadvantages, this would facilitate new synergies between the current endeavour to enhance the security of a post-quantum internet and the construction of efficient and dependable homomorphic encryption algorithms. For instance, enhanced and high-performing implementations or even hardware accelerators could be reused, thereby enhancing the efficacy of homomorphic encryption while simultaneously reducing the cost of development.

The thesis is divided into five principal sections. The first section of the thesis provides an introduction to the mathematical background, wherein all necessary mathematical operations will be explained in sufficient detail. Subsequently, the LWE problems will be described in greater detail, and a basic LWE-based encryption scheme will be constructed. The scheme is capable of functioning on Plain-, Ring-, and Module-LWE. Following this, homomorphic encryption will be outlined, and the LWE-based encryption scheme will be expanded to become homomorphic for all three LWE modes. These schemes will then be evaluated based on their memory usage, processing performance, and calculation depth. Ultimately, the aforementioned question will be addressed based on these findings.

\chapter{Mathematical Background}
\label{MathBack}

In order to understand the mathematical concepts behind the encryption algorithms described here, some basic concepts are explained here. However, you should have a basic knowledge of linear algebra and polynomial calculus.

\section{Lattice}

% Based on \cite{LatticeTutorial}.

All of the algorithms discussed in this thesis are based on lattices, which is why we will briefly focus on them in more detail. In general, lattices behave like any other vector space, but they only consist of discrete vectors. This means that the vectors only contain integers and not real numbers as in a vector space.

Let $B = \{b_1, b_2, \ldots, b_m\}$ be a set of linearly independent vectors of $\mathbb{R}^n$. The lattice $L$ generated by $B$ is the set of integer linear combinations of $B$. $B$ is called the basis of the lattice $L$. That is,
$$L(B) = \{a_1b_1 + \ldots + a_mb_m | a_1, \ldots, a_m \in \mathbb{Z}  \} \subset \mathbb{R}^n$$


Using a matrix $B$, which contains the basis vectors as column vectors, we can generate $L$ equivalently.

$$L(B) = \{Bx | x \in \mathbb{Z}^m  \} \subset \mathbb{R}^n$$

As in this definition, the integer $n$ is the \textbf{dimension} of the lattice and $m$ is its \textbf{rank}. If $m = n$, then $L$ is a \textbf{full-rank} lattice, which is the usual case in this thesis. 

An example of a lattice based on a basis $B$ and all the points that can be created with it, also called the \textbf{span}, can be seen in the figure \ref{fig:latticeGrid}.

\begin{figure}[h]
  \centering
  \includegraphics[scale=0.2]{images/LatticeGrid.png}
  \caption[Span of an Lattice]{The span of an two-dimensional lattice with basis  $B = \{b_1, b_2\}$.}
  \label{fig:latticeGrid}
\end{figure}

% Nur dann benötigt wenn ich genauer auf die Funktionsweise von LWE eingehen will
% \section{Shortest Vector \& Closest Vector Problem}


\section{Integer \& Polynomial Rings with modulus}
This section is based on \cite{Algebra}.

\subsection*{Rings}
A ring is a set $R$ on which addition ($+$) and multiplication ($\cdot$) can be performed and results in a new Element, which is also part of the set R.
\begin{center}
  $ +: R+R\rightarrow R$ (Addition) and $\cdot: R \cdot R \rightarrow R$ (Multiplication)
\end{center}

These calculations need to fulfill the following conditions:
\begin{description}
  \item for addition: $R$ is an abelian group
        \begin{itemize}
          \item Associative property: $(a+b)+b = a+(b+c) | a,b,c \in R$
          \item Commutative property: $a+b = b+a | a,b \in R$
          \item Additive identity: There exists and element $0 \in R$ so that $a+0 = a | a \in R$
          \item Additive inverse: For each $a \in R$ there is an $-a \in R$ so that $a+(-a)=0$
        \end{itemize}
  \item for multiplication: R is an monoid
        \begin{itemize}
          \item Associative property: $(a\cdot b) \cdot b = a \cdot(b\cdot c) | a,b,c \in R$
          \item Multiplicative identity: There exists and element $1 \in R$ so that $a \cdot 1 = 1 \cdot a = a | a \in R$
        \end{itemize}
  \item Addition and Multiplication are distributive
        \begin{itemize}
          \item  $a\cdot (b + c) = a\cdot b + a\cdot c | a,b,c \in R$
          \item  $(a + b) \cdot c= a\cdot c + b\cdot c | a,b,c \in R$
        \end{itemize}
\end{description}

A ring is also called commutative if the multiplication is also commutative. For example, the ring over all integers $\mathbb{Z}$ is a commutative ring.

\subsection*{Modular arithmetic on Rings}

Congruence arithmetic, or modular arithmetic, is the term used to describe arithmetic with remainders when dividing integers. In everyday life, this is mainly encountered in connection with clocks. After 60 minutes, the minute hand returns to the same position as before. 

More generally, this can be described as $a \equiv b \mod n | a,b \in \mathbb{Z}, n \in \mathbb{N}$, where $n$ is the module by which $a$ and $b$ are divided until the remainder of both is less than $n$. If $a$ and $b$ are then equal, they are congruent. Or in a more mathematical expression: If there is a $k$, such as $a-b = k\cdot m$, then $a$ and $b$ are congruent.

An congruence relation with module $n$ on the set $\mathbb{Z}$, has the following properties:
% Based on \cite{Algebra} Seite 13.
$k,n \in \mathbb{N}$ and $a, a', b, b', c \in \mathbb{Z}$
\begin{enumerate}
  \item $a \equiv a \mod n$ (Reflexivity)
  \item $a \equiv b \mod n$ if $b \equiv a \mod n$ (Symmetry)
  \item If $a \equiv b \mod n$ and $b \equiv c \mod n$ then $a \equiv c \mod n$ (Transitivity)
  \item If $a \equiv a' \mod n$ and $b \equiv b' \mod n$ then $a+b \equiv a'+b' \mod n$
  \item If $a \equiv a' \mod n$ and $b \equiv b' \mod n$ then $a\cdot b \equiv a'\cdot b' \mod n$
  \item If $c$ and $n$ are coprime and $c \cdot a \equiv c \cdot b \mod n$ then $a \equiv b \mod n$
  \item If $a \equiv b \mod k\cdot n$ then $a \equiv b \mod n$
\end{enumerate}

The congruence class is the set of all numbers for an integer $a \in \mathbb{Z}$ modulus $n$ that produce the same remainder. It is defined as
$$[a]_n = \{b \in \mathbb{Z} | a \equiv b \mod n\}$$.

It follows that two numbers are congruent if both congruence classes are equal:

$$a \equiv b \mod n \Leftrightarrow [a]_n = [b]_n$$

With this we can create a set of all congruence classes modulo $n$:
$$\mathbb{Z}_n = \mathbb{Z}/n = \mathbb{Z} \mod n = \{[a]_n | a = 0, 1, \dot, n-1 \}$$.

For example, $Z_3 = \{[0]_3, [1]_3, [2]_3\}$. With addition and multiplication it is possible to create a commutative ring from $\mathbb{Z}_n$.
\begin{center}
    $[a]_n + [b]_n = [a+b]_n$ (addition) and $[a]_n \cdot [b]_n = [a\cdot b]_n$ (multiplication)
\end{center}

This allows to create finite rings $\mathbb{Z}_n$ for every natural number $n$ with $n$ elements in each ring and to perform calculations inside these ring. For example, an ring with $n=60$ can be created, which represents the minutes in every hour. If the minute hand shows now $48$ and we want to know where it is after $3$ times $13$ minutes, we can calculate it like:
$$[48]_{60} + [3]_{60}\cdot [13]_{60} = [48+3\cdot 13]_{60} = [87]_{60} = [27]_{60}$$

\subsection*{Polynomial Rings}

A polynomial with coefficients in a ring $R$ is expressed as 
$$f = a_nx^n + a_{n-1}x^{n-1}+\cdots+a_1x+a_0 | a_0, \cdots, a_n \in R$$

The variable $n$ defines the degree $\deg(f)$ of the polynomial, which is the largest exponent in a polynomial.

They can be added and multiplied like any other polynomial. Such a polynomial ring with one variable $x$ and its coefficients in $R$ is written as $R[x]$. This is a generalization of the rings we had before, because $R$ is a subset of $R[x]$ ($R \subset R[x]$), since $R$ is a polynomial with $\deg(0)$: $R = R[x] := a\cdot x^0 = a$.

Such a polynomial ring can also be defined over a finite ring, so that each coefficient is part of that finite ring. This is written as $R_n = \mathbb{Z}_n[x]$. The coefficients follow the same rules for addition and multiplication as described above. The following example takes place in the ring $R_5 = \mathbb{Z}_5[x]$ and $f, g \in R_5$ with $f=3x^2+2x+1$ and $g=2x+4$:

\begin{align*}
  f\cdot 4 &= (3x^2+2x+1) * 4 \\
      &= [12]_5x^2+[8]_5x+[4]_5 \\
      &= 2x^2+3x+4 \\
  f+g &= (3x^2+2x+1)+(2x+4) \\
      &= [3]_5x^2+[4]_5x+[5]_5\\
      &= 3x^2+4x \\
  f\cdot g &= (3x^2+2x+1)+(2x+4) \\ 
      &= [6]_5x^3+[4]_5x^2+[2]_5x+[12]_5x^2+[8]_5x+[4]_5 \\
      &= [6]_5x^3+[16]_5x^2+[10]_5x+[4]_5 \\
      &= 1x^3+1x^2+0x+4 \\
\end{align*}

As with any polynomial multiplication, the degree can increase as you multiply two polynomials, leading to increasingly larger polynomials with each multiplication. Since the modulo operation creates a finite ring, we can also create a modulo operation that creates a finite ring over a polynomial where the degree stays the same or is less than some upper bound. For this we have a ring $R$ and $f, g, q, r \in R[x], g\neq 0$, where $f$ is a polynomial, $g$ is the modulus, and $r$ is the remainder:
\begin{center}
  $f = g\cdot q + r $ and $\deg(r)<\deg(g)$.
\end{center}

After this calculation, $r$ will be the remainder of $f$ with a degree smaller than that of $g$, which will be used for further calculations. With this, we can now define polynomial rings that have a module to generate finite coefficients and a polynomial function to generate finite degree. This is written as 
$$R_n = \mathbb{Z}_n[x]/f(x)$$

For most cases in cryptography, $f(x)$ will be the function $x^d+1$, as this simplifies the calculation of the remainder. Instead of doing lengthy polynomial divisions, one can simply subtract $d$ from the exponent and invert the coefficient if the exponent is greater than or equal to $d$. This must be repeated until the largest exponent is less than $d$. For example, $f \in \mathbb{Z}_5[x]/(x^3+1)$

\begin{align*}
  f &= 3x^6+x^5+2x^3+4x^2+3 &\mod (x^3+1) \\
    &= -3x^3-x^2-2+4x^2+3 &\mod (x^3+1) \\
    &= 3-x^2-2+4x^2+3 \\
    &= 3x^2+4
\end{align*}

\subsection*{Polynomial Ring multiplication using a Matrix}

\chapter{Learning with Errors}
\label{LWE}

In this section, we will take a closer look at the Learning with Errors (LWE) algorithm (also called Plain LWE) and its different versions, namely Ring LWE (R-LWE) and Module LWE (M-LWE).

\section{The Learning with Errors Problem}

In 2005, Regev first described the LWE problem \cite{Regev2005OnLL}. He also proved its hardness, but we won't go into those details here. The basic idea is to add an error vector to a linear system of equations. This makes the normally trivially solvable system surprisingly hard to solve.

In more mathematical terms, $\mathbb{Z}_q = \mathbb{Z}/q$, $\textbf{A} \in \mathbb{Z}_q^{n \times m}$, $\textbf{s} \in \mathbb{Z}_q^m$, $\textbf{b} \in \mathbb{Z}_q^n$, with which we can form the linear system of equations, where $\textbf{A}$ and $\textbf{b}$ are given and the vector $\textbf{s}$ represents the unknowns we want to retrieve

$$\textbf{A}\cdot \textbf{s} = \textbf{b}$$

Or written as system of equations it would look like:
$$
  \setlength\arraycolsep{0pt}
  \begin{array}{ c  >{{}}c<{{}} c  >{{}}c<{{}}  c >{{}}c<{{}}  c @{{}={}} c }
    \textbf{A}_{11}\textbf{s}_1 & + & \textbf{A}_{12}\textbf{s}_2 & + & \cdots & + & \textbf{A}_{1m}\textbf{s}_m & \textbf{b}_1 \\
    \textbf{A}_{21}\textbf{s}_1 & + & \textbf{A}_{22}\textbf{s}_2 & + & \cdots & + & \textbf{A}_{2m}\textbf{s}_m & \textbf{b}_1 \\
    \vdots                      &   & \vdots                      &   & \vdots &   & \vdots                      & \vdots       \\
    \textbf{A}_{n1}\textbf{s}_1 & + & \textbf{A}_{n2}\textbf{s}_2 & + & \cdots & + & \textbf{A}_{nm}s_m          & \textbf{b}_n \\
  \end{array}
$$

This can easily be solved with the Gaussian algorithm. But if we just add an error vector $\textbf{e} \in \mathbb{Z}_q^n$ with small values, it becomes surprisingly hard. The hardness is based on variants of the Shortest Vector Problem (SVP), which describes the hardness of finding the shortest vector in the lattice. This is easily solvable in smaller dimensions, but gets harder as the dimensions are increased. The equation after adding the small error term is the following:

$$\textbf{A}\cdot \textbf{s} + \textbf{e}= \textbf{b}$$

This is the fundamental equation that underlies all LWE problems. The majority of the differences will be attributable to the ring or dimensions in use. All LWE-based encryption schemes presented in this thesis will be asymmetric encryption schemes (\cite{Eckert2018}). In consequence, each of these schemes comprises a private key, $pk$, and a secret key, $sk$, and is composed of three principal functions:

\begin{enumerate}
  \item \textbf{KeyGen}: For generating the private and secret key
  \item \textbf{Encryption}: For encrypting some message $m$ with the private key $pk$ creating a ciphertext $ct$
  \item \textbf{Decryption}: For decrypting the ciphertext $ct$ with the secret key $sk$ retrieving the original message $m$
\end{enumerate}

\info[inline]{Maybe going more into detail of the SVP or general the hardness of the LWE Problem?}

\section{LWE based encryption scheme}
\label{sec:Lwe-Encryption}

In this section, we will describe a simple LWE-based encryption scheme and how it can be converted to R-LWE and M-LWE. The following algorithm is loosely based on the Kyber \cite{CyrstalsKyber} scheme, with some simplifications.

All calculations are done in the ring $R = \mathbb{Z}_q$, where $q$ is the modulus. If values from $R$ are chosen uniformly, this is denoted by $x \leftarrow R$. Otherwise, if small values are chosen from $R$, this is written as $x \leftarrow \chi_R$. This can be done by choosing uniformly from a set of small numbers all in $R$ (e.g., ${-4,\ldots, 4}$ if $q$ is big enougth), or by choosing from an error distribution, such as the discrete Gaussian, as described in \cite{Regev2005OnLL}.

The following three algorithms describe the example schema. Algorithm \ref{alg: SampleLweKeyGen}, the key generation, describes how to generate the private key $pk$ and the secret key $sk$. It uses the LWE problem as described above. The secret key, which the owner should never share, is the vector $\textbf{s}$. The public key $pk$, which can be shared, consists of the transformation matrix $\textbf{A}$ and the transformed secret key plus the error $\textbf{b}$. The error $\textbf{e}$ is discarded after the computation of $\textbf{b}$. The values of $\textbf{e}$ and $\textbf{s}$ should be rather small, and $\textbf{A}$ is uniformly sampled from $R$.

\begin{algorithm}[htb]
  \begin{algorithmic}[1]
    \STATE $\textbf{s} \leftarrow \chi_R^n$
    \STATE $\textbf{A} \leftarrow R^{n \times n}$
    \STATE $\textbf{e} \leftarrow \chi_R^n$
    \STATE $\textbf{b} = \textbf{A}\cdot \textbf{s}+\textbf{e}$
    \RETURN $(pk:=(\textbf{A}, \textbf{b}), sk:=\textbf{s} )$
  \end{algorithmic}
  \caption{Sample LWE: KeyGen}
  \label{alg: SampleLweKeyGen}
\end{algorithm}

Algorithm \ref{alg: SampleLweEncryption}, the encryption, describes how to encrypt a message $m$ with the public key $pk$. The errors $\textbf{e}_1$ and $e_2$ are randomly sampled with small values and used to create more uncertainty around the message. The same message can therefore be decrypted with different errors and yield different values. This makes it harder for attackers to find patterns in the decryption. The idea behind $\textbf{r}$ is to select a subset of $\textbf{A}$ and $\textbf{b}$, since $~50\%$ of the values in $\textbf{r}$ will be $0$, meaning that these columns in $\textbf{A}$ and $\textbf{b}$ are irrelevant (multiplied by $0$). This helps to create more entropy between different encryptions, as a different subset of $\textbf{A}$ and $\textbf{b}$ will be used to encrypt each time.

The new values and the public key are used to calculate two values: $\textbf{u}$ and $v$. The first term, $\textbf{u}$, can be considered the cancel term for $\textbf{b}$, where the secret $\textbf{s}$ is missing. $v$ is the actual value term, which is composed of a subset of $\textbf{b}$ with some small error added and the scaled message $m'$. For the scaled message $m' = m\cdot \left\lfloor q/2\right\rfloor$, the message is multiplied with the rounded down version of half the modulus. This operation results in the values of the message $0$ and $1$ in the ring being approximately as distant from each other as possible.

\info[inline]{Als formeln nochmal kurz aufzeigen, dann pseudo code}
\begin{algorithm}[htb]
  \begin{algorithmic}[1]
    \REQUIRE $m \in \mathbb{Z}_2 = \{0, 1\}$, $pk = (\textbf{A}, \textbf{b})$
    \STATE $\textbf{r} \leftarrow \{0, 1\}^n$
    \STATE $\textbf{e}_1 \leftarrow \chi_R^n$
    \STATE $\textbf{u} = \textbf{A}^T \cdot \textbf{r} + \textbf{e}_1$
    \STATE $e_2 \leftarrow \chi_R$
    \STATE $v = \textbf{b}^T \cdot \textbf{r} + e_2 + (m\cdot \left\lfloor q/2\right\rfloor)$
    \RETURN $ct := (\textbf{u}, v)$
  \end{algorithmic}
  \caption{Sample LWE: Encryption}
  \label{alg: SampleLweEncryption}
\end{algorithm}

Algorithm \ref{alg: SampleLweDecryption}, the Decryption, describes how to decrypt an ciphertext $ct$ using the secret key $sk$.



\begin{algorithm}[htb]
  \begin{algorithmic}[1]
    \REQUIRE $ct = (\textbf{u}, v)$, $sk = \textbf{s}$
    \RETURN $\left\lfloor \frac{1}{\left\lfloor q/2\right\rfloor}\cdot \left[v-\textbf{s}^T \cdot \textbf{u}\right]_q\right\rceil _2$
  \end{algorithmic}
  \caption{Sample LWE: Decryption}
  \label{alg: SampleLweDecryption}
\end{algorithm}


To get a better understanding, consider the following simplification of the term in algorithm \ref{alg: SampleLweDecryption}.

\begin{align*}
   & \left\lfloor \frac{1}{\left\lfloor q/2\right\rfloor}\cdot \left[v-\textbf{s}^T \cdot \textbf{u}\right]_q\right\rceil _2                                                                                                                                        \\
   & = \left\lfloor \frac{1}{\left\lfloor q/2\right\rfloor}\cdot \left[\textbf{b}^T \cdot \textbf{r} + e_2 + (m\cdot \left\lfloor q/2\right\rfloor)-\textbf{s}^T \cdot (\textbf{A}^T \cdot \textbf{r} + \textbf{e}_1)\right]_q \right\rceil _2                      \\
   & = \left\lfloor \frac{1}{\left\lfloor q/2\right\rfloor}\cdot \left[(\textbf{As}+\textbf{e})^T \cdot \textbf{r} + e_2 + (m\cdot \left\lfloor q/2\right\rfloor)-\textbf{s}^T \textbf{A}^T \cdot \textbf{r} - \textbf{s}^T \textbf{e}_1\right]_q \right\rceil _2   \\
   & = \left\lfloor \frac{1}{\left\lfloor q/2\right\rfloor}\cdot \left[(\textbf{As})^T \cdot \textbf{r} + \textbf{e}^T\textbf{r}+ e_2 + (m\cdot \left\lfloor q/2\right\rfloor)-(\textbf{As})^T \cdot \textbf{r} - \textbf{s}^T \textbf{e}_1\right]_q\right\rceil _2 \\
   & = \left\lfloor \frac{1}{\left\lfloor q/2\right\rfloor}\cdot \left[\textbf{e}^T\textbf{r}+ e_2 + (m\cdot \left\lfloor q/2\right\rfloor)- \textbf{s}^T \textbf{e}_1\right]_q\right\rceil _2                                                                      \\
   & = \left\lfloor \frac{\textbf{e}^T\textbf{r}}{\left\lfloor q/2\right\rfloor}+ \frac{e_2 }{\left\lfloor q/2\right\rfloor}+ m - \frac{\textbf{s}^T \textbf{e}_1}{\left\lfloor q/2\right\rfloor}\right\rceil _2                                                    \\
   & = \left\lfloor m' \right\rceil _2                                                                                                                                                                                                                              \\
   & = m \in \{0,1\}
\end{align*}
As demonstrated by the calculation, by multiplying the cancellation term $\textbf{u}$ with the secret $\textbf{s}$, the transformation $(\textbf{As})^T \cdot \textbf{r}$ in $v$ can be canceled out. This results in the message with some error values being added to it. The erroneous message will then be rounded, which will result in the original message. This process will only be successful if all error terms together are smaller than $\frac{q}{4}$. This is due to the fact that the possible values in the message are separated by a distance of $\frac{q}{2}$ from each other. Consequently, all values between $-\frac{q}{4}\mod q=\frac{3q}{4}$ and $\frac{q}{4}$ are rounded back to $0$, while all values between $\frac{q}{4}$ and $\frac{3q}{4}$ are rounded to $1$. Consequently, provided that the message (either $0$ or $\frac{q}{2}$) is not shifted by more than $\frac{q}{4}$, it will remain within the rounding area of the original message.

\info[inline]{Maybe add an image which shows the rounding with an clock}

The current definition of this algorithm allows only $1$ bit to be encoded at the time. This could be improved with some tricks, but for simplicity reasons we wont do that here. 

To observe the functioning of this algorithm in practice, please refer to the example calculation in the appendix, which can be found in Appendix \ref{app:PlainLweCalc}.

\section{Transforming LWE to R-LWE and M-LWE}
\label{sec:TransformingLweToRlweAndMlwe}

To transform the algorithms described above into Ring-LWE, only a few changes need to be made. Most importantly, a polynomial ring will be defined as $R = \mathbb{Z}[x]_q/(x^d+1)$, with the dimension $n=1$, which means that there are only polynomials. Instead of having a vector $\textbf{r}$, it will now be a polynomial in the ring $R$, where all coefficients are either $0$ or $1$. The message to be encrypted is also transformed into a polynomial in $R$ with the message bits being the coefficients of the polynomial. Consequently, $d$ bits can now be encoded in one message. As all values are now polynomials, polynomial arithmetic is used in place of matrix arithmetic. However, as previously stated, the polynomial arithmetic in the ring can also be transformed into matrix arithmetic. All equations stay the same and the structure of the Algorithms does not change.

An illustrative example of the three-step process for RLWE can be found in \ref{app:RlweExampleCalc}.

As next step, Ring-LWE can be transformed into Module-LWE. Todo this we only need to increase the dimensions, so that $n>1$. So instead of working with polynomials as in R-LWE, matrices and vectors of these polynomials will be used.

An example can found in Appendix \ref{app:MlweExampleCalc}

So in total, the only real differences between the Plain LWE, R-LWE and M-LWE are the dimensions and the ring. The computation itself stays the same. An summarized overview of the differences can be found in table \ref{table:LweDiffs}

\begin{table}[htbp]
  \caption[LWE variables shape comparison]{Comparison between the shapes of the variables for the different LWE Types}
  \label{table:LweDiffs}
  \centering
  \begin{tabular}{|c|l|l|l|}
    \hline
                                                    & Plain LWE        & R-LWE                     & M-LWE                         \\
    \hline
    Ring $R$                                        & $\mathbb{Z}_q$   & $\mathbb{Z}[x]_q/(x^d+1)$ & $\mathbb{Z}[x]_q/(x^d+1)$     \\
    $\textbf{A}$                                    & $R^{n\times n}$  & $R$                       & $R^{n\times n}$               \\
    $\textbf{s},\textbf{b},\textbf{e},\textbf{e}_1$ & $R^{n}$          & $R$                       & $R^{n}$                       \\
    $e_2$                                           & $R$              & $R$                       & $R$                           \\
    $\textbf{r}$                                    & $\mathbb{Z}_2^n$ & $\mathbb{Z}[x]_2/(x^d+1)$ & $(\mathbb{Z}[x]_2/(x^d+1))^n$ \\
    $m$                                             & $\mathbb{Z}_2$   & $\mathbb{Z}[x]_2/(x^d+1)$ & $\mathbb{Z}[x]_2/(x^d+1)$     \\
    \hline
  \end{tabular}
\end{table}

As the variables of the different LWE types have different dimensions, also the keys that need to be stored and shared and the messages have different dimensions. A comparison can be found in Table \ref{table:LweKeys}. As can be seen there, Plain LWE and R-LWE each depend only on one variable ($n$ or $d$ respectively), while M-LWE depends on both. This results in the secret key and private key for M-LWE being quite large, as they are always matrices or even 3D-tensors. In contrast, the secret key for Plain LWE and R-LWE is the same, but the dimensions are larger for the Plain LWE public key, which consists of a matrix and a vector, in contrast to two vectors in R-LWE.

One significant deficiency of Plain LWE is that only a single bit can be encoded at a time. Consequently, the resulting encrypted messages are of the form $\ell \times (\mathbb{Z}_q^{n}\times\mathbb{Z}_q)$, where $\ell$ is the number of bits that needs to be encoded. In contrast, R-LWE and M-LWE permit the encryption of $\ell$ bits in chunks of size $d$. If the number of bits, denoted by $\ell$, is a multiple of the dimension $d$, then the transformation of each message bit into two ciphertext integers is applicable to R-LWE. In contrast, for Plain LWE and M-LWE, each message bit is transformed into $n+1$ ciphertext integers.  
The security of Plain LWE is entirely reliant on the size of $n$, whereas in M-LWE, it is a combination of $n$ and $d$. In this context, it is possible to conclude that $n$ can be smaller in M-LWE than in Plain LWE. Consequently, it can be stated that R-LWE has the smallest cipher text dimension per bit, after which comes M-LWE, and the largest one has the Plain-LWE algorithm.

\begin{table}[htbp]
  \caption[LWE dimensions]{Comparison between the dimensions for keys and messages for the different LWE Types}
  \label{table:LweKeys}
  \centering
  \begin{tabular}{|c|l|l|l|}
    \hline
         & Plain LWE                                        & R-LWE                                                            & M-LWE                                                                    \\
    \hline
    $sk$ & $\mathbb{Z}_q^{n}$                               & $R_q^{d}$                                                        & $R_q^{n\times d}$                                                        \\
    $pk$ & $\mathbb{Z}_q^{n\times n}\times\mathbb{Z}_q^{n}$ & $R_q^{d}\times R_q^{d}$                                          & $R_q^{n\times n \times d}\times R_q^{n \times d}$                        \\
    $m$  & $\ell \times \mathbb{Z}_2$                       & $\left\lceil \ell / d\right\rceil \times R_2^{d}$                & $\left\lceil \ell / d\right\rceil \times R_2^{d}$                        \\
    $ct$ & $\ell\times(\mathbb{Z}_q^{n}\times\mathbb{Z}_q)$ & $\left\lceil \ell / d\right\rceil \times(R_q^{d}\times R_q^{d})$ & $\left\lceil \ell / d\right\rceil \times(R_q^{n\times d}\times R_q^{d})$ \\    \hline
  \end{tabular}
\end{table}

So in total it can be stated, that R-LWE has the smallest overall dimensions for the keys and for the ciphertext. Plain-LWE in contrast to M-LWE has a smaller key space, but the ciphertext space per bit is bigger.
\chapter{Homomorphic Encryption}

Homomorphic encryption (HE) is a specialized cryptographic system that enables the execution of operations on encrypted data in a similar fashion to that of unencrypted data. This capability allows for the outsourcing of data storage and computation to external services while maintaining the confidentiality of the data. This creates a zero-trust environment, where there is no need to trust external providers as they are unable to decrypt the data. Furthermore, the occurrence of data breaches would be effectively eliminated, as the data is always encrypted.


As described in \cite{FheImplementations}, HE algorithms can be grouped into 3 classes:
\begin{description}
  \item [Partially Homomorphic Encryption (PHE)]\hfill \\one type of operation can be performed an unlimited amount of times
  \item [Somewhat Homomorphic Encryption (SWHE)]\hfill \\some types of operations for an limited number of times
  \item [Fully Homomorphic Encryption (FHE)]\hfill \\an unlimited type of operations for an unlimited amount of times
\end{description}

As the algorithm will be used on binary data, the operations are often reduced to addition and multiplication, as with these two, all other basic operations can be done in binary space.

The Idea was first developed by Rivest et al. \cite{Rivest1978} in 1978. They also proposed an PHE scheme, based on RSA., for multiplication only. More PHE schemes were developed over time and in 2009 C. Gentry proposed the first FHE scheme \cite{Gentry2009AFH} based on a bootstrapping technique, which refreshes the ciphertext, so that the internal errors are reduced and further calculations can be done. With that, all SWHE systems can be converted into FHE systems. However, the conversion results in a significant reduction in performance due to the computational intensity of the bootstrapping operation.

In order to implement a homomorphic encryption scheme based on LWE, it is necessary to define three additional functions in addition to the three functions defined in section \ref{sec:LweProblem}:

\begin{enumerate}
  \item \textbf{Addition}: This operation takes two ciphertexts as inputs and returns a new ciphertext by adding them together.
  \item \textbf{Relinearization KeyGen}: This operation accepts a secret and a mapping value as input and generates a Relinearization Key Pair (RLK). The RLK is necessary for creating a functional multiplication algorithm for LWE-based schemes.
  \item \textbf{Multiplication}: This operation takes two ciphertexts and a RLK as inputs and performs a multiplication operation on the ciphertexts, with the help of the RLK, returning a new ciphertext.
\end{enumerate}

The binary space allows for the derivation of all other operations based on addition and multiplication. Consequently, the majority of HE schemes concentrate on developing functional versions of these operations.


\section{The R-LWE SWHE Scheme}

As previously stated, the two operations, addition and multiplication, must be implemented within the ciphertext space in order to construct an HE scheme. The aforementioned operations will first be created for the R-LWE version of the scheme, which has already been developed. Construction of addition and multiplication will be based on a modified version of the BFV scheme (\cite{bfv}).

\subsection*{Addition}

The objective is to develop a method for adding encrypted messages in such a way that the result is identical to that obtained by adding the plaintext messages. This can be achieved by adding the ciphertext together, with the error increasing linearly. Further details on this approach can be found in the BFV scheme \cite{bfv}. 

\begin{algorithm}[htb]
  \begin{algorithmic}[1]
    \REQUIRE $ct_1 = (u_1, v_2)$, $ct_2 = (u_2, v_2)$
    \RETURN $ct_{add} = ([u_1 + u_2]_q, [v_1 + v_2]_q)$
  \end{algorithmic}
  \caption{R-LWE: Addition}
  \label{alg:RlweAddition}
\end{algorithm}

The newly created $ct_{add}$ can then be used, like any other ciphertext, to be decrypted and used for other operations. However, it should be noted that the error in it has increased, which may result in the incorrect result being produced at some point.

\subsection*{Multiplication}

The process of deriving the same result for multiplication is somewhat less straightforward. In order to simplify the following derivations and explanations, the following simplification is made, based on Algorithm \ref{alg: SampleLweDecryption}:

\begin{equation}
  ct(s)_q = v-s\cdot u
  \label{eq:baseCt}
\end{equation}

Also let $ct_1$ and $ct_2$ be two ciphertext that we want to use, with $ct_1(s) = v_1-s\cdot u_1$ and $ct_2(s) = v_2-s\cdot u_2$
The multiplication of these two values results in the equation \ref{eq:ringCiphertextMultiplication}
\begin{equation}
  \begin{split}
    [ct_1(s)\cdot ct_2(s)]_q & = [(v_1-s\cdot u_1) \cdot (v_2-s\cdot u_2)]_q                                                                                              \\
                             & = [v_1\cdot v_2 - v_1\cdot u_2 \cdot s- v_2\cdot u_1\cdot s + u_1\cdot u_2\cdot s^2]_q                                                     \\
                             & = [\underbrace{v_1\cdot v_2}_{v_m} - \underbrace{(v_1\cdot u_2 + v_2\cdot u_1)}_{u_m}\cdot s + \underbrace{u_1\cdot u_2\cdot}_{x_m} s^2]_q \\
                             & = [v_m - u_m\cdot s + x_m \cdot s^2]_q
  \end{split}
  \label{eq:ringCiphertextMultiplication}
\end{equation}

Equation \ref{eq:ringCiphertextMultiplication} results in the formation of three blocks, each dependent on a different power of $s$. In comparison to Equation \ref{eq:baseCt}, it can be observed that the current equation is similar, with the exception of the additional $x_m\cdot s^2$ factor. 
A method is required to approximate $x_m\cdot s^2$ and combine it with $v_m$ and $u_m$. This will reduce the degree of the equation from two to one, which is known as relinearization. The formalization can be observed in Equation \ref{eq:relinFormalized}, where r represents an error that should be minimized to ensure successful decryption.

\begin{equation}
  [v_m - u_m\cdot s + x_m \cdot s^2]_q = [v'_m - u'_m\cdot s + r]_q
  \label{eq:relinFormalized}
\end{equation}

In order to resolve this issue, the "modulus switching" technique from \cite{bfv} will be employed. The initial step is to define a Relinearization Key ($rlk$), which masks $s^2$. In this process, the value $s^2$ will be multiplied with a new constant, $p$. This constant is essential for reducing the error that is generated when "decrypting" the $rlk$ (see equation \ref{eq:RlkDecryption}).
The form of the masked value is based on the public key, such that when $A_{rlk}$ and $b_{rlk}$ are "decrypted" with $s$, the original value $p\cdot s^2$ is obtained. The generation of this $rlk$ is described in Algorithm \ref{alg:RingRLKGeneration}.

\begin{algorithm}[htb]
  \begin{algorithmic}[1]
    \REQUIRE $s$
    \STATE $A \leftarrow R_{p \cdot q}$
    \STATE $e \leftarrow \chi_R^{'}$
    \STATE $b = [A\cdot s+e+p\cdot s^2]_{p \cdot q}$
    \RETURN $rlk:=(A_{rlk}, b_{rlk})$
  \end{algorithmic}
  \caption{R-LWE: RLK Generation}
  \label{alg:RingRLKGeneration}
\end{algorithm}

Utilizing the $rlk$, $x_m\cdot s^2$ is now decomposited into two distinct components. One component, designated $xv_m$, is added to $v_m$, while the other, $xu_m$, is added to $u_m$.

\begin{equation}
  (xu_m, xv_m) = \left(\left[\left\lfloor \frac{x_m \cdot A_{rlk}}{p}  \right\rceil \right]_q, \left[\left\lfloor \frac{x_m \cdot b_{rlk}}{p}  \right\rceil \right]_q\right)
  \label{eq:ringXmSplitting}
\end{equation}

The process of "decryption," as illustrated in equation \ref{eq:RlkDecryption}, reveals that $x_m$, a random element within $R_q$, is multiplied with the error $e_{rlk}$. This results in a significant error value. To mitigate this, the error is divided by $p$, thereby reducing its impact. In order to permit the creation of $x_m \cdot s^2$, $s^2$ was multiplied by $p$ in the $rlk$.

\begin{equation}
  \begin{split}
    xv_m - xu_m \cdot s & = \left[\left\lfloor \frac{x_m \cdot b_{rlk}}{p}  \right\rceil \right]_q - \left[\left\lfloor \frac{x_m \cdot A_{rlk}}{p}  \right\rceil \right]_q \cdot s  \\
                        & \approx \left[\frac{x_m \cdot b_{rlk}}{p} - \frac{x_m \cdot A_{rlk}}{p} \cdot s\right]_q                                                                   \\
                        & \approx \left[\frac{x_m \cdot (A_{rlk}\cdot s+e_{rlk}+p\cdot s^2)}{p} - \frac{x_m \cdot A_{rlk} \cdot s}{p}\right]_q                                       \\
                        & \approx \left[\frac{x_m \cdot A_{rlk}\cdot s}{p}+\frac{x_m \cdot e_{rlk}}{p}+\frac{x_m \cdot p\cdot s^2}{p} - \frac{x_m \cdot A_{rlk} \cdot s}{p}\right]_q \\
                        & \approx \left[\frac{x_m \cdot e_{rlk}}{p}+ x_m \cdot s^2 \right]_q
  \end{split}
  \label{eq:RlkDecryption}
\end{equation}

The complete algorithm for multiplying can be found in Algorithm \ref{alg:RingMultiplication}. In order for the algorithm to function correctly, it is necessary to multiply the value of $\frac{t}{q}$ by each of the factors. Further details on this process can be found in \cite{bfv}.

\begin{algorithm}[htb]
  \begin{algorithmic}[1]
    \REQUIRE $rlk=(A_{rlk}, b_{rlk})$, $ct_1 = (v_1, u_1)$, $ct_2 = (v_2, u_2)$
    \STATE $v_m = \left[\left\lfloor \frac{t}{q}\cdot (v_1 \cdot v_2)\right\rceil\right] _q $
    \STATE $u_m = \left[\left\lfloor \frac{t}{q}\cdot(v_1 \cdot u_2 + v_2 \cdot u_1)\right\rceil\right] _q$
    \STATE $x_m = \left[\left\lfloor \frac{t}{q}\cdot(u_1 \cdot u_2)\right\rceil\right] _q$
    \STATE $xu_m = \left[\left\lfloor \frac{x_m \cdot A_{rlk}}{p}  \right\rceil \right]_q$
    \STATE $xv_m = \left[\left\lfloor \frac{x_m \cdot b_{rlk}}{p}  \right\rceil \right]_q$
    \RETURN $ct_m:=(\left[u_m + xu_m\right]_q , \left[v_m + xv_m\right]_q )$
  \end{algorithmic}
  \caption{R-LWE: Multiplication}
  \label{alg:RingMultiplication}
\end{algorithm}

\section{Generalizing from R-LWE to M-LWE}

As previously stated in Section \ref{sec:TransformingLweToRlweAndMlwe}, M-LWE is a generalization of R-LWE, where matrices and vectors of polynomials are used. Consequently, the dimension $n$ will be set to a value greater than $1$. In consequence, the dimensions of nearly all variables do change (see Table \ref{table:LweKeys}). Most importantly, $\textbf{u}$ and $\textbf{s}$ will now be vectors of length $n$, instead of single polynomials. To make the difference now more visible, vectors and matrices will now be written with bold.

\subsection*{Addition}

The addition operation has no impact on the shape of u and v, and thus the same algorithm can be used as before. The only difference is that the input ciphertexts and the newly created ciphertext are of shape $R_q^{n}\times R_q$.

\subsection*{Multiplication}

In contrast, the concept of multiplication is more complex due to the necessity of dealing with changing dimensions. When equation \ref{eq:baseCt} is applied, the term $\textbf{s}\cdot \textbf{u}$ is now a dot product between two vectors, rather than a simple polynomial multiplication. The objective is, as before, to generate new $v'_m$ and $\textbf{u}'_m$ terms, which can be used for further operations or decryption. In contrast to the previous iteration, $\textbf{u}'_m$ must now be represented as a vector rather than a polynomial.

When two ciphertexts are multiplied, the resulting equation is given by equation \ref{eq:moduleCiphertextMultiplication}. Given the complexity of the aforementioned process, a brief overview of the requisite steps will be provided in the subsequent paragraphs.

\begin{equation}
  \begin{split}
    [ct_1(s)\cdot ct_2(s)]_q & = [(v_1-\textbf{s}\cdot \textbf{u}_1) \cdot (v_2-\textbf{s}\cdot \textbf{u}_2)]_q                                                                                                                                                              \\
                             & = [(v_1-\sum_{i=0}^{n-1}s_iu_{1i}) \cdot (v_2-\sum_{i=0}^{n-1}s_iu_{2i})]_q                                                                                                                                                                    \\
                             & = [v_1\cdot v_2 - v_1\cdot \sum_{i=0}^{n-1}s_iu_{2i}- v_2\cdot \sum_{i=0}^{n-1}s_iu_{1i} + \sum_{i=0}^{n-1}\sum_{j=0}^{n-1}u_{1i}u_{2j}s_is_j]_q                                                                                               \\    
                             & = [v_1\cdot v_2 - v_1\cdot \textbf{u}_2\cdot \textbf{s} - v_2\cdot \textbf{u}_1\cdot \textbf{s} + \mathrm{sum}((\textbf{u}_{1}\otimes\textbf{u}_{2})\odot(\textbf{s}\otimes\textbf{s}))]_q                                                              \\
                             & = [\underbrace{v_1\cdot v_2}_{v_m} - \underbrace{(v_1\cdot \textbf{u}_2 + v_2\cdot \textbf{u}_1)}_{\textbf{u}_m}\cdot \textbf{s} + \mathrm{sum}((\underbrace{\textbf{u}_{1}\otimes\textbf{u}_{2}}_{\textbf{X}_m})\odot(\textbf{s}\otimes\textbf{s}))]_q \\
                             & = [v_m - \textbf{u}_m\cdot \textbf{s} + \mathrm{sum}(\textbf{X}_m\odot(\textbf{s}\otimes\textbf{s}))]_q
  \end{split}
  \label{eq:moduleCiphertextMultiplication}
\end{equation}

The first technique employed was to convert the vector dot product into its sum form. As per the definition of the dot product between two vectors, it can be rewritten as a sum: $\textbf{s}\cdot \textbf{u} = \sum_{i=0}^{n-1}s_iu_i$. This step is derived from the calculations presented in \cite{ModHE}.

The subsequent step is to transform the resulting sums once more. With the single sums, this is a relatively straightforward process, as they can simply be reformulated as dot products with an additional scalar (polynomial) multiplication. As with scalar-vector multiplication, the scalar is multiplied with each value in the vector. With this, some further transformations can be made to create a new $\textbf{u}_m$, which is a vector. 

For the double sum, it is a bit more difficult process to extract a new $x_m$. The main Idea here is, that because of the double sum, essentially an $n\times n$ matrix with all combinations of $i$ and $j$ is generated and all values are then added up. For example with $n=3$ the following matrix will be created:

$$
  \sum_{i=0}^{n-1}\sum_{j=0}^{n-1}u_{1i}u_{2j}s_is_j = \mathrm{sum}\left(\begin{bmatrix}
      u_{11}u_{21}s_{1}s_{1} & u_{12}u_{21}s_{2}s_{1} & u_{13}u_{21}s_{3}s_{1} \\
      u_{11}u_{22}s_{1}s_{2} & u_{12}u_{22}s_{2}s_{2} & u_{13}u_{22}s_{3}s_{2} \\
      u_{11}u_{23}s_{1}s_{3} & u_{12}u_{23}s_{2}s_{3} & u_{13}u_{23}s_{3}s_{3} \\
    \end{bmatrix}\right)
$$

The $\mathrm{sum}$ is simply a summation of all values, which is sometimes referred to as the "grand sum." This is essentially a double dot product with a vector of length $n$, where all values are $1$, which is denoted by the symbol $1$-vector ($\textbf{1}$): $\mathrm{sum}(\textbf{X}):= \textbf{1}\cdot \textbf{X} \cdot \textbf{1} = \sum_{i=0}^{n-1}\sum_{j=0}^{n-1} \textbf{X}_{ij}$

The next step involves splitting the $n \times n$ matrix into two matrices, one for the $u$ values and one for the $s$ values. Each term in the matrix is a product of four values, which can be split apart using the Associative Law. The $\textbf{u}$ and $\textbf{s}$ values are then multiplied separately, and the two matrices are multiplied together again using element-wise multiplication, also known as the Hadamard product, denoted by the $\odot$ symbol. Finally, the individual matrices can be decomposed into vector operations. This can be achieved through the use of the outer product, also referred to as the tensor product, which is represented by the symbol $\otimes$.

\begin{align*}
  \begin{bmatrix}
    u_{11}u_{21}s_{1}s_{1} & u_{12}u_{21}s_{2}s_{1} & u_{13}u_{21}s_{3}s_{1} \\
    u_{11}u_{22}s_{1}s_{2} & u_{12}u_{22}s_{2}s_{2} & u_{13}u_{22}s_{3}s_{2} \\
    u_{11}u_{23}s_{1}s_{3} & u_{12}u_{23}s_{2}s_{3} & u_{13}u_{23}s_{3}s_{3} \\
  \end{bmatrix}
   & = \begin{bmatrix}
         u_{11}u_{21} & u_{12}u_{21} & u_{13}u_{21} \\
         u_{11}u_{22} & u_{12}u_{22} & u_{13}u_{22} \\
         u_{11}u_{23} & u_{12}u_{23} & u_{13}u_{23} \\
       \end{bmatrix} \odot \begin{bmatrix}
                             s_{1}s_{1} & s_{2}s_{1} & s_{3}s_{1} \\
                             s_{1}s_{2} & s_{2}s_{2} & s_{3}s_{2} \\
                             s_{1}s_{3} & s_{2}s_{3} & s_{3}s_{3} \\
                           \end{bmatrix}                \\
   & = (\textbf{u}_1 \otimes \textbf{u}_2) \odot (\textbf{s} \otimes \textbf{s} )
\end{align*}

Having achieved a separation between $u$ and $s$, the next step is to find a method for approximating the double sum in order to add it to $v_m$ and $\textbf{u}_m$. This needs to be done in a manner analogous to equation \ref{eq:relinFormalized}. Previously, a masking of $s^2$ was employed in order to eliminate this term. In the current context, an analogous issue arises with $s \otimes s$ and a second problem emerges, namely the shape of $v_m$ as a polynomial and $\textbf{u}_m$ as a vector.

As a first step, it is necessary to revisit the original $rlk$ generation process, as outlined in Algorithm \ref{alg:RingRLKGeneration}. In order to transform it into M-LWE, the same dimensions as those employed in the standard M-LWE key generation process are used: specifically, $\textbf{A} \in R^{n \times n}_{p \cdot q}$, $\textbf{e} \in \chi^{'n}_{R}$ and $\textbf{s} \in R^n_q$. When calculating the first part of $\textbf{b}$ we get $\textbf{A}\cdot \textbf{s} + \textbf{e}$, which is an vector in $R^n$. As the second part needs to be added to this vector, the masked part (formerly $s^2$) needs to be a vector of the same dimension. As $s \otimes s$ is a matrix of dimension $n \times n$, it cannot be used directly. However, it can be split into $n$ $n$-dimensional vectors, which can then be used instead. A similar approach must be taken with the $u_1 \otimes u_2$ matrix. The matching vectors of both matrices must be multiplied elementwise. This is a feasible approach, as all values within the matrix will be summed collectively at the final step, which can be done in any order (commutative law).

\begin{align*}
   & \mathrm{sum}\left((\textbf{u}_1 \otimes \textbf{u}_2) \odot (\textbf{s} \otimes \textbf{s} )\right)                                                             \\
   & = \mathrm{sum}\left(\begin{bmatrix}
                    u_{11}u_{21} & u_{12}u_{21} & u_{13}u_{21} \\
                    u_{11}u_{22} & u_{12}u_{22} & u_{13}u_{22} \\
                    u_{11}u_{23} & u_{12}u_{23} & u_{13}u_{23} \\
                  \end{bmatrix} \odot \begin{bmatrix}
                                        s_{1}s_{1} & s_{2}s_{1} & s_{3}s_{1} \\
                                        s_{1}s_{2} & s_{2}s_{2} & s_{3}s_{2} \\
                                        s_{1}s_{3} & s_{2}s_{3} & s_{3}s_{3} \\
                                      \end{bmatrix}\right)                                                                                 \\
   & = \mathrm{sum}\left(\begin{bmatrix}
                    u_{11}u_{21} \\
                    u_{11}u_{22} \\
                    u_{11}u_{23} \\
                  \end{bmatrix} \odot \begin{bmatrix}
                                        s_{1}s_{1} \\
                                        s_{1}s_{2} \\
                                        s_{1}s_{3} \\
                                      \end{bmatrix}
  + \begin{bmatrix}
        u_{12}u_{21} \\
        u_{12}u_{22} \\
        u_{12}u_{23} \\
      \end{bmatrix} \odot \begin{bmatrix}
                            s_{2}s_{1} \\
                            s_{2}s_{2} \\
                            s_{2}s_{3} \\
                          \end{bmatrix}
  + \begin{bmatrix}
        u_{13}u_{21} \\
        u_{13}u_{22} \\
        u_{13}u_{23} \\
      \end{bmatrix} \odot \begin{bmatrix}
                            s_{3}s_{1} \\
                            s_{3}s_{2} \\
                            s_{3}s_{3} \\
                          \end{bmatrix}\right)                                                                                                                \\
   & = \mathrm{sum}\left(\sum_{i=0}^{n-1}(\underbrace{u_{1i}\cdot \textbf{u}_2}_{\textbf{x}_{mi}}) \odot (\underbrace{s_i \cdot \textbf{s}}_{\textbf{s}'_i}) \right) \\
\end{align*}

This allows us to employ $s_i \cdot \textbf{s}$ in the $rlk$ generation algorithm \ref{alg:RingRLKGeneration}. However, this necessitates the computation of $n$ $rlk$ values, as is required for each $s_i$. The complete module $rlk$ generation process is illustrated in Algorithm \ref{alg:ModuleRLKGeneration}, where $\textbf{s}'_i$ represents the individual $s_i\cdot \textbf{s}$ values. 

\begin{algorithm}[htb]
  \begin{algorithmic}[1]
    \REQUIRE $\textbf{s}$, $\textbf{s}'$
    \STATE $\textbf{A} \leftarrow R_{p \cdot q}^{n \times n}$
    \STATE $\textbf{e} \leftarrow \chi_R^{'n}$
    \STATE $\textbf{b} = [\textbf{A}\cdot \textbf{s}+\textbf{e}+p\cdot \textbf{s}']_{p \cdot q}$
    \RETURN $rlk:=(\textbf{A}_{s'}, \textbf{b}_{s'})$
  \end{algorithmic}
  \caption{M-LWE: RLK Generation}
  \label{alg:ModuleRLKGeneration}
\end{algorithm}

As before, the $rlk$ can be used to create two values, $xu_m$ and $xv_m$. These values will be added to $\textbf{u}_m \in R^n_q$ and $v_m \in R_q$, respectively. Therefore, it is necessary for these values to have the same shape. However, as the encryption process used near identical formulas, the correct shapes will be produced automatically. Thus, equation \ref{eq:ringXmSplitting} can be translated into M-LWE, as shown in equation \ref{eq:moduleXmSplitting}.

\begin{equation}
  (\textbf{xu}_m, xv_m) = \left(\sum_{i=0}^{n-1}\left[\left\lfloor\frac{\textbf{A}_{\textbf{s}'_i} \cdot \textbf{x}_{mi}}{p}  \right\rceil \right]_q, \sum_{i=0}^{n-1}\left[\left\lfloor \frac{b_{\textbf{s}'_i} \cdot \textbf{x}_{mi}}{p}  \right\rceil \right]_q\right)
  \label{eq:moduleXmSplitting}
\end{equation}

All this can be combined now into a single M-LWE multiplication algorithm, as seen in \ref{alg:moduleMultiplication}.

\begin{algorithm}[htb]
  \begin{algorithmic}[1]
    \REQUIRE $rlk=((\textbf{A}_{\textbf{s}'_0}, \textbf{b}_{\textbf{s}'_0}), \ldots ,(\textbf{A}_{\textbf{s}'_{n-1}}, \textbf{b}_{\textbf{s}'_{n-1}}))$, $ct_1 = (\textbf{u}_1, v_1)$, $ct_2 = (\textbf{u}_2, v_2)$
    \STATE $v_m = \left[\left\lfloor \frac{t}{q}\cdot (v_1 \cdot v_2)\right\rceil\right] _q $
    \STATE $\textbf{u}_m = \left[\left\lfloor \frac{t}{q}\cdot(v_1 \cdot \textbf{u}_2 + v_2 \cdot \textbf{u}_1)\right\rceil\right] _q$
    \STATE $\textbf{x}_m = \left(\left[\left\lfloor \frac{t}{q}\cdot(u_{10} \cdot \textbf{u}_2)\right\rceil\right]_q,\ldots, \left[\left\lfloor \frac{t}{q}\cdot(u_{1n-1} \cdot \textbf{u}_2)\right\rceil\right]_q\right) $
    \STATE $\textbf{xu}_m = \sum_{i=0}^{n-1}\left[\left\lfloor\frac{\textbf{A}_{\textbf{s}'_i} \cdot \textbf{x}_{mi}}{p}  \right\rceil \right]_q$
    \STATE $xv_m = \sum_{i=0}^{n-1}\left[\left\lfloor \frac{b_{\textbf{s}'_i} \cdot \textbf{x}_{mi}}{p}  \right\rceil \right]_q$
    \RETURN $ct_m:=(\left[\textbf{u}_m + \textbf{xu}_m\right]_q , \left[v_m + xv_m\right]_q )$
  \end{algorithmic}
  \caption{M-LWE: Multiplication}
  \label{alg:moduleMultiplication}
\end{algorithm}

\subsection*{Generate R-LWE from M-LWE}

One simple method for evaluating the efficacy of the generalization is to generate the R-LWE scheme with the M-LWE scheme and $n=1$. The initial step is to calculate the $rlk$. With a dimension of 1, there is only a single $rlk$, which is calculated with $s'_0 = s_0 \cdot \textbf{s} = s_0 \cdot s_0 = s^2$. It can be seen that the M-LWE $rlk(s, s^2)$ (see algorithm \ref{alg:ModuleRLKGeneration}) is identical to the R-LWE $rlk(s)$ (see algorithm \ref{alg:RingRLKGeneration}), as both $A^{1 \times 1}$ and $e^1$ are both single polynomials. As with $s$, the vector $\textbf{u}$ is a single polynomial. Consequently, the calculation of $u_m$ is identical between M-LWE (algorithm \ref{alg:moduleMultiplication}) and R-LWE (algorithm \ref{alg:RingMultiplication}). Furthermore, only a single $x_m$ value is required in M-LWE (as $n=1$), which is then used to calculate both $xu$ and $xv$, which are also single polynomials. Therefore, the entire calculation is identical to the R-LWE multiplication, which is a positive indication.


\section{Criteria for Comparing LWE based Homomorph encryption schemes}

This section outlines the  criteria for comparing the three homomorphic LWE schemes. The primary distinctions between the three LWE schemes are related to two dimension variables, matrix dimension $n$ and polynomial rank $d$. As previously outlined in section \ref{sec:TransformingLweToRlweAndMlwe}, when $n>1$ and $d=1$, an Plain-LWE scheme is obtained. Conversely, when $n=1$ and $d>1$, the resulting scheme is R-LWE, and when $n>1$ and $d>1$, the scheme is M-LWE.

The size of the output values of the different steps can be compared based on these two dimension variables. The outputs in question are the secret key $sk$, the public key $pk$, the relinearization key $rlk$ and the ciphertext $ct$.

% Variable sizes based on n and d
% SK Size
% PK Size (including rlk)
% CT size

To facilitate a comparative analysis of the models' performance, an example implementation in Python will be constructed and the resulting data will be evaluated. As the same algorithm, with varying dimensions ($n$ and $d$), can be utilized to describe all three schemes, the relative performance between them can be effectively assessed. However, given that Python is a relatively slow-performing language and the algorithm is not optimized, the absolute values may not be entirely meaningful. Nevertheless, the relative performance between the schemes should remain highly comparable.

Unless otherwise indicated, all performance tests will be conducted on all algorithms, specifically: \textit{KeyGen (with )}, \textit{Encryption}, \textit{Decryption}, \textit{Addition} and \textit{Multiplication}. The first stage of the testing process will be to assess the performance of each model in isolation, with the results expressed in terms of the dimension variables $n$ and $d$, as well as the modulus $q$ and $p$ (where applicable). Subsequently, the objective will be to evaluate the relative performance of the various algorithms by comparing them against each other, with the same modulus values employed in each comparison. 

% Practical Test - Performance Comparison based on n,d,q,p
% KeyGen (with RLK)
% Encryption
% Addition
% Multiplication
% Decryption

Furthermore, the addition and multiplicative depth should be identified based on the variables. This allows for the assessment of the number of consecutive calculations that can be performed while maintaining the ability to successfully decrypt the ciphertext. To enhance the precision of the predictions, it is essential to examine the evolution of the errors associated with addition and multiplication. This enables a more accurate understanding of the impact of the variables and depth on the outcomes.


% Practical Test - Error Development - Max Rounds based on n,d,q,p 
% Addition
% Multiplication

% error development (linear/quadratic) test?
% Addition
% Multiplication

A comparative analysis of the security features of the various schemes will not be undertaken, as it would require a significant investment of time and resources that would exceed the scope of this thesis.
\chapter{Practical comparison between the Models}
Tasks:
\begin{itemize}
    \item Creating an Test Setup based on the compare criteria
    \item Run the Tests
    \item Results and Evaluation
\end{itemize}

% ===============
%  Verzeichnisse
% ===============

% Verzeichnisse mit einzeiligem Zeilenabstand
\singlespacing

% Literaturverzeichnis
\listofreferences

% Abbildungsverzeichnis einfügen
\listoffigures

% Tabellenverzeichnis einfügen
\listoftables

% Algorithmenverzeichnis einfügen
\listofalgorithms

% % Quelltextverzeichnis einfügen
\listoflistings

% =========
%  Appendix
% =========

\begin{appendices}

  \chapter{Example Calculations}

\section{Example Multidimensional Ring Calculation}
\label{app:ExampleMultiRingCalc}

Consider the ring $R = \mathbb{Z}_5[x]/(x^3+1)$ and 
$$
f =\begin{bmatrix}
    1+2x+3x^2 & 2+3x+4x^2 \\
    3+4x+x^2  & 1+3x+4x^2 \\
  \end{bmatrix} \in R^{2\times 2}
$$

$$
g = \begin{bmatrix}
    1+x+x^2   \\
    2+2x+2x^2 \\
  \end{bmatrix} \in R^2
$$

\begin{align*}
  f \cdot g & = {
  \begin{bmatrix}
    1+2x+3x^2 & 2+3x+4x^2 \\
    3+4x+x^2  & 1+3x+4x^2 \\
  \end{bmatrix}
  \cdot
  \begin{bmatrix}
    1+x+x^2   \\
    2+2x+2x^2 \\
  \end{bmatrix}
  }               \\
            & = {
  \begin{bmatrix}
    \begin{bmatrix}
      1 & -3 & -2 \\
      2 & 1  & -3 \\
      3 & 2  & 1  \\
    \end{bmatrix} & 
    \begin{bmatrix}
      2 & -4 & -3 \\
      3 & 2  & -4 \\
      4 & 3  & 2  \\
    \end{bmatrix}   \\
    \begin{bmatrix}
      3 & -1 & -4 \\
      4 & 3  & -1 \\
      1 & 4  & 3  \\
    \end{bmatrix} & 
    \begin{bmatrix}
      1 & -4 & -3 \\
      3 & 1  & -4 \\
      4 & 3  & 1  \\
    \end{bmatrix}   \\
  \end{bmatrix}
  \cdot
  \begin{bmatrix}
    \begin{bmatrix}
      1 \\
      1 \\
      1 \\
    \end{bmatrix} \\
    \begin{bmatrix}
      2 \\
      2 \\
      2 \\
    \end{bmatrix} \\
  \end{bmatrix}
  }               \\
            & = {
  \begin{bmatrix}
    1 & -3 & -2 & 2 & -4 & -3 \\
    2 & 1  & -3 & 3 & 2  & -4 \\
    3 & 2  & 1  & 4 & 3  & 2  \\
    3 & -1 & -4 & 1 & -4 & -3 \\
    4 & 3  & -1 & 3 & 1  & -4 \\
    1 & 4  & 3  & 4 & 3  & 1 
  \end{bmatrix}
  \cdot
  \begin{bmatrix}
    1 \\
    1 \\
    1 \\
    2 \\
    2 \\
    2 
  \end{bmatrix}
  }               \\
            & = {
  \begin{bmatrix}
    -14 \\
    2   \\
    24  \\
    -14 \\
    6   \\
    24
  \end{bmatrix}
  \mod 5
  }
  = {
  \begin{bmatrix}
    1 \\
    2 \\
    4 \\
    1 \\
    1 \\
    4
  \end{bmatrix}
  }
  = {
  \begin{bmatrix}
    \begin{bmatrix}
      1 \\
      2 \\
      4
    \end{bmatrix} \\
    \begin{bmatrix}
      1 \\
      1 \\
      4
    \end{bmatrix} \\
  \end{bmatrix}
  }               \\
            & = {
  \begin{bmatrix}
    1+2x+4x^2 \\
    1+1x+4x^2
  \end{bmatrix}
  }
\end{align*}


\section{Plain LWE}
\label{app:PlainLweCalc}
The following calculations should show the working of the Plain LWE encryption for the algorithms \ref{alg: SampleLweKeyGen} to \ref{alg: SampleLweDecryption}. The ring used for this calculations is defined as $R=\mathbb{Z}_{100}$ and $n=2$ for the dimensions. Starting first with the key generation:


\begin{align*}
  s  & = \begin{bmatrix}1 \\ 2 \end{bmatrix}
  A  = \begin{bmatrix}56 & 77 \\ 29 & 59 \end{bmatrix}
  e  = \begin{bmatrix}99 \\ 1 \end{bmatrix}  \\
  b  & = As+e                                \\
     & = \begin{bmatrix}
           56 & 77 \\
           29 & 59
         \end{bmatrix}
  \cdot
  \begin{bmatrix}
    1 \\
    2
  \end{bmatrix}
  +
  \begin{bmatrix}
    99 \\ 
    1 
  \end{bmatrix}
  \\
     & = 1
  \cdot
  \begin{bmatrix}
    56 \\
    29
  \end{bmatrix}
  + 2 
  \cdot
  \begin{bmatrix}
    77 \\ 
    59 
  \end{bmatrix}
  + 
  \begin{bmatrix}
    99 \\ 
    1 
  \end{bmatrix}                             \\
     & = \begin{bmatrix}
           309 \\
           148\end{bmatrix}_{100}              \\
     & = \begin{bmatrix}
           9 \\ 
           48 
         \end{bmatrix}                      \\
  sk & = s                                   \\
  pk & = (A, b) = \left (
  \begin{bmatrix}
      56 & 77  \\
      29 & 59 
    \end{bmatrix},
  \begin{bmatrix}
      9 \\
      48 
    \end{bmatrix} \right )                     \\
\end{align*}

With the secret and public key generated, the next step is to encrypt the message $m=1$ with the public key $pk$

\begin{align*}
  r       & = \begin{bmatrix}0 \\ 1 \end{bmatrix}
  e_1 = \begin{bmatrix}2 \\ 0 \end{bmatrix}
  e_2 = 99                                                          \\
  \\
  u       & = A^T \cdot r + e_1                                     \\
          & = \begin{bmatrix}
                56 & 77 \\
                29 & 59
              \end{bmatrix}^T
  \cdot
  \begin{bmatrix}
    0 \\
    1
  \end{bmatrix}
  +
  \begin{bmatrix}
    2 \\
    0
  \end{bmatrix}                                                    \\
          & = \begin{bmatrix}
                56 & 29 
                \\ 77 & 59 
              \end{bmatrix}
  \cdot 
  \begin{bmatrix}
    0 \\
    1 
  \end{bmatrix}
  +
  \begin{bmatrix}
    2 \\
    0
  \end{bmatrix}                                                    \\
          & = 0\cdot
  \begin{bmatrix}
    56 \\
    77
  \end{bmatrix}
  + 1 \cdot 
  \begin{bmatrix}
    29 \\
    59
  \end{bmatrix}
  +
  \begin{bmatrix}
    2 \\
    0
  \end{bmatrix}                                                    \\
          & = \begin{bmatrix}
                31 \\ 
                61 
              \end{bmatrix}_{100}
  = 
  \begin{bmatrix}
    31 \\
    61
  \end{bmatrix}                                                    \\
  \\
  v       & = b^T \cdot r + e_2 + (m*\left\lfloor q/2\right\rfloor) \\
          & =\begin{bmatrix}
               9 \\
               48
             \end{bmatrix}^T
  \cdot
  \begin{bmatrix}
    0 \\
    1
  \end{bmatrix}
  + 99 + 1 \cdot \left\lfloor 100/2\right\rfloor                    \\
          & =\begin{bmatrix}
               9 & 48
             \end{bmatrix}
  \cdot
  \begin{bmatrix}
    0 \\ 
    1 
  \end{bmatrix}
  + 99 + 50                                                         \\
          & = 9 \cdot 0 +48 \cdot 1 + 99 + 50                       \\
          & = 197_{100}                                             \\
          & = 97                                                    \\
  m_{enc} & = (u, v) = \left (
  \begin{bmatrix}
      31 \\
      61
    \end{bmatrix}, 97   \right )                                      \\
\end{align*}

Now the encrypted message $M_{enc}$ can be decrypted again, using the secret key $sk$:
\begin{align*}
  m & = \left\lfloor \frac{1}{\left\lfloor q/2\right\rfloor} *(v-s^T \cdot u)\right\rceil _2 \\
    & = \left\lfloor \frac{1}{\left\lfloor 100/2\right\rfloor} * \left (97-
  \begin{bmatrix}
      1 \\
      2
    \end{bmatrix}^T
  \cdot
  \begin{bmatrix}
      31 \\
      61
    \end{bmatrix} \right )\right\rceil _2                                                      \\
    & = \left\lfloor \frac{1}{50} * \left (97-
  \begin{bmatrix}
      1 & 2 
    \end{bmatrix}
  \cdot 
  \begin{bmatrix}
      31 \\ 
      61
    \end{bmatrix}\right )\right\rceil _2                                                       \\
    & = \left\lfloor \frac{1}{50} * (97-(31 \cdot 1 + 61 \cdot 2))\right\rceil _2            \\
    & = \left\lfloor \frac{1}{50} * (-56)_{100}\right\rceil _2                               \\
    & = \left\lfloor \frac{1}{50} * 44\right\rceil _2                                        \\
    & = \left\lfloor \frac{44}{50}\right\rceil _2  = \left\lfloor 0.88\right\rceil _2        \\
    & = 1                                                                                    \\
\end{align*}

\end{appendices}


% Abkürzungsverzeichnis
% \listofacronyms

% % Symbolverzeichnis
% \listofsymbols

% falls ein anderer Glossar-Stil genutzt wird und die zweite Spalte zu schmal ist:
% \setlength{\glsdescwidth}{0.8\linewidth}

% Glossar einfügen
% \printglossary

% Index einfügen
\printindex

% wieder auf 1½-fachen Zeilenabstand umschalten
\normalspacing

\end{document}
