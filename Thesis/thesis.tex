\documentclass{fiwthesis}

% ========
%  Pakete
% ========

\usepackage{textgreek}           % griechische Buchstaben außerhalb des Math-Mode
\usepackage{amsmath}             % zentrierte Formeln
\usepackage{amssymb}             % erweiterter Formelsatz mathem. Symbole

\usepackage{boldline}            % breitere Linien in Tabellen
\usepackage{booktabs}            % typographisch richtige Tabellen setzen
\usepackage{tabularx}            % Erweiterte Tabellendarstellung
\usepackage{multirow}            % Spalte über mehrere Zeilen oder Spalten ausdehnen
\usepackage{xltabular}           % Zeilenumbrüche in tabularx erlauben

\usepackage{graphicx}            % ermöglicht das Einbinden von Grafiken
\usepackage{subcaption}          % mehrere Bilder in einem Bild
\usepackage{pgfplots}            % Grafiken erzeugen
\usepackage{smartdiagram}        % schnelle und einfache Grafiken

\newtheorem{definition}{Definition}

% Commands for TODOs
% \usepackage[pdftex,dvipsnames]{xcolor}  % Coloured text etc.

\usepackage{xargs}
\setlength{\marginparwidth }{2cm}
\usepackage[colorinlistoftodos,prependcaption,textsize=tiny]{todonotes}
\newcommandx{\unsure}[2][1=]{\todo[linecolor=red,backgroundcolor=red!25,bordercolor=red,#1]{#2}}
\newcommandx{\change}[2][1=]{\todo[linecolor=blue,backgroundcolor=blue!25,bordercolor=blue,#1]{#2}}
\newcommandx{\info}[2][1=]{\todo[linecolor=OliveGreen,backgroundcolor=OliveGreen!25,bordercolor=OliveGreen,#1]{#2}}
\newcommandx{\improvement}[2][1=]{\todo[linecolor=Plum,backgroundcolor=Plum!25,bordercolor=Plum,#1]{#2}}
\newcommandx{\thiswillnotshow}[2][1=]{\todo[disable,#1]{#2}}

% ===========
%  Metadaten
% ===========

\thesis{Master-Thesis}
\title{Homomorphic Post-Quantum Cryptography - Evaluation of Module Learning with Error in Homomorphic Cryptography}
\author{Pascal Stehling}
\matrnr{455051}
\bdate{18.12.1997}
\bcity{Wismar}
\supervisor{Prof.~Dr.-Ing.~habil.~Andreas Ahrens}
% \secsupervisor{ZWEITBETREUER}
\keywords{Logik, Mathematik}

% Metadaten in die PDF-Datei schreiben
\makepdfmetadata

% ===============
%  Präambel
% ===============

% PGF Kompatibilitätseinstellung
\pgfplotsset{width=0.95\textwidth,compat=newest}

% % Bibliographie einbinden
\bibliography{quellen}

% Glossar einbinden
% this cant be removed, for some reason the build breaks
\newdualentry{dos}% label
{DoS}% short form
{Denial of Service}% long form
{Ein Denial of Service (im Deutschen: Dienstverweigerung) ist ein Angriffe auf Computer- oder Netzwerksysteme, wobei das Zielsystem durch Überlastung oder durch andere Mittel außer Betrieb gesetzt wird}% description


% Abkürzungen einbinden
%\gls{}         normal zu nutzen (erstes Mal: 'lange Form (kurze Form)'), danach nur 'kurze Form'
%\glspl{}       wie \gls{} nur als Plural
%\acrfull{qrc}  gibt volle Form ('lange Form (kurze Form)') egal wo
%\acrlong{qrc}  gibt lange Form ('lange Form') egal wo
%
%\newacronym{tag}{short}{long}
\newacronym{lamp}{LAMP}{Linux, Apache, MySQL, PHP}
\newacronym{qrc}{QR-Code}{Quick Response Code}


% Symbole einbinden
\newglossaryentry{symb:phi}{
  name=$\phi$,
  description={Ein beliebiger Winkel},
  sort=symbolphi, type=symbolslist
}

\newglossaryentry{symb:e}{
  name=$e$,
  description={Die Eulersche Zahl},
  sort=symbole, type=symbolslist
}


% % Glossar- und Abkürzungsverzeichniserstellung
\makeglossaries{}

% Index erzeugen
\makeindex[
  intoc=true,
  title=Index,
  columns=2]{}
\indexsetup{headers={\indexname}{\indexname}}

% ===============
%  Eigene Makros
% ===============

\newcommand*{\code}[1]{\texttt{#1}}

\begin{document}

% Titelseite
\maketitle

\maketask{
In 2009, Craig Gentry published the first fully homomorphic encryption (FHE) algorithm in his PhD dissertation. With that it was possible for the first time to do any number of calculations on encrypted messages without having to decrypt them. Since the discovery of this first FHE algorithm, other such algorithms have been developed, but instead of using Ideal Lattices as the mathematical foundation, newer ones tend to use the Learning with Errors (LWE) and Ring-LWE (RLWE) problems.

In July 2022, the American National Institute of Standards and Technology (NIST) published the first group of winners of its competition for quantum-safe algorithms. The winner for general asymmetric encryption was the CRYSTALS-Kyber algorithm, which is based on Module-LWE (MLWE). This is an extension of the RLWE method in which polynomials in higher dimensions (vectors and matrices) are used. Even though it needs more computational power, in contrast to RLWE, it can be offset through parallelization of the calculations and higher security.

With these two distinct developments, the questions can be asked, if it possible to combine them and see if it’s possible to transfer existing FHE cryptosystems, which are based on RLWE to the MLWE method. This will first be researched theoretically and then verified with practical tests. 
These tests will be used to examine the advantages and disadvantages of the different Methods in terms of various properties, such as computation speed, error rate during decryption, number of possible calculations without errors and others. In order to structure these tests and thus establish good comparability, a test concept will be created in this thesis. 
At the end, the following question should be answered: Are their practical advantages to transferring existing FHE systems from RLWE to MLWE or whether the associated increased computing effort nullifies the advantages again?}

\makeabstract{
  Abstract.
}

\maketoc[compact]

% ==========
%  Textteil
% ==========

\chapter{Introduction}
\label{Introduction}

% \section{Background}

In the early months of 1978, one of the most significant cryptographic systems, the RSA system \cite{RSA}, was published. With the advent of the Internet in the 1990s and the subsequent need for secure data transfer, it became one of the most widely used encryption schemes to date. In the subsequent period of slightly more than half a year, two of the authors of the RSA paper published a new concept, based on the RSA concept, which they designated ''privacy homomorphism`` \cite{Rivest1978}. This concept would later be known as homomorphic encryption. This is an encryption system whereby operations can be executed directly on encrypted data, eliminating the necessity of first decrypting it, running the operations, and then encrypting it again. Such a system would not only eliminate the necessity for decryption and encryption at the processing stage, it would also ensure that the plain text is not accessible to the party undertaking this processing. However, at the system's inception, only one operation was feasible: multiplication. To develop a system capable of general computing, the addition operation was necessary as a second operation, as these two operations enable the recreation of all other operations at the bit level. Unfortunately, the creation of a homomorphic encryption scheme with unlimited additions and multiplications, also known as full homomorphic encryption (FHE), proved to be a formidable challenge.\\
In 1994, Peter Shor published his algorithm \cite{Shor}, which describes how a quantum computer could factorize numbers in polynomial time. In contrast to classical computers, for which this problem is categorized as a hard problem. As the RSA cryptosystem is based on this exact issue being hard to solve, it would be possible to find the private key for any public key, thus undermining the cryptosystem's security. Fortunately, no quantum computer capable of such an operation was anywhere near availability at the time, so this problem remained theoretical.\\
Approximately a decade later, in 2005, O. Regev devised a novel mathematical framework, termed Learning with Error (LWE) \cite{Regev2005OnLL}, which enables the construction of new cryptosystems. This framework is based on an error term within a linear system of equations constructed on a lattice. The mathematical problem that he exploits for security is the hardness of the shortest vector problem (SVP). There are variants of this problem, called Ring-LWE, where a polynomial is used instead of a matrix, and M-LWE, which mixes Ring-LWE and the (Plain-)LWE together, resulting in matrices of polynomials. In 2009, Craig Gentry published the first full-homomorphic encryption scheme \cite{Gentry2009AFH}. This development prompted renewed optimism regarding the advancement of FHE schemes, as it became evident that the concept was indeed feasible. However, the primary challenge that remained was the issue of performance. To enhance the efficiency of this scheme, the initial version, which was based on the ideal lattice, was adapted to the R-LWE scheme. Over time, significant advancements have been made in the development of these FHE schemes, which are constructed on basis of R-LWE \cite{FHESurvey}. However, the primary challenge persists, namely the performance, which is frequently 1000s of times slower than operations on the plain text.\\
In recent years, there has been a resurgence of interest in quantum computers as various companies compete to develop the first practical and useful quantum computer \cite{googleQuantumComputing} \cite{ibmQuantumComputing}. Consequently, the performance of these computers has been steadily improving. If the promises made are accurate, it is possible that in 10 years, viable quantum computers will be available on the market. These computers could run Shor's Algorithm and thereby breach the security of RSA (and other) cryptosystems, potentially undermining the security of the internet as it currently stands. To circumvent such potential issues, the US National Institute of Standards and Technology (NIST) initiated an open competition in 2016, wherein individuals could submit novel cryptographic systems for analysis. Research teams from around the globe would then endeavor to identify vulnerabilities in these systems. In 2022, the NIST announced the first four winners \cite{nistAnouncement}, three of which were based on LWE. The two most recommended systems, CRYSTALS-Kyber \cite{CyrstalsKyber} and CRYSTALS-Dilithium \cite{crystalsDilithium}, are both based on M-LWE.


% \section{Goal of this Thesis}

In light of these recent advancements in M-LWE-based encryption schemes and the established R-LWE-based homomorphic encryption schemes, a question arises concerning the potential for integrating these two approaches: Is it possible to port the R-LWE-based homomorphic encryption schemes to M-LWE, and whether this results in an improvement in performance? Should the advantages outweigh the disadvantages, this would facilitate new synergies between the current endeavour to enhance the security of a post-quantum internet and the construction of efficient and dependable homomorphic encryption algorithms. For instance, enhanced and high-performing implementations or even hardware accelerators could be reused, thereby enhancing the efficacy of homomorphic encryption while simultaneously reducing the cost of development.

The thesis is divided into five principal sections. The first section of the thesis provides an introduction to the mathematical background, wherein all necessary mathematical operations will be explained in sufficient detail. Subsequently, the LWE problems will be described in greater detail, and a basic LWE-based encryption scheme will be constructed. The scheme is capable of functioning on Plain-, Ring-, and Module-LWE. Following this, homomorphic encryption will be outlined, and the LWE-based encryption scheme will be expanded to become homomorphic for all three LWE modes. These schemes will then be evaluated based on their memory usage, processing performance, and calculation depth. Ultimately, the aforementioned question will be addressed based on these findings.

\chapter{Mathematical Background}
\label{MathBack}

In order to understand the mathematical concepts behind the encryption algorithms described here, some basic concepts are explained here. However, you should have a basic knowledge of linear algebra and polynomial calculus.

\section{Lattice}

\section{Shortest Vector \& Closest Vector Problem}

\section{Polynomial Rings}
- explain also the modulus


\chapter{Learning with Errors}
\label{LWE}

In this section, we will take a closer look at the Learning with Errors (LWE) algorithm (also called Plain LWE) and its different versions, namely Ring LWE (R-LWE) and Module LWE (M-LWE).

\section{The Learning with Errors Problem}

In 2005, Regev first described the LWE problem \cite{Regev2005OnLL}. He also proved its hardness, but we won't go into those details here. The basic idea is to add an error vector to a linear system of equations. This makes the normally trivially solvable system surprisingly hard to solve.

In more mathematical terms, $\mathbb{Z}_q = \mathbb{Z}/n$, $A \in \mathbb{Z}_q^{n \times m}$, $s \in \mathbb{Z}_q^m$, $b \in \mathbb{Z}_q^n$, with which we can form the linear system of equations, where $A$ and $b$ are given and the vector $s$ represents the unknowns we want to retrieve

$$\textbf{A}\cdot \textbf{s} = \textbf{b}$$

Or written as system of equations it would look like:
$$
  \setlength\arraycolsep{0pt}
  \begin{array}{ c  >{{}}c<{{}} c  >{{}}c<{{}}  c >{{}}c<{{}}  c @{{}={}} c }
    A_{11}s_1 & + & A_{12}s_2 & + & \cdots & + & A_{1m}s_m & b_1    \\
    A_{21}s_1 & + & A_{22}s_2 & + & \cdots & + & A_{2m}s_m & b_1    \\
    \vdots    &   & \vdots    &   & \vdots &   & \vdots    & \vdots \\
    A_{n1}s_1 & + & A_{n2}s_2 & + & \cdots & + & A_{nm}s_m & b_n    \\
  \end{array}
$$

This can easily be solved with the Gaussian algorithm. But if we just add an error vector $e \in \mathbb{Z}_q^n$ with small values, it becomes surprisingly hard. The hardness is based on variants of the Shortest Vector Problem (SVP), which describes the hardness of finding the shortest vector in the lattice. This is easily solvable in smaller dimensions, but gets harder as the dimensions are increased. The equation after adding the small error term is the following:

$$\textbf{A}\cdot \textbf{s} + \textbf{e}= \textbf{b}$$

This is the main equation that all LWE problems are based on. Most of the differences will come from the ring or the dimensions used.

\info[inline]{Maybe going more into detail of the SVP or general the hardness of the LWE Problem?}

\section{LWE based encryption scheme}
\label{sec:Lwe-Encryption}

In this section, we will describe a simple LWE-based encryption scheme and how it can be converted to R-LWE and M-LWE. The following algorithm is loosely based on the Kyber \cite{CyrstalsKyber} scheme, with some simplifications.

All calculations are done in the ring $R = \mathbb{Z}_q$, where $q$ is the modulus. If values from $R$ are chosen uniformly, this is denoted by $x \leftarrow R$. Otherwise, if small values are chosen from $R$, this is written as $x \leftarrow \chi_R$. This can be done by choosing uniformly from a set of small numbers all in $R$ (e.g., ${-4,\ldots, 4}$ if $q$ is big enougth), or by choosing from an error distribution, such as the discrete Gaussian, as described in \cite{Regev2005OnLL}.

The following three algorithms describe the example schema. Algorithm \ref{alg: SampleLweKeyGen}, the key generation, describes how to generate the private key $pk$ and the secret key $sk$. It uses the LWE problem as described above. The secret key, which the owner should never share, is the vector $s$. The public key $pk$, which can be shared, consists of the transformation matrix $A$ and the transformed secret key plus the error $b$. The error $e$ is discarded after the computation of $b$. The values of $e$ and $s$ should be rather small, and $A$ is uniformly sampled from $R$.

\begin{algorithm}[htb]
  \begin{algorithmic}[1]
    \STATE $s \leftarrow \chi_R^n$
    \STATE $A \leftarrow R^{n \times n}$
    \STATE $e \leftarrow \chi_R^n$
    \STATE $b = A\cdot s+e$
    \RETURN $(pk:=(A, b), sk:=s )$
  \end{algorithmic}
  \caption{Sample LWE: KeyGen}
  \label{alg: SampleLweKeyGen}
\end{algorithm}

Algorithm \ref{alg: SampleLweEncryption}, the encryption, describes how to encrypt a message $m$ with the public key $pk$. The errors $e_1$ and $e_2$ are randomly sampled with small values and used to create more uncertainty around the message. The same message can therefore be decrypted with different errors and yield different values. This makes it harder for attackers to find patterns in the decryption. The idea behind $r$ is to select a subset of $A$ and $b$, since $~50\%$ of the values in $r$ will be $0$, meaning that these columns in $A$ and $b$ are irrelevant (multiplied by $0$). This helps to create more entropy between different encryptions, as a different subset of $A$ and $b$ will be used to encrypt each time.

The new values and the public key are used to calculate two values: $u$ and $v$. The first term, $u$, can be considered the cancel term for $b$, where the secret $s$ is missing. $v$ is the actual value term, which is composed of a subset of $b$ with some small error added and the scaled message $m'$. For the scaled message $m' = m*\left\lfloor q/2\right\rfloor$, the message is multiplied with the rounded down version of half the modulus. This operation results in the values of the message $0$ and $1$ in the ring being approximately as distant from each other as possible.

\begin{algorithm}[htb]
  \begin{algorithmic}[1]
    \REQUIRE $m \in \mathbb{Z}_2 = \{0, 1\}$, $pk = (A, b)$
    \STATE $r \leftarrow \{0, 1\}^n$
    \STATE $e_1 \leftarrow \chi_R^n$
    \STATE $u = A^T \cdot r + e_1$
    \STATE $e_2 \leftarrow \chi_R$
    \STATE $v = b^T \cdot r + e_2 + (m*\left\lfloor q/2\right\rfloor)$
    \RETURN $ct := (u, v)$
  \end{algorithmic}
  \caption{Sample LWE: Encryption}
  \label{alg: SampleLweEncryption}
\end{algorithm}

Algorithm \ref{alg: SampleLweDecryption}, the Decryption, describes how to decrypt an ciphertext $ct$ using the secret key $sk$.



\begin{algorithm}[htb]
  \begin{algorithmic}[1]
    \REQUIRE $ct = (u, v)$, $sk = s$
    \RETURN $\left\lfloor \frac{1}{\left\lfloor q/2\right\rfloor}*\left[v-s^T \cdot u\right]_q\right\rceil _2$
  \end{algorithmic}
  \caption{Sample LWE: Decryption}
  \label{alg: SampleLweDecryption}
\end{algorithm}


To get a better understanding, consider the following simplification of the term in algorithm \ref{alg: SampleLweDecryption}.

\begin{align*}
   & \left\lfloor \frac{1}{\left\lfloor q/2\right\rfloor}*\left[v-s^T \cdot u\right]_q\right\rceil _2                                                                                     \\
   & = \left\lfloor \frac{1}{\left\lfloor q/2\right\rfloor}*\left[b^T \cdot r + e_2 + (m*\left\lfloor q/2\right\rfloor)-s^T \cdot (A^T \cdot r + e_1)\right]_q \right\rceil _2             \\
   & = \left\lfloor \frac{1}{\left\lfloor q/2\right\rfloor}*\left[(As+e)^T \cdot r + e_2 + (m*\left\lfloor q/2\right\rfloor)-s^T A^T \cdot r - s^T e_1\right]_q \right\rceil _2            \\
   & = \left\lfloor \frac{1}{\left\lfloor q/2\right\rfloor}*\left[(As)^T \cdot r + e^Tr+ e_2 + (m*\left\lfloor q/2\right\rfloor)-(As)^T \cdot r - s^T e_1\right]_q\right\rceil _2         \\
   & = \left\lfloor \frac{1}{\left\lfloor q/2\right\rfloor}*\left[e^Tr+ e_2 + (m*\left\lfloor q/2\right\rfloor)- s^T e_1\right]_q\right\rceil _2                                          \\
   & = \left\lfloor \frac{e^Tr}{\left\lfloor q/2\right\rfloor}+ \frac{e_2 }{\left\lfloor q/2\right\rfloor}+ m - \frac{s^T e_1}{\left\lfloor q/2\right\rfloor}\right\rceil _2 \\
   & = \left\lfloor m' \right\rceil _2                                                                                                                                       \\
   & = m \in \{0,1\}
\end{align*}
As demonstrated by the calculation, by multiplying the cancellation term $u$ with the secret $s$, the transformation $(As)^T \cdot r$ in $v$ can be canceled out. This results in the message with some error values being added to it. The erroneous message will then be rounded, which will result in the original message. This process will only be successful if all error terms together are smaller than $\frac{q}{4}$. This is due to the fact that the possible values in the message are separated by a distance of $\frac{q}{2}$ from each other. Consequently, all values between $-\frac{q}{4}\mod q=\frac{3q}{4}$ and $\frac{q}{4}$ are rounded back to $0$, while all values between $\frac{q}{4}$ and $\frac{3q}{4}$ are rounded to $1$. Consequently, provided that the message (either $0$ or $\frac{q}{2}$) is not shifted by more than $\frac{q}{4}$, it will remain within the rounding area of the original message.

\info[inline]{Maybe add an image which shows the rounding with an clock}

The current definition of this algorithm allows only $1$ bit to be encoded at the time. This could be improved with some tricks, but for simplicity reasons we wont do that here. 

To observe the functioning of this algorithm in practice, please refer to the example calculation in the appendix, which can be found in Appendix \ref{app:PlainLweCalc}.

\section{Transforming LWE to R-LWE and M-LWE}

To transform the algorithms described above into Ring-LWE, only a few changes need to be made. Most importantly, a polynomial ring will be defined as $R = \mathbb{Z}[x]_q/(x^d+1)$, with the dimension $n=1$, which means that there are only polynomials. Instead of having a vector $r$, it will now be a polynomial in the ring $R$, where all coefficients are either $0$ or $1$. The message to be encrypted is also transformed into a polynomial in $R$ with the message bits being the coefficients of the polynomial. Consequently, $d$ bits can now be encoded in one message. As all values are now polynomials, polynomial arithmetic is used in place of matrix arithmetic. However, as previously stated, the polynomial arithmetic in the ring can also be transformed into matrix arithmetic. All equations stay the same and the structure of the Algorithms does not change.

An illustrative example of the three-step process for RLWE can be found in \ref{app:RlweExampleCalc}.

As next step, Ring-LWE can be transformed into Module-LWE. Todo this we only need to increase the dimensions, so that $n>1$. So instead of working with polynomials as in R-LWE, matrices and vectors of these polynomials will be used.

An example can found in Appendix \ref{app:MlweExampleCalc}

So in total, the only real differences between the Plain LWE, R-LWE and M-LWE are the dimensions and the ring. The computation itself stays the same. An summarized overview of the differences can be found in table \ref{table:LweDiffs}

\begin{table}[htbp]
  \caption[LWE variables shape comparison]{Comparison between the shapes of the variables for the different LWE Types}
  \label{table:LweDiffs}
  \centering
  \begin{tabular}{|c|l|l|l|}
    \hline
                & Plain LWE        & R-LWE                     & M-LWE                         \\
    \hline
    Ring $R$    & $\mathbb{Z}_q$   & $\mathbb{Z}[x]_q/(x^d+1)$ & $\mathbb{Z}[x]_q/(x^d+1)$     \\
    $A$         & $R^{n\times n}$  & $R$                       & $R^{n\times n}$               \\
    $s,b,e,e_1$ & $R^{n}$          & $R$                       & $R^{n}$                       \\
    $e_2$       & $R$              & $R$                       & $R$                           \\
    $r$         & $\mathbb{Z}_2^n$ & $\mathbb{Z}[x]_2/(x^d+1)$ & $(\mathbb{Z}[x]_2/(x^d+1))^n$ \\
    $m$         & $\mathbb{Z}_2$   & $\mathbb{Z}[x]_2/(x^d+1)$ & $\mathbb{Z}[x]_2/(x^d+1)$     \\
    \hline
  \end{tabular}
\end{table}

As the variables of the different LWE types have different dimensions, also the keys that need to be stored and shared and the messages have different dimensions. A comparison can be found in Table \ref{table:LweKeys}. As can be seen there, Plain LWE and R-LWE each depend only on one variable ($n$ or $d$ respectively), while M-LWE depends on both. This results in the secret key and private key for M-LWE being quite large, as they are always matrices or even 3D-tensors. In contrast, the secret key for Plain LWE and R-LWE is the same, but the dimensions are larger for the Plain LWE public key, which consists of a matrix and a vector, in contrast to two vectors in R-LWE.

One significant deficiency of Plain LWE is that only a single bit can be encoded at a time. Consequently, the resulting encrypted messages are of the form $\ell \times (\mathbb{Z}_q^{n}\times\mathbb{Z}_q)$, where $\ell$ is the number of bits that needs to be encoded. In contrast, R-LWE and M-LWE permit the encryption of $\ell$ bits in chunks of size $d$. If the number of bits, denoted by $\ell$, is a multiple of the dimension $d$, then the transformation of each message bit into two ciphertext integers is applicable to R-LWE. In contrast, for Plain LWE and M-LWE, each message bit is transformed into $n+1$ ciphertext integers.  
The security of Plain LWE is entirely reliant on the size of $n$, whereas in M-LWE, it is a combination of $n$ and $d$. In this context, it is possible to conclude that $n$ can be smaller in M-LWE than in Plain LWE. Consequently, it can be stated that R-LWE has the smallest cipher text dimension per bit, after which comes M-LWE, and the largest one has the Plain-LWE algorithm.

\begin{table}[htbp]
  \caption[LWE dimensions]{Comparison between the dimensions for keys and messages for the different LWE Types}
  \label{table:LweKeys}
  \centering
  \begin{tabular}{|c|l|l|l|}
    \hline
              & Plain LWE                                        & R-LWE                                                                              & M-LWE                                                                                     \\
    \hline
    $sk$      & $\mathbb{Z}_q^{n}$                               & $\mathbb{Z}_q^{d}$                                                                 & $\mathbb{Z}_q^{n\times d}$                                                                \\
    $pk$      & $\mathbb{Z}_q^{n\times n}\times\mathbb{Z}_q^{n}$ & $\mathbb{Z}_q^{d}\times \mathbb{Z}_q^{d}$                                          & $\mathbb{Z}_q^{n\times n \times d}\times\mathbb{Z}_q^{n \times d}$                        \\
    $m$       & $\ell \times \mathbb{Z}_2$                       & $\left\lceil \ell / d\right\rceil \times \mathbb{Z}_2^{d}$                         & $\left\lceil \ell / d\right\rceil \times\mathbb{Z}_2^{d}$                                 \\
    $ct$ & $\ell\times(\mathbb{Z}_q^{n}\times\mathbb{Z}_q)$ & $\left\lceil \ell / d\right\rceil \times(\mathbb{Z}_q^{d}\times \mathbb{Z}_q^{d})$ & $\left\lceil \ell / d\right\rceil \times(\mathbb{Z}_q^{n\times d}\times\mathbb{Z}_q^{d})$ \\
    \hline
  \end{tabular}
\end{table}

So in total it can be stated, that R-LWE has the smallest overall dimensions for the keys and for the ciphertext. Plain-LWE in contrast to M-LWE has a smaller key space, but the ciphertext space per bit is bigger.

% ===============
%  Verzeichnisse
% ===============

% Verzeichnisse mit einzeiligem Zeilenabstand
\singlespacing

% Literaturverzeichnis
\listofreferences

% Abbildungsverzeichnis einfügen
\listoffigures

% Tabellenverzeichnis einfügen
\listoftables

% Algorithmenverzeichnis einfügen
\listofalgorithms

% % Quelltextverzeichnis einfügen
\listoflistings

% =========
%  Appendix
% =========

\begin{appendices}

  \chapter{Example LWE-Calculations}
\label{app:Lwe-Calc}
The following calculations are based on the algorithms explained in Section \ref{sec:Lwe-Encryption}.

\section{Plain LWE}
\label{app:PlainLweCalc}
The following calculations should show the working of the Plain LWE encryption for the algorithms \ref{alg: SampleLweKeyGen} to \ref{alg: SampleLweDecryption}. The ring used for this calculations is defined as $R=\mathbb{Z}_{100}$ and $n=2$ for the dimensions. Starting first with the key generation:


\begin{align*}
  s  & = \begin{bmatrix}1 \\ 2 \end{bmatrix}
  A  = \begin{bmatrix}56 & 77 \\ 29 & 59 \end{bmatrix}
  e  = \begin{bmatrix}99 \\ 1 \end{bmatrix}  \\
  b  & = As+e                                \\
     & = \begin{bmatrix}
           56 & 77 \\
           29 & 59
         \end{bmatrix}
  \cdot
  \begin{bmatrix}
    1 \\
    2
  \end{bmatrix}
  +
  \begin{bmatrix}
    99 \\ 
    1 
  \end{bmatrix}
  \\
     & = 1
  \cdot
  \begin{bmatrix}
    56 \\
    29
  \end{bmatrix}
  + 2 
  \cdot
  \begin{bmatrix}
    77 \\ 
    59 
  \end{bmatrix}
  + 
  \begin{bmatrix}
    99 \\ 
    1 
  \end{bmatrix}                             \\
     & = \begin{bmatrix}
           309 \\
           148\end{bmatrix}_{100}              \\
     & = \begin{bmatrix}
           9 \\ 
           48 
         \end{bmatrix}                      \\
  sk & = s                                   \\
  pk & = (A, b) = \left (
  \begin{bmatrix}
      56 & 77  \\
      29 & 59 
    \end{bmatrix},
  \begin{bmatrix}
      9 \\
      48 
    \end{bmatrix} \right )                     \\
\end{align*}

With the secret and public key generated, the next step is to encrypt the message $m=1$ with the public key $pk$

\begin{align*}
  r       & = \begin{bmatrix}0 \\ 1 \end{bmatrix}
  e_1 = \begin{bmatrix}2 \\ 0 \end{bmatrix}
  e_2 = 99                                                          \\
  \\
  u       & = A^T \cdot r + e_1                                     \\
          & = \begin{bmatrix}
                56 & 77 \\
                29 & 59
              \end{bmatrix}^T
  \cdot
  \begin{bmatrix}
    0 \\
    1
  \end{bmatrix}
  +
  \begin{bmatrix}
    2 \\
    0
  \end{bmatrix}                                                    \\
          & = \begin{bmatrix}
                56 & 29 
                \\ 77 & 59 
              \end{bmatrix}
  \cdot 
  \begin{bmatrix}
    0 \\
    1 
  \end{bmatrix}
  +
  \begin{bmatrix}
    2 \\
    0
  \end{bmatrix}                                                    \\
          & = 0\cdot
  \begin{bmatrix}
    56 \\
    77
  \end{bmatrix}
  + 1 \cdot 
  \begin{bmatrix}
    29 \\
    59
  \end{bmatrix}
  +
  \begin{bmatrix}
    2 \\
    0
  \end{bmatrix}                                                    \\
          & = \begin{bmatrix}
                31 \\ 
                61 
              \end{bmatrix}_{100}
  = 
  \begin{bmatrix}
    31 \\
    61
  \end{bmatrix}                                                    \\
  \\
  v       & = b^T \cdot r + e_2 + (m*\left\lfloor q/2\right\rfloor) \\
          & =\begin{bmatrix}
               9 \\
               48
             \end{bmatrix}^T
  \cdot
  \begin{bmatrix}
    0 \\
    1
  \end{bmatrix}
  + 99 + 1 \cdot \left\lfloor 100/2\right\rfloor                    \\
          & =\begin{bmatrix}
               9 & 48
             \end{bmatrix}
  \cdot
  \begin{bmatrix}
    0 \\ 
    1 
  \end{bmatrix}
  + 99 + 50                                                         \\
          & = 9 \cdot 0 +48 \cdot 1 + 99 + 50                       \\
          & = 197_{100}                                             \\
          & = 97                                                    \\
  m_{enc} & = (u, v) = \left (
  \begin{bmatrix}
      31 \\
      61
    \end{bmatrix}, 97   \right )                                      \\
\end{align*}

Now the encrypted message $M_{enc}$ can be decrypted again, using the secret key $sk$:
\begin{align*}
  m & = \left\lfloor \frac{1}{\left\lfloor q/2\right\rfloor} *(v-s^T \cdot u)\right\rceil _2 \\
    & = \left\lfloor \frac{1}{\left\lfloor 100/2\right\rfloor} * \left (97-
  \begin{bmatrix}
      1 \\
      2
    \end{bmatrix}^T
  \cdot
  \begin{bmatrix}
      31 \\
      61
    \end{bmatrix} \right )\right\rceil _2                                                      \\
    & = \left\lfloor \frac{1}{50} * \left (97-
  \begin{bmatrix}
      1 & 2 
    \end{bmatrix}
  \cdot 
  \begin{bmatrix}
      31 \\ 
      61
    \end{bmatrix}\right )\right\rceil _2                                                       \\
    & = \left\lfloor \frac{1}{50} * (97-(31 \cdot 1 + 61 \cdot 2))\right\rceil _2            \\
    & = \left\lfloor \frac{1}{50} * (-56)_{100}\right\rceil _2                               \\
    & = \left\lfloor \frac{1}{50} * 44\right\rceil _2                                        \\
    & = \left\lfloor \frac{44}{50}\right\rceil _2  = \left\lfloor 0.88\right\rceil _2        \\
    & = 1                                                                                    \\
\end{align*}

\end{appendices}


% Abkürzungsverzeichnis
\listofacronyms

% Symbolverzeichnis
\listofsymbols

% falls ein anderer Glossar-Stil genutzt wird und die zweite Spalte zu schmal ist:
% \setlength{\glsdescwidth}{0.8\linewidth}

% Glossar einfügen
\printglossary

% Index einfügen
\printindex

% wieder auf 1½-fachen Zeilenabstand umschalten
\normalspacing

\end{document}
